\documentclass[14pt,a4paper,onecolumn]{article}
\usepackage[utf8]{inputenc}
\usepackage{amsmath}
\usepackage{amsfonts}
\usepackage{amssymb}
\usepackage{graphicx}
\usepackage{natbib}


\bibliographystyle{apalike}
\setlength{\parskip}{1.2em}
\setlength{\parindent}{0pt}
% \graphicspath{{./images/}}

\author{Ibrahim AbuBakr ElSeddiq}
\title{The predictive value of NTproBNP on postoperative outcome in patients undergoing offpump CABG}

\begin{document}

\maketitle

\tableofcontents
\clearpage

\listoffigures
\clearpage

\listoftables
\clearpage

\section{Introduction}

Coronary heart disease is the main cause of morbidity and mortality in developed countries, and the prevalence is increasing in developing countries.  Several studies have reported biomarker clusters which are associated with CHD.  The assessment of these biomarkers, alone or in combination, may improve the long-term prediction of mortality of first major cardiovascular event compared to conventional risk markers. \citep{Zethelius2008}

Brain type natriuretic peptide (BNP) is primarily produced by cardiac myocytes. Physiological effects of BNP are a peripheral vasodilatation and inhibition of renin-angiotensin production. \citep{DanielsandMaisel2007} The precursor peptide proBNP is split into the active hormone BNP and the N-terminal fragment (NT-proBNP).  Both BNP and NT-proBNP are established markers for cardiac failure.  NT-proBNP is also more stable, which makes its measurement more reliable. \citep{Thay-Hsiung2013} NT-proBNP was identified as a novel and important CHD biomarker, and has prognostic value in patients with stable CHD. \citep{Kragelund2005} A report from the BELSTRESS study suggested that NT-proBNP levels were a strong predictor of coronary events, even after adjustment for conventional risk factors. \citep{DeSutter2005} In patients with coronary artery disease increased BNP levels are associated with an increased  rate of myocardial infarction and cardiovascular death during mid-term follow-up. \citep{Schnabel2006}

However, Other pathologies such as exacerbated chronic obstructive pulmonary disease, atrial fibrillation, and myocarditis can cause elevated BNP levels. Additionally, higher NT-proBNP levels are associated with : female gender, impaired renal fuction, and older age.  Increased BNP levels are a prognostic marker associated with higher mortality in patients with myocardial infarction, cardiogenic shock, and pulmonary embolism. \citep{Rodseth2009}

\section{Review of Literature}

\subsection{Natriuretic Peptides}


Key cell biological observations that predicted the existence of natriuretic peptides were reported over fifty years ago. In 1956, Kisch found that guinea pig atrial, but not ventricular, cells contained highly developed Golgi networks, similar to those observed in secretory cells \citep{Kisch1956}. Jamieson and Palade reported that atrial, but not ventricular, myocytes contain spherical, electron opaque granules \citep{Jamieson1964}. At the same time, physiological experiments conducted by Henry and colleagues revealed that balloon distension of the atria correlated with increased urination in dogs \citep{Henry1964}; the heart is recognized not only as the pump of the circulatory system but also an endocrine organ\citep{deBold1979}.
In a seminal study published in 1981, de Bold and colleagues elegantly linked the seemingly disparate studies of Kisch and Henry by showing that atrial, but not ventricular, extracts contain a potent blood pressure decreasing component that works by stimulating renal sodium and water secretion \citep{deBold1981}. Thus, the first natriuretic peptide was discovered. Shortly after the publication of this landmark paper, a number of groups reported the purification and sequencing of atrial peptides of varying sizes that possessed natriuretic, diuretic, and/or smooth muscle relaxing activity \citep{Currie1984}  \citep{Flynn1983} \citep{Kangawa1984} \citep{Misono1984}.
 Several different names were given to these peptides such as atrial natriuretic factor, atriopeptin, cardionatrin, and cardiodilatin. However, atrial natriuretic peptide (ANP) is most often used to describe this peptide in the current literature. The second member of the family to be discovered, B-type natriuretic peptide (BNP), was originally called brain natriuretic peptide because it was purified and sequenced from porcine brain \citep{Sudoh1988}. However, subsequent studies found that it is more highly concentrated in cardiac ventricles of patients with heart failure \citep{Mukoyama1991}  \citep{Mukoyama1990}. Therefore, it is often described as B-type natriuretic peptide today. Finally, the third member of the family, C-type natriuretic peptide (CNP) \citep{Sudoh1990} was purified in 1991 from porcine brain extracts based on its ability to relax smooth muscle. All three members are similar in primary amino acid structure, contain a 17-residue disulfide ring, and are the products of separate genes.

BNP can be produced in both atria and ventricles, and is upregulated in failing ventricular myocardium. In response to increased myocardial stretch and wall stress, ventricular myocytes secret the pro-hormone pre-proBNP, which is then cleaved into biologically active BNP and the inactive byproduct N-terminal-proBNP (NT-proBNP). The biological actions of NPs are mediated through membrane-bound natriuretic peptide receptors (NPR) that are linked to a cyclic guanosine monophosphate-dependent signaling cascade, including NPR-A, which preferentially binds ANP and BNP, and NPR-B, which preferentially binds CNP. Elevated BNP levels have been demonstrated to responsed to increased angiotensin II and sympathetic tones. \citep{Iwanaga2006}

Elevated NPs levels can be also found in many circumstances involving LV dysfunction or hypertrophy; right ventricular (RV) dysfunction secondary to pulmonary diseases; cardiac inflammatory or infectious diseases; and endocrinology diseases and high output status without decreased left ventricular ejection fraction (EF), e.g., sepsis, renal failure, cirrhosis of liver, or intracranial pathologies. Even in the absence of significant clinical evidence of volume overload or LV dysfunction, markedly elevated NP levels can be found in patients with multiple comorbidities with certain degree of prognostic value. The causes and mechanisms of elevated NPs levels are summarized in Fig. \ref{NP_causes}. The potential clinical applications in the non-HF settings are summarized in Table \ref{NP_applications}.

\subsubsection{Atrial Natriuretic Peptide}

All natriuretic peptides are synthesized as preprohormones \ref{NP_structure}.
The human gene encoding ANP is called NPPA (GeneID 4878) and is located on chromosome 1. The resulting mRNA gives rise to a 151 amino acid polypeptide, known as preproANP. The first 25 amino acids constitute a signal sequence that is cleaved to yield a 126 amino acid peptide called proANP, which is the major form of ANP stored in the atrial granules \citep{Oikawa1984}.
Upon release from these granules, proANP is rapidly cleaved by corin, a transmembrane cardiac serine protease. Corin is highly expressed on the extracellular surface of atrial cardiomyocytes and has been shown to cleave proANP into the biologically active 28-amino acid form of ANP in vitro \citep{Yan2000}.
Mice lacking functional corin have dramatically reduced levels of fully processed ANP in their hearts and are mildly hypertensive \citep{Chan2005}.
Alternative processing of proANP in the kidney by an unknown protease results in a 32-amino acid peptide called urodilatin that contains four additional amino-terminal residues \citep{Forssmann1998}.
Disruption of the murine ANP gene, NPPA, results in marked hypertension, which was initially described as salt-sensitive \citep{John1995}, but later found not to be correlated with dietary salt intake \citep{John1996}.

ANP is well conserved between species \ref{NP_sequence}. The 28-amino acid mature form of ANP in humans and rats differs by only one amino acid at position 12, where the human peptide contains a methionine and the rat peptide contains an isoleucine. The mature, circulating form of ANP is identical in humans, chimps, dogs, pigs, horses, and sheep. The sequence of mature rat ANP is identical in mice and rabbits. Additionally, the entire length of the preproANP polypeptide, not just the carboxyl terminal biologically active end, is well conserved.

Release of proANP from the atrial granules is primarily stimulated by stretch of the atrial wall caused by increased intravascular volume \citep{Bilder1986} \citep{Edwards1988} \citep{Lang1985}, but pressor hormones also stimulate ANP release \citep{Ruskoaho2003}.
Upon secretion and cleavage into the mature peptide, ANP enters the coronary sinus and is distributed to its target organs via the circulation. Plasma levels of ANP are relatively low (10 fmol/ml), but in patients with congestive heart failure, circulating ANP levels are elevated from 10- to 30-fold \citep{Burnett1986} \citep{Cody1986}.
The plasma half-life of ANP in humans is approximately 2 min, with reported values falling between 1.7 and 3.1 min \citep{Nakao1986} \citep{Yandle1986}.
Degradation of the active ANP peptide occurs through the actions of neutral endopeptidase (NEP) \citep{Stephenson1987} \citep{Vanneste1988} as well as through binding to the natriuretic peptide clearance receptor (NPR-C).

NPR-C is a cell surface receptor that lacks guanylyl cyclase activity and controls the local concentrations of natriuretic peptides via constitutive receptor mediated internalization and degradation. Inhibiting NEP, increases the half-life of ANP both in vitro \citep{Stephenson1987} \citep{Yandle1989} and in vivo \citep{Yandle1989},
suggesting that NEP activity contributes to the rapid clearance of ANP. However, it is important to note that mice lacking functional NEP do not exhibit increased natriuretic peptide function \citep{Lu1995}.
In contrast, mice lacking NPR-C are hypotensive, exhibit skel overgrowth and have reduced ability to clear ANP compared to wild type mice, suggesting that NPR-C is also a physiologic regulator of circulating natriuretic peptide concentrations \citep{Matsukawa1999}.

ANP is secreted in response to:
    \begin{itemize}
        \item Stretching of the atrial wall \citep{Widmaier2008}
        \item Reduced Sympathetic stimulation of $\beta$-adrenoceptors
        \item Raised sodium concentration (hypernatremia), though sodium concentration is not the direct stimulus for increased ANP secretion. \citep{Widmaier2008}
        \item Endothelin, a potent vasoconstrictor
        \item exercise \citep{Kokkonen2002}
    \end{itemize}


\begin{figure}
    \includegraphics[scale=0.3]{./images/NP_causes.png}
    \caption{The causes and mechanisms of elevated natriuretic peptides levels.}
    \label{NP_causes}
\end{figure}

\begin{table}
    \caption{Potential Clinical Applications of Natriuretic Peptides in Selected Diseases}
    \begin{tabular}{|l|c|c|c|}
        \hline
        Diseases & Screening \footnote{Screening for the presence of cardiac dysfunction} &  & Prognosis \\
        \hline
        Heart failure \footnote{The best-established clinical application of these natriuretic peptides testing} & + & + & + \\
        Acute coronary syndrome & + & + & + \\
        Cardiac procedures & + & + & + \\
        Pulmonary embolism & + & + & + \\
        Pulmonary hypertension & + & + & - \\
        Chronic lung diseases & + & + & + \\
        Valvular heart diseases & + & + & N/A \\
        Cardiac dysrhythmia & + & + & +/- \\
        Cardiac inflammatory or infectious diseases & + & + & N/A \\
        Cardiogenic syncope & + & N/A & N/A \\
        Sleep apnea & + & + & N/A \\
        Hypertension & + & + & N/A \\
        Sepsis & + & + & + \\
        Renal failure & + & + & + \\
        Cirrhosis of liver & + & + & + \\
        Hyperthyroidism & + & + & N/A \\
        Intracranial pathologies & + & + & + \\
        Epilepsy / Seizures & + & - & - \\
        Carbone monoxide poisoning & + & N/A & N/A \\
        \hline
    \end{tabular}
    \label{NP_applications}
\end{table}


\subsubsection{B-Type Natriuretic Peptide}
BNP was initially purified and sequenced from extracts of porcine brain tissue and hence it was named “brain natriuretic peptide” \citep{Sudoh1988}. Subsequently, BNP was found at much higher concentrations in cardiac tissues \citep{Mukoyama1991} \citep{Mukoyama1990}. The gene encoding human BNP, NPPB (GeneID 4879), is located on chromosome 1.

Like NPPA , the NPPB gene consists of 3 exons and 2 introns \citep{Ogawa1994a}. PreproBNP is 134 amino acids in length, consisting of a 26 amino acid signal sequence followed by 108 amino acids that constitute proBNP. Unlike preproANP, which has high species homology throughout the entire polypeptide sequence, preproBNP sequences in mammals only have high homology at the amino and carboxyl terminal ends of the polypeptide. For example, the homology of canine preproBNP to human preproBNP is 53\%, whereas the homology for preproANP between these species is 85\%. This lower level of homology gives rise to differing lengths of the active, circulating BNP between mammalian species. For example, in humans and pigs circulating BNP is 32 amino acids in length, while in rats and mice the circulating form is 45 amino acids. The peptidase that cleaves proBNP to its active form has not been identified, but corin is a reasonable suspect.

Although low levels of BNP are stored with ANP in atrial granules, BNP is found at greater concentrations in cardiac ventricles. In this tissue, BNP is not stored in granules, but rather transcribed as needed in response to cardiac stress states such as volume overload. The transcription of BNP is under the regulatory control of GATA4, a transcription factor \citep{Grepin1994} \citep{Thuerauf1994}. In normal human subjects, plasma concentrations of BNP are very low (1 fmol/ml), but in response to congestive heart failure, circulating concentrations of BNP are dramatically elevated \citep{Mukoyama1991} \citep{Mukoyama1990}.

BNP is eliminated by binding to the NPR-C or degradation by NEP on endothelial cells, smooth muscle cells, cardiac myocytes, renal epithelium, and fibroblasts. NT-proBNP is cleared mainly by the kidney.\citep{Schrier1999}  Compared to ANP, circulating BNP has a significantly longer half-life of around 20 min in humans \citep{Mukoyama1991} \citep{Mukoyama1990}; the half-life of NT-proBNP is about 60-90 minutes and would be expected to be longer in the setting of renal dysfunction.. Unlike ANP, BNP is not initially cleaved by NEP. Instead, the first six amino-terminal amino acids of BNP are first cleaved by the metalloprotease, meprin A in the kidney brush border, which then allows further degradation by NEP \citep{Pankow2007}. Obese patients (especially those who have body mass index greater than 30) tend to have lower BNP levels than others. Neural endopeptidases that can be secreted by adipose tissue may be related to increased BNP clearance in obese patients.\citep{Young2004}

Plastic tubes containing ethylenedinitrolotetraacetic acid (EDTA) are desirable for BNP determination and refrigeration is required if the interval between blood collection and analysis is over 4 hours; whereas NT-proBNP can be measured in both serum or plasma, collected in glass or plastic tubes, and has no significant loss of immunoreactivity after 48 hours at room temperature. Although these existing BNP assays correlate closely, BNP assays are not currently analytical equivalent due to lack of assay standardization.\citep{Omland2008}  A multicenter colloborative proficiency testing study conducted in 90 Italian laboratories had demonstrates that there are significant differences in analytical characteristics and measured values among the most popular commercial methods for BNP and NT-proBNP. Thus, clinicians should be very careful when comparing results obtained by laboratories that use different methods.\citep{Prontera2009}

Both knockout and overexpression models of NPPB have been generated in mice. The knockout model of Nppb was created by targeted deletion of exons 1 and 2 \citep{Tamura2000}. In contrast to ANP knockout mice, Nppb −/− mice showed no signs of systemic hypertension or ventricular hypertrophy on standard or high salt diets. However, Nppb −/− mice had ventricular fibrotic lesions that increased in size and number in response to pressure overload, compared to wild type animals. Thus, these studies suggest that BNP is not a regulator of blood pressure, at least in mice. Rather, it is a paracrine regulator of cardiac remodeling. In murine overexpression models of BNP, blood pressure reduction of 20 mmHg was seen with 10- to 100-fold increases in plasma BNP levels \citep{Ogawa1994a}. Interestingly, these mice had marked increases in long bone length compared to their wild-type littermates, which most likely resulted from overactivation of NPR-B, the receptor of CNP

\subsubsection{C-Type Natriuretic Peptide}
C-type natriuretic peptide (CNP) was initially purified and sequenced from porcine brain extracts \citep{Sudoh1990}. It is the most highly expressed natriuretic peptide in the brain but is also highly expressed in chondrocytes and endothelial cells. Unlike ANP and BNP, the human gene encoding CNP, NPPC (GeneID 4880), is not located on chromosome 1 but on chromosome 2 \citep{Ogawa1994b}. Another difference of NPPC is that it consists of only 2 exons and 1 intron. In the murine genome, Nppc is also separated spatially from both ANP and BNP being located on chromosome 2 \citep{Ogawa1994b}.

NPPC encodes a polypeptide of 126 amino acids, with a 23 amino acid signal sequence followed by a 103 amino acid proCNP \citep{Tawaragi1991}. PreproCNP shows remarkable homology between species, even more so than preproANP. The preproCNP polypeptides of mammalian species show 99, 96, 91, and 94\% homology to the human form in chimpanzees, dogs, mice, and rats, respectively. Perhaps even more telling is that the circulating 22 amino acid carboxyl terminal form of CNP is absolutely identical in all of the above species. Processing of proCNP to its mature form may occur through the action of the intracellular serine endoprotease, furin. In vitro, furin cleaves the 103 amino acid proCNP into a 53 amino acid carboxyl-terminal biologically active peptide \citep{Wu2003}. This 53 amino acid form of CNP (CNP-53) is the major active form of CNP, at the tissue level \citep{Brown1997}. However, in the systemic circulation, a shorter 22 amino acid form dominates (CNP-22). The protease responsible for this cleavage is not known. Importantly, CNP-53 and CNP-22 appear to bind and activate their cognate receptor, NPR-B, equally well \citep{Yeung1996}.

CNP is not stored in granules and its secretion is increased by growth factors \citep{Suga1993} \citep{Suga1992b} and sheer stress \citep{Chun1997} in cultured endothelial cells. CNP expression in neointimal vascular smooth muscle cells is increased in response to vascular injury \citep{Brown1997}. In normal human subjects, mean CNP concentration is very low (1 fmol/ml). It is elevated in patients with congestive heart failure, although to a much lower extent than ANP and BNP \citep{Charles2006} \citep{Del-Ry2005} \citep{Kalra2003}.

The clearance of CNP-22 in human plasma is very rapid, with a calculated half-life of 2.6 min \citep{Hunt1994}. Like ANP, CNP has been shown in vitro to be inactivated by neutral endopeptidase \citep{Kenny1993} and is internalized and degraded by NPR-C. The NPR-C route of CNP degradation is especially important at the tissue level, as can be seen in NPR-C knockout models. Mice lacking functional NPR-C receptors have disproportionately long bones, most likely due to the failed clearance of CNP from specific regions in the growth plate \citep{Jaubert1999} \citep{Matsukawa1999}.

Both over and under expression models of CNP have been created in mice. Knockout animals were generated by disruption of exons 1 and 2 of Nppc \citep{Chusho2001}. The Nppc −/− mice are dwarfs due to impaired endochondral ossification and have severely reduced life spans. Targeted expression of CNP in the growth plate chondrocytes rescued the skeletal defect of Nppc −/− mice and prolonged their survival. Importantly, the Nppc transgene partially rescued skeletal growth in a murine model of achondroplasia caused by a gain of function mutation in the fibroblast growth factor receptor 3 ( Fgf r3 ach ), which accounts for most types of human dwarfism and may indicate that CNP analogs could be used to treat various forms of human dwarfism \citep{Yasoda2004}. A recent spontaneous point mutation in Nppc was characterized by a dwarfism phenotype similar to that seen in the Nppc −/− mice \citep{Jiao2007}. The mutation resulted in a missense mutation in proCNP, which is one of the highly conserved residues in the cysteine ring found in all natriuretic peptides.
Our recent data indicate that the mutant CNP has reduced ability to bind and activate NPR-B \citep{Yoder2008}. Finally, balanced translocations of chromosome 2, which separate putative negative transcriptional regulators from the CNP promoter, cause abnormal elevations in CNP and associated Marfanoid-like skeletal overgrowth in humans \citep{Bocciardi2007} \citep{Moncla2007}. Hence, moderate elevations of CNP are correlated with demonstrative skeletal overgrowth.

\subsection{Natriuretic Peptide Receptors}
There are three known natriuretic peptide binding proteins. All members contain a relatively large (\~450 amino acid) extracellular ligand binding domain and a single membrane-spanning region of about 20 residues. Natriuretic peptide receptors A and B contain an equally large intracellular domain consisting of a so-called kinase homology domain, dimerization domain, and carboxyl-terminal guanylyl cyclase domain. Thus, NPR-A and NPR-B signal by catalyzing the synthesis of the intracellular signaling molecule cGMP. In contrast, NPR-C only contains a 37 residue intracellular domain and lacks guanylyl cyclase activity. It primarily controls local natriuretic peptide concentrations via receptor-mediated internalization and degradation, although many groups have reported signaling functions for NPR-C as well \citep{Rose2008}.

\subsubsection{Natriuretic Peptide Receptor-A}
Natriuretic peptide receptor-A (NPR-A) is the principal receptor of atrial natriuretic peptide (ANP) and B-type natriuretic peptide (BNP). Its extracellular domain contains three intramolecular disulfide bonds and five N-linked glycosylation sites \citep{Miyagi2000}.
NPR-A exists as a homodimer or homotetramer in its native state, and oligomerization is ligand-independent \citep{Chinkers1992} \citep{Iwata1991}, although ligand binding does bring the juxtamembrane regions of each monomer closer together \citep{Labrecque2001}.
NPR-A binds natriuretic peptides at a stoichiometry of 2:1 with a rank natriuretic peptide preference of: ANP $\ge$ BNP $>$ CNP \citep{Bennett1991} \citep{Koller1991} \citep{Suga1992a}.  % The human NPR-A gene is located on chromosome 1q21–22 and consists of 22 exons and 21 introns within 16 kilobases \citep{Lowe 1990} \citep{Takahashi1998}. The murine NPR-A gene, Npr1 , is located on chromosome 3. Under basal conditions, NPR-A is phosphorylated on four serine and two threonine residues located in the N-terminal portion of the kinase homology domain \citep{Potter1998b}.

Phosphorylation is essential for activation of NPR-A and dephosphorylation is a mechanism of desensitization in response to prolonged ANP exposure or protein kinase C activation \citep{Potter1992} \citep{Potter1994}. Although ATP increases ANP-dependent guanylyl cyclase activity, the mechanism for this effect is debatable \citep{Antos2005} \citep{Antos2007} \citep{Burczynska2007} \citep{Joubert2005}.
Recent data indicate that ATP reduces the Km for NPR-A \citep{Antos2007}. NPR-A internalization and degradation is also controversial. One group consistently reports that the majority of internalized ANP-NPR-A complexes are degraded via a lysosomal pathway with a small portion returning intact to the plasma membrane \citep{Pandey2002}. Meanwhile, studies in primary kidney and Chinese Hamster ovary indicate that NPR-A is a membrane resident protein that does not undergo acute internalization and degradation \citep{Fan2005} \citep{Koh1992} \citep{Vieira2001}.

NPR-A and/or its mRNA is expressed in kidney, lung, adipose, adrenal, brain, heart, testis, and vascular smooth muscle tissue \citep{Goy2001} \citep{Lowe1989} \citep{Nagase1997} \citep{Wilcox1991}. NPR-A null mice exhibit chronic salt-resistant hypertension and cardiac hypertrophy and fibrosis \citep{Kuhn2002} \citep{Lopez1995} \citep{Oliver1997}.
A deletion in the human NPR-A gene was identified in nine Japanese individuals, of which eight had essential hypertension; the normotensive individual with the altered allele had left ventricular hypertrophy \citep{Nakayama2000}.  %To our knowledge, this study has not been repeated.

\subsubsection{Natriuretic Peptide Receptor-B}
Natriuretic peptide receptor-B (NPR-B) is the principal receptor of C-type natriuretic peptide (CNP) and exhibits similar topology, glycosylation, and intramolecular disulfide bonding patterns as NPR-A. The extracellular and intracellular regions of rat NPR-B are 43 and 78\% identical to rat NPR-A at the amino acid level, respectively \citep{Schulz1989}. NPR-B binds natriuretic peptides with a selectivity preference of CNP $>$ ANP $\ge$ BNP \citep{Bennett1991} \citep{Koller1991} \citep{Suga1992a}. %The human NPR-B gene is located on chromosome 9p12–21 and the murine version, Npr2 , is located on chromosome 4 \citep{Lowe1990}.

% Four serine and one threonine phosphorylation sites have been identified within the amino-terminal portion of NPR-B's kinase homology domain \citep{Potter1998a}, and receptor dephosphorylation has been shown to mediate desensitization in response to prolonged CNP exposure, protein kinase C activation, and intracellular calcium elevations \citep{Potter1998} \citep{Potter2000} \citep{Potthast2004}. ATP increases the guanylyl cyclase activity of NPR-B, by decreasing its Michaelis constant \citep{Antos2007}.
NPR-B and/or its mRNA is expressed in bone, brain, fibroblasts, heart, kidney, liver, lung, uterine, and vascular smooth muscle tissue \citep{Bryan2006} \citep{Chrisman1993} \citep{Dickey2007} \citep{Herman1996} \citep{Langub1995}. Mice with a targeted disruption of the NPR-B gene, display dwarfism and  female sterility \citep{Tamura2004}.
% The achondroplastic (cn/cn) mouse has a spontaneous mutation in NPR-B resulting in the substitution of a highly conserved Leu with Arg in the guanylyl cyclase domain, which inactivates the enzyme. Endochondral ossification is disrupted in mice with two defective alleles, leading to dwarfism \citep{Tsuji2005}.  However, these mice exhibit no fertility defect.
NPR-B dominant negative mutant transgenic rats have also been generated \citep{Langenickel2006}. In addition to mild growth retardation of the long bones, the rats displayed progressive, blood pressure-independent cardiac hypertrophy and an elevated heart rate. Consistent with a prominent role for CNP in the heart, NPR-B, not NPR-A, is the most active natriuretic peptide receptor in the failed heart \citep{Dickey2007}. Homologous loss-of-function mutations in human NPR-B result in a rare form of dwarfism called acromesomelic dysplasia, type Maroteaux (AMDM) \citep{Bartels2004}. Individuals with single defect NPR-B alleles are statistically shorter than the average person \citep{Olney 2006}.

\subsubsection{Natriuretic Peptide Receptor-C}
Natriuretic peptide receptor-C (NPR-C) consists of a large extracellular ligand-binding domain that is approximately 30–35\% identical to NPR-A and NPR-B, a single membrane-spanning region, but only 37 intracellular amino acids \citep{Chang1989} \citep{Fuller1988} \citep{Porter1990}. Unlike NPR-A and NPR-B, it contains one or two juxtamembrane intermolecular disulfide bonds. Hence, it is a disulfide-linked dimer. It has no known enzymatic activity but has been  suggested to signal in a G protein-dependent manner \citep{Rose2008}.
It contains three known N-linked extracellular glycosylation sites \citep{Stults1994} and binds natriuretic peptides with a stoichiometry of two molecules of receptor to one molecule of ligand \citep{Ammarguellat2001}. Its ligand selectivity preference is: ANP $>$ CNP $\ge$ BNP \citep{Bennett1991} \citep{Suga1992a}.

% Compared to NPR-A and NPR-B, NPR-C has much less stringent specificity for structural variants of ANP and will bind with high affinity to ring-deleted ANP analogs \citep{Maack1987}.
The main function of NPR-C, also known as the clearance receptor, is to clear circulating natriuretic peptides through the process of receptor-mediated internalization and degradation \citep{Koh1992} \citep{Nussenzveig1990}. Internalization of NPR-C occurs in the absence of ligand; thus, this is a constitutive process \citep{Nussenzveig1990}. Osteocrin, an endogenous protein with limited homology to members of the natriuretic peptide family, binds NPR-C, but not NPR- A or NPR-B \citep{Moffatt2007}. Osteocrin is thought to compete with CNP for binding to NPR-C in bone, and therefore, increase local CNP levels during critical periods for bone development \citep{Moffatt2007}.

% The human NPR-C gene is located on chromosome 5p13–14 and contains 8 exons and 7 introns spanning more than 65  kilobases \citep{Lowe1990} \citep{Rahmutula2002}. Npr3 , the murine version of the NPR-C gene, is located on chromosome 15. A splice variant of NPR-C containing an additional cysteine residue has also been identified and characterized from bovine lung \citep{Mizuno1993}.
NPR-C is the most widely and abundantly expressed natriuretic peptide receptor; for example, it constitutes \~94\% of the total ANP binding sites in endothelial cells \citep{Leitman1986}. NPR-C and/or its mRNA is expressed in adrenal, brain, heart, kidney, mesentery, and vascular smooth muscle tissue \citep{Nagase1997} \citep{Porter1990} \citep{Suga1992c} \citep{Wilcox1991}.
NPR-C knockout mice exhibit increased ANP half-lives, long bone overgrowth, hypotension, mild diuresis, dilute urine, and blood volume depletion  \citep{Matsukawa1999}. Mouse strains containing chemically induced loss-of-function mutations in the extracellular domain of NPR-C display skeletal overgrowth from endochondral ossification defects as well \citep{Jaubert1999}.

\subsection{Physiologic Effects of Natriuretic Peptides}


\subsubsection{Natriuretic Peptide Effects on Blood Pressure}
ANP binding to NPR-A is a key-signaling pathway, which regulates normal homeostatic blood pressure. This is clearly demonstrated in mice lacking ANP or its receptor NPR-A, which have blood pressures that are elevated 20–40mm mercury, compared to control mice \citep{John1995} \citep{John1996} \citep{Lopez1995} \citep{Oliver1997}.
The link between NPR-A and blood pressure in mice is particularly strong because Smithies and colleagues demonstrated that NPR-A copy number is inversely related to blood pressure in a  remarkably linear manner \citep{Oliver1998}.
Conversely, blood pressures in transgenic mice overexpressing ANP or BNP are substantially decreased \citep{Ogawa1994a} \citep{Steinhelper1990}. Although infusion of supraphysiological levels of CNP into animals acutely decreases blood pressure \citep{Clavell1993} \citep{Sudoh1990}, mice lacking functional CNP or NPR-B are normotensive \citep{Chusho2001} \citep{Tamura2004}, suggesting that the CNP/NPR-B pathway is not a fundamental regulator of basal blood pressure in mice.

NPR-A dependent decreases in blood pressure are achieved through natriuresis and diuresis, vasorelaxation, increased endothelium permeability, and antagonism of the renin-angiotensin system. Classic experiments showed that atrial extract infusions resulted in rapid renal excretion of water and sodium \citep{deBold1981}. Studies by Garbers and colleagues indicated that the renal response requires NPR-A because mice lacking this receptor do not respond to ANP, BNP, or to acute volume expansion \citep{Kishimoto1996}. Similar studies found that NPR-A was also required for ANP- or BNP-dependent vasorelaxation in mice \citep{Lopez1997}. Physiological experiments involving mice with severe reductions of NPR-A in vascular smooth muscle cells demonstrated that while smooth muscle NPR-A is required for acute ANP- or BNP-dependent vasorelaxation, this response does not play a significant role in controlling chronic blood pressure \citep{Holtwick2002}.

The ability of the ANP/NPR-A pathway to increase  endothelial permeability is supported by the observation that hematocrit levels are elevated prior to urination and are preserved in nephrectomized animals \citep{Almeida1986} \citep{Fluckiger1986} \citep{Richards1988}.
 Furthermore, mice with genetically engineered reductions of NPR-A in vascular endothelium exhibit volume expansion, hypertension, and reduced albumin clearance from the vascular system \citep{Sabrane2005}.

\subsubsection{Effects of Natriuretic Peptides on Cardiac Hypertrophy and Fibrosis}
% In addition to regulating blood pressure, natriuretic peptides inhibit cardiac hypertrophy and remodeling. Hypertrophy is regulated by ANP and NPR-A, whereas remodeling is regulated by both the ANP/BNP/NPR-A and the CNP/NPR-B pathways.

Although prolonged hypertension can cause hypertrophy, the level of hypertrophy in NPR-A deficient mice is significantly greater than that observed in other genetic models that cause similar levels of hypertension, suggesting that NPR-A elicits a local growth inhibitory signal in the heart. Data for this idea was initially shown in NPR-A knockout mice, which have enlarged hearts even when effectively treated with antihypertensive drugs from birth \citep{Knowles2001}. Additional studies determined that transgenic re-expression of NPR-A in the hearts of NPR-A −/− mice reduced cardiomyocyte size without affecting heart rate or blood pressure \citep{Kishimoto2001}.
Finally, mice with reduced cardiomyocyte expression of NPR-A exhibited moderate hypertrophy even though they were slightly hypotensive \citep{Holtwick2003} \citep{Patel2005}. In terms of natriuretic peptides, mice lacking ANP have larger hearts, whereas mice transgenically overexpressing ANP have smaller hearts \citep{Barbee1994} \citep{Steinhelper1990}.
In contrast, targeted deletion of BNP resulted in normotensive mice with normal heart size but with increased ventricular fibrosis – especially when subjected to pressure overload \citep{Tamura2000}. Thus, genetic studies in mice strongly support a role for ANP activation of NPR-A in the local inhibition of cardiac hypertrophy and BNP activation of NPR-A in the inhibition of cardiac fibrosis.

Data supporting a role for the CNP/NPR-B pathway in cardiac remodeling has been reported. Although NPR-B inactivation mutations in mice have not been shown to cause hypertrophy \citep{Tamura2004} \citep{Tsuji2005}, transgenic rats expressing a dominant negative form of NPR-B exhibit mild blood pressure-independent cardiac hypertrophy and increased heart rate \citep{Langenickel2006}.
In addition, CNP infusion was shown to reduce cardiac remodeling in response to experimentally induced myocardial infarction in rats, and transgenic expression of CNP improved outcomes in mice subjected to ischemia/reperfusion injury or myocardial infarction \citep{Wang2007}.

\subsubsection{Effects of CNP and NPR-B on Bone Growth}
The most obvious function of the CNP/NPR-B pathway is to stimulate long bone growth. Though undetectable at birth, mice lacking functional CNP or NPR-B develop dwarfism due to impaired endochondrial ossification \citep{Chusho2001} \citep{Tamura2001} \citep{Tsuji2005}.
Conversely, transgenic CNP overexpression or reduced degradation of CNP due to loss-of-function mutations in NPR-C result in skeletal overgrowth \citep{Jaubert1999} \citep{Matsukawa1999}  \citep{Yasoda2004}. Growth plate histology reveals that the endochondral proliferative and hypertrophic zones are reduced in mice with impaired CNP or NPR-B signaling, whereas overexpressing mice have enlarged growth plates \citep{Chusho2001} \citep{Tamura2004} \citep{Yasoda2004}.

One cGMP effector involved in the long bone growth pathway is cGMP-dependent protein kinase II, also known as PKGII or cGKII. Loss-of-function mutations in the mouse or rat gene that encodes this kinase also cause dwarfism \citep{Chikuda2004} \citep{Pfeifer1996}. Interestingly, the growth plates of rodents with defective cGKII are enlarged, which differs from the diminished growth plates seen in the CNP or NPR-B deleted mice, suggesting that a cGKII-independent pathway is also involved in CNP-dependent long bone growth.

Humans with two loss-of-function alleles for NPR-B suffer from a rare type of autosomal recessive dwarfism, called acromesomelic dysplasia, type Maroteaux \citep{Bartels2004}. These individuals are characterized by disproportionate limb to torso ratios that are only obvious a year or more after birth. Interestingly, although single copy carriers of a nonfunctional NPR-B allele do not suffer from disease, they are statistically shorter than comparable individuals with two wild type NPR-B alleles \citep{Olney2006}. Thus, it is possible that NPR-B mutations could have a significant effect on the stature of the general population.

\subsection{Therapeutics of Natriuretic Peptides}
% The Encyclopedia Britannica defines therapeutics as “the treatment and care of a patient for the purpose of both preventing and combating disease”. By these criteria, natriuretic peptides have already found their way into the clinical arsenal – especially ANP and BNP.
Measurement of serum BNP levels is used in the clinic as a diagnostic indicator for heart failure, and synthetic forms of both ANP and BNP have been approved in some countries for the treatment of heart failure \citep{Gardner2003}. The extent of their usefulness, however, has come under question due to their limited renal actions, and trials are underway to determine the most effective use of these peptides. In this section, we will explore the history of both synthetic ANP and BNP as therapeutic agents. %TO-DO search for more recent data

\subsubsection{Synthetic ANP (Anaritide and Carperitide)}
The natriuretic peptide field emerged with the key discovery that specific peptides present in atrial extracts cause natriuresis and diuresis. Studies revealed that the mature form of ANP is a 28-amino-acid peptide and that smaller versions are degradation products that maintain various levels of activity. The most widely studied of these is the 25-amino acid peptide lacking the first three amino-terminal residues. This peptide is referred to as ANF IV and its synthetic form is called anaritide.

Since the activities of the 25-amino acid and mature 28-amino acid peptide were similar, many studies were conducted with the smaller peptide. Studies by Cody and colleagues indicated that infusion of anaritide in healthy male volunteers resulted in natriuresis, diuresis, and reduction in systolic blood pressure; however, in seven patients with congestive heart failure, the changes in urine volume and sodium excretion were minimal \citep{Cody1986}. Saito and colleagues observed a similar lack of diuresis and natriuresis, when congestive heart failure patients were infused with the mature form of ANP \citep{Saito1987}.
Meanwhile, others acknowledged the renal hyporesponsiveness to anaritide in congestive heart failure patients, but indicated that the renal parameters did show a statistically significant increase in larger patient samples \citep{Fifer1990}. In Japan, clinical studies on the effectiveness of mature ANP continued; and in 1995, synthetic full length ANP (carperitide) was approved for the treatment of acute decompensated heart failure. In the United States, clinical use of BNP, not ANP, was explored for the treatment of heart failure due to its larger renal responsiveness, and possibly due to unique patient opportunities. %TO-DO lacking reference!

Investigations were also initiated to study the effectiveness of ANP in the treatment of human renal disease. Specifically, trials were conducted to evaluate the ability of anaritide infusion to reduce the need for dialysis in patients with acute tubular necrosis. The initial study with 53 patients suggested a positive outcome for patients receiving anaritide because they had increased creatinine clearance and a decreased need for dialysis \citep{Rahman1994}. This led to the formation of a multicenter placebo-controlled clinical trial in 504 patients with acute tubular necrosis. While 24-h infusion of anaritide did not improve the overall survival of the patients without dialysis, it appeared that a subset of patients might have benefited \citep{Allgren1997}.
Thus, a second trial was conducted in patients with oliguric acute renal failure. However, this 222 patient trial indicated no statistically significant benefit of anaritide in dialysis-free survival \citep{Lewis2000}. Both trials remarked on the severe hypotension that often occurred as a result of the anaritide infusion. In fact, it is this severe hypotension that appears to be limiting the utility of anaritide or nesiritide as a therapy for either heart failure or renal disease. The authors stated in their discussion, “it is possible that if this hypotension could have been avoided, anaritide would have been efficacious” \citep{Lewis2000}.  Anaritide was also investigated for its ability to prevent radiocontrast-induced nephropathy. However, in a 247 person clinical trial anaritide along with hydration was no more effective at preventing radiocontrast- induced nephropathy than hydration alone \citep{Kurnik1998}.

Finally, in 2004, studies conducted in Sweden compared the ability of the loop diuretic, furosemide, or mature ANP (1–28) to increase GFR, renal blood flow, and reduce renal oxygen consumption in patients with acute renal failure. They concluded that furosemide was a more effective agent \citep{Sward2005}. Therefore, despite its potent natriuretic and diuretic effects in normal, healthy subjects, clinical studies conducted to date indicate little or no therapeutic benefit of ANP analogs in the successful treatment of renal disease.

\subsubsection{Synthetic BNP (Nesiritide)}
Given the natriuretic effects of ANP, the related peptide BNP, was assumed to elicit a similar response. McGregor and colleagues demonstrated that administration of porcine BNP resulted in a natriuretic response and an increase in urinary excretion of cGMP \citep{McGregor1990}. Yoshimura and colleagues reported the same response in healthy volunteers to infusion of human BNP \citep{Yoshimura1991}. Furthermore, patients with congestive heart failure also responded to infusion of BNP.

The effectiveness of 24-h infusion of nesiritide to patients with congestive heart failure was examined in a multicenter, placebo-controlled trial. The peptide resulted in a reduction of both preload and afterload resulting in an increase in stroke volume and cardiac output \citep{Mills1999}.
The results of a second multicenter trial, called the Vasodilation in the Management of Acute Congestive Heart Failure (VMAC) study, compared the effects of the addition of nitroglycerin or nesiritide versus placebo to standard therapy. The group treated with nesiritide had improved dyspnea after 3 h treatment, while there was no difference in the other groups. The nitroglycerin group reported more adverse effects than the nesiritide group. Additionally, patients receiving nesiritide had less adverse cardiovascular effects at either the 0.015 or 0.03μg/kg/min infusion rate compared to patients receiving dobutamine as determined by the 246-patient PRECEDENT Trial \citep{deLissovoy2003}.  % TO-DO study the VMAC and PRECEDENT trials

The results of these studies likely led to the approval of nesiritide, marketed under the trade name Natrecor, for the treatment of acute decompensated heart failure in the United States in 2001. In 2003, nesiritide became commercially available in Israel and Switzerland, under the trade name Noratak. However, approval in the rest of Europe was delayed pending further investigations into the renal responses of nesiritide infusion – a request that in retrospect is prophetic. %TO-DO reference and further data

With the approval of the first new intravenous compound for the treatment of heart failure in many years, use of nesiritide was immediate. After approval, the number of patients treated with nesiritide was larger than any clinical trial and with the larger sample population came some unpleasant findings. Initially, Wang and colleagues reported in 2004 that nesiritide does not improve renal function in patients with chronic heart failure \citep{Wang2004}, but more damaging were two meta-analysis studies by Sackner-Bernstein and colleagues indicating that nesiritide worsened renal function and increased the likelihood of death \citep{Sackner-Bernstein2005a} \citep{Sackner-Bernstein2005b}. % TO-DO stugy them

The results of a 75-person study (BNP-CARDS study), however, suggest nesiritide has no detrimental effect on renal function, when cohorts of similar baseline renal function were compared \citep{Witteles2007}. The number of persons in this study was small, however, so a more definitive conclusion on whether nesiritide impairs renal function will have to wait until the result of more detailed, larger studies are released. Several such studies are currently in progress. One is a clinical trial enlisting at least 1,900 patients throughout Europe and Latin America – the ETNA (Evaluating Treatment with Nesiritide in Acute Decompensated Heart Failure) trial.
This trial was scheduled to begin in 2006 to study the efficacy of nesiritide on treatment of acutely decompensated heart failure. Results from the trial are not yet available. The second study involving about 900 patients, called FUSION II, was conducted to determine the safety and efficacy of outpatient administration of nesiritide to patients with heart failure. Preliminary analysis indicates that nesiritide did not induce renal complications or increase patient mortality \citep{Cleland2007}. %TO-DO find more recent data
Finally, there is the ASCEND HF trial (Acute Study of Clinical Effectiveness of Nesiritide in Decompensated Heart Failure). This trial is scheduled to compare the effects of nesiritide treatment versus placebo for a minimum of 24 h up to a maximum of 7 days in 7,000 heart failure patients. Meanwhile, other therapeutic applications of nesiritide have also been investigated. Given that nesiritide was often reported to decrease pulmonary capillary wedge pressure, Michaels and colleagues tested its effectiveness in pulmonary hypertension \citep{Michaels2005}.
However, they found no effect of a 30 min infusion. Chen and colleagues have investigated the effectiveness of subcutaneous injections of nesiritide. Their most recent paper on effects in a dog heart failure pacing model suggest that subcutaneous injection of nesiritide reduces both preload and afterload but has no effect on cardiac output \citep{Chen2006}.

\subsection{Coronary Artery Disease}
An extremely short chapter.  No time to be wasted on it.

\subsection{Coronary Artery Bypass Grafting}
Also short
\subsubsection{on-pump vs OPCAB}


\clearpage
\section{Methods}


60 cases registered for elective off-pump coronary artery bypass grafting OPCAB were recruited in this study. Patients were interviewed preoperatively for history taking and clinical examination. EuroScore II was calculated. Demographic, past medical and surgical history, medications and baseline laboratory results (labs. on admission to hopsital) and preoperative angiography results were recorded.  No specific attempts were made to standardize the aneshetic and surgical management.  Venous samples for measuring NT-proBNP were collected on the day of surgery before induction. Samples were sent for analysis in Cairo University Clinical Pathology department. Intra-operative and postoperative data were recorded, including: duration of surgery, number of grafts, intraoperative blood transfusion and, in case CPB was need arotic cross clamp time, CPB time : lowest naso-pharyngeal temp on CPB.Patients were followed during their in-hospital stay and events recorded including: death from cardiovascular causes, ischemic stroke (defined as new neurologic deficit lasting for  24hours with definite image evidence of cerebrovascular accident by computed tomography), low output heart failure (defined as need of any of the following: CPB during off-pump surgery, intra-aortic balloon pump, inotropes at 48 hours post-operatively),  myocardial infarction (defined as elevated tropoponin 10x99th percentile URL at 12 hours after surgery associated with characteristic ECG changes or echocardiographically documented new regional wall motion abnormality). prolonged intubaton (24 hours postoperatively and reintubation) and arrhythmias.

Primary outcomes:
    \begin{itemize}
        \item low output heart failure and myocardial infarction
    \end{itemize}

Secondary outcome parameters:
    \begin{itemize}
        \item moratlity
        \item arrhythmias
        \item length of ICU and in-hospital stay
        \item prolonged intubation
    \end{itemize}
    
Data were coded and entered using the statistical package SPSS (Statistical Package for the Social Sciences) version 24. Data was summarized using mean, standard deviation, median, minimum and maximum in quantitative data and using frequency (count) and relative frequency (percentage) for categorical data. Comparisons between quantitative variables were done using the non-parametric Mann-Whitney test \citep{Chan2003a}. Correlations between quantitative variables were done using Spearman correlation coefficient \citep{Chan2003b}.  ROC curve was constructed with area under curve analysis performed to detect best cutoff value of NTproBNP for detection of outcomes. P-values less than 0.05 were considered as statistically significant.


\section{Results}

%You will be heavily qouting Chen et al 2013..filename '../PIIS1607551X12002586.pdf'








%The preoperative NT-proBNP ranged from 100 pg/mL to 14400 pg/mL (median 1600, IQR 213.5 to 2551 pg/mL).  The correlation between preoperative patient characteristics and NT-proBNP level were shown in Table 1.Univariate analysis showed that preoperative atrial fibrillation (Af), New York heart association (NYHA) class III/IV, ejection fraction (EF). A total of 11 deaths (4.89%) occurred after surgery. Major death cause included postoperative heart failure, neurologic deficit, postoperative renal failure, acute lung failure and sepsis. Median NT-proBNP level in patients with versus without clinical end points was 1998.5 pg/mL and 630.5 pg/mL (p = 0.016), respectively.
% ROC curves demonstrated the assessment of the prognostic performance of an elevated NT-ProBNP to predict the mortality within 30d of surgery (Figure 3). The area under the curve (AUC) was 0.738 (p = 0.008, 95% Confidence Interval 0.58-0.89). A NT-proBNP of 2773.5 pg/ml was the best cutoff (sensitivity of 63.6% and specificity of 80.8%).
% In the analysis, the preoperative NT-proBNP showed a significant influence on the need for ICU length of stay (LOS), hospital LOS and ventilation time (Table 2).

\section{Discussion}

% BNP is originally isolated from the porcine brain, although the highest concentration of BNP is actually found in the heart rather than in the brain. It is a cardiac hormone released by ventricular myocytes in response to ventricular dysfunction and wall stress \citep{Yasue1994}. Probrain natriuretic peptide is a 108 amino-acid propeptide synthesized by myocytes, predominantly from the left ventricle, in response to the increased wall stretch. When released, it is cleaved into 2 fragments: the active brain peptide (BNP) which has the function of natriuresis, diuresis and vasorelaxation \citep{Houben2005}, and the N-terminal amino acid sequence (NT-proBNP). In rapid time, both BNP and NT-proBNP have emerged as established cardiovascular biomarkers, especially in heart failure, hypertension, acute coronary syndrome or stable ischemic cardiac disease \citep{Steiner2008}. The main objective of our study was to find out the factors which influence preoperative plasma NT-proBNP levels, then to assess the predictive value of BNP in cardiac surgery.

% Steady-state levels of NT-proBNP are as much as four-to six-fold higher than BNP. We used NT-proBNP instead of BNP because of its longer plasma half-life (60-120 min). The longer half-life of NT-proBNP suggests it is more independent of inter- and intra-individual variations \citep{Clerico2006}. Furthermore, molecules of BNP are unstable at room temperature and start degrading immediately after blood draw if not processed. And in their relation to clinical characteristics and prognostic performance in a large population of patients with heart failure, BNP and NT-proBNP showed subtle differences \citep{Masson2006}. This is why we decided to use NT-proBNP in our study.

% There is a wide individual variation in plasma NT-proBNP values. As a general guideline, in young, healthy adults, 90% will have BNP <25 pg/ml and NTproBNP ≤70 pg/ml \citep{Daniels2007}. Our study showed that the preoperative NT-proBNP ranged from 19.3 pg/mL to 35000 pg/mL (median 728.4, IQR 213.5 to 2551 pg/mL). Univariate analysis showed that preoperative atrial fibrillation, NYHA class III/IV, LVEF, pulmonary pressure, LVEDD, preoperative creatinine, preoperative cTnT were signifi- cantly associated with preoperative NT-proBNP levels. NT-proBNP had a significant positive correlation with LVEDD, pulmonary pressure, and a significant negative correlation with LVEF and NYHA classification, respectively. This indicates that NT-proBNP is released increasingly from the left ventricle when wall tension or stretch increased as a consequence of volume overload or as left ventricular function deteriorates \citep{Provenchere2006}. Also high NT-proBNP levels are associated with increased creatinine levels and kidney failure. Besides of a reduced renal clearance of NTproBNP, increased cardiac afterload caused by fluid re- tention is a possible explanation \citep{Raymond2003}. Experimental and clinical studies have suggested that atrial volume, pressure and wall stretch are the main determinants of the activation of these peptides and are all potentially altered in atrial fibrillation \citep{Rossi2000}, so atrial fibrillation is also a risk factor of increased NT-proBNP. Furthermore, our research found that patients undergoing valvular surgery had higher NT-proBNP levels compared to patients undergoing CABG. We think that patients with valvular diseases had higher heart overload and wall tension than patients with coronary artery diseases.

% The postoperative NT-proBNP ranged from 132.1 pg/mL to 35000 pg/mL (median 1940.5, IQR 995.9 to 3892 pg/mL).  The difference between the preoperative and postoperative NT-proBNP level was statistically significant (p = 0.015). Previous studies demonstrated that BNP decreases when the aortic clamp is applied and the heart is isolated from circulation. Then there is a minor increase in BNP concentrations following cardiac reperfusion \citep{Avidan2001}. 24 h post-CPB it arrived its peak value, and returned to the pre-operative baseline values 3 weeks after surgery \citep{Morimoto1998}. The postoperative peak concentration of plasma NT-proBNP correlated with the preoperative plasma NT-proBNP concentration (r = 0.526, p < 0.001) in our research. This finding offers a possible explanation why the outcomes after cardiac surgery can be predicted by the preoperative plasma concentration.

% A variety of multifactorial risk indexes have been described to help preoperative risk assessment of patients undergoing cardiac surgery. Some studies had focused on the predictive value of preoperative NT-proBNP \citep{Nozohoor2009} \citep{Attaran2009}. Hutfless et al. considered that BNP levels > 385 pg/ml could predict the postoperative complications and one-year mortality after heart surgery \citep{Hutfless2004}. Sodeck’s research showed increased levels of NT-proBNP (>647 pg/ml) indicated unfavorable outcome \citep{Sodeck2008}. Grescenzi et al. showed that postoperative NT-proBNP levels are associated with in-hospital mortality and prolonged ICU stay after CABG surgery \citep{Crescenzi2009}. Our study demonstrated that a NT-proBNP of 2773.5 pg/ml was the best cutoff (sensitivity of 63.6% and specificity of 80.8%) to predict the mortality within 30d of surgery. In our analysis, median NT-proBNP level in patients with unfavorable clinical end points had significantly higher NT-proBNP level than those without (1998.5 pg/ml Vs. 630.5 pg/ml, p = 0.016). At the same time, the preoperative NT-proBNP showed a significant influence on the ICU LOS, hospital LOS and ventilation time. So NT-proBNP may be become a clinical routine biomarker capable of predicting patients’ perioperative and early postoperative risk because of it is convenient, quick and not expensive. The quantitative cut-off values may be used by physicians in their decision to delay heart surgery in order to further ameliorate the patient. This might be through diuresis, and NT-proBNP level could be quantitatively followed until it reaches more reassuring levels.

\section{Conclusion}

\section{Summary}

\newpage

\bibliography{thesis}

\end{document}
