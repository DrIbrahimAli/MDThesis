\documentclass[14pt,a4paper,onecolumn]{extarticle}
\usepackage[utf8]{inputenc}
\usepackage{amsmath}
\usepackage{amsfonts}
\usepackage{amssymb}
\usepackage{graphicx}
\usepackage{natbib}
\usepackage{multirow}
\usepackage{pdflscape}

\bibliographystyle{plainnat}
\setlength{\parskip}{1.2em}
\setlength{\parindent}{0pt}
% \graphicspath{{../images/}}

\author{Ibrahim AbuBakr ElSeddiq}
\title{The predictive value of NTproBNP on postoperative outcome in patients undergoing offpump CABG}

\begin{document}

\maketitle
\clearpage



\subsection{Chemical Structure and Gene Evolution of Cardiac Natriuretic Hormones
}
All natriuretic peptides share a similar structural conformation, characterized by a peptide ring with a cysteine bridge (Fig. 3.1). This ring is well preserved throughout the phylogenetic evolution, since it constitutes the portion of the peptide hormone that binds to its specific receptor (Fig. 3.2). Conversely, the two terminal amino acid chains (i.e., NH 2 - and COOH-terminus) show a high degree of variability among the natriuretic peptides, in terms of both length and amino acidic composition \citep{1}.

Natriuretic hormone-like peptides seem to be present in the plant kingdom as well as in the animal kingdom \citep{1-3}; thus indicating that this hormonal system, which has been shown to regulate solute transport in vertebrates, has evolved early in evolution \citep{1-6}.  ANP-, BNP- and CNP-like peptides have been identified in tetrapods, ranging from amphibians to mammals \citep{1}. Natriuretic peptides also exist in fish as a family of structurally related iso-hormones, including ANP, CNP and ventricular natriuretic peptide (VNP) (Fig. 3.2) \citep{1}. A recent study has indicated that BNP is also present in some fish \citep{6}. In vertebrates, ANP, BNP and VNP are endocrine hormones secreted from the heart, while CNP is principally a paracrine factor in the brain and periphery. In elasmobranches, only CNP is present in the heart and brain, and it functions as a circulating hormone as well as a paracrine factor \citep{1}. Immunoreactive ANP-like pep tide has also been detected in the unicellular Paramecium and in various species of plants \citep{1}.

CNP is the most structurally conserved member of the natriuretic peptide family (Fig. 3.3) \citep{1}\citep{4}. The ANP sequence is conserved in mammals; however, the identity across different groups (e.g. between mammals and frogs) is rather low (about 50\%) (Fig. 3.4). BNP is highly variable even in mammalian species (Fig. 3.5) \citep{1}\citep{4}.  It is noteworthy that the complete pro-hormone sequences of ANP (proANP) and CNP (proCNP) are more conserved within mammals, but they are highly variable across different classes of vertebrates, except at the C-terminal mature ANP and CNP sequences.  Thus, the N-terminal part of pro-hormones ANP (NT-proANP) and CNP may not have common biological function throughout vertebrate species \citep{1}. The NT-proBNP is not well conserved even in mammals, as observed for BNP as well \citep{1}.

Studies in fish, based on nucleotide and amino acid sequence similarity, suggest that the natriuretic peptide family of iso-hormones may have evolved from a neuromodulatory CNP-like brain peptide (Fig. 3.6) \citep{4}. However, caution should be exercised in identification and comparison of vertebrate hormones from phylogenetically distant organisms \citep{1}.

\subsection{Genes Encoding for Cardiac Natriuretic Hormones
}
The ANP and BNP coding genes, named NPPA and NPPB, are tightly linked on human chromosome 1 and mouse chromosome 4 \citep{7}\citep{8}. Arden et al. \citep{7} confirmed the assign ment to the 1 p 36 region by FISH and Southern blot analysis. Moreover, pulsed-field gel electrophoresis placed NPPA and NPPB within 50 kb of each other.

The Gene Cards Data Base reports that NPPA maps on 1p36.2 and that it is less than 10 kb more telomeric than the BNP coding gene. NPPA spans 2,075 bp and constitutes three exons and two introns (Fig. 3.7). The splicing produces an mRNA of 840 bp that is translated in the ANP precursor of 153 amino acids.

NPPB is also organized in three exons and two introns that span 1,466 bp. The mRNA is 691 bp and the relative primary product is a peptide of 134 amino acids (Fig. 3.7).  The NPPA and NPPB genes are expressed in almost all tissues but, for both, the heart is the organ in which the expression is higher (Figs. 3.8 and 3.9).

On the basis of PCR-analyzed microsatellite length polymorphisms among recom binant inbred strains of mice, Ogawa et al. \citep{10} found that the CNP gene is located on mouse chromosome 1 (Fig. 3.10). Using somatic hybrid cell methodology, the human CNP gene was assigned to chromosome 2 \citep{10}. Extrapolating from studies of homology of synteny, they suggested that the gene endocing CNP may lie in the 2q24-qter region.  This gene constitutes 824 bp, organized in two exons separated by an intron, and its mRNA (of 381 bp) gives a C natriuretic peptide precursor of 126 amino acids (Fig. 3.10).

The natriuretic peptide genes encode for the precursor sequences of these hormones, named pre-pro-hormones, which are then split into pro-hormones by proteolytic cleav age of an N-terminal hydrophobic signal peptide. This cleavage occurs cotranslation ally during protein synthesis in the rough endoplasmic reticulum, before the synthesis of the C-terminal part of the pro-hormone sequence is completed \citep{11-13}.  The pro-hormone of ANP is stored as a 126-amino acid peptide, proANP 1-126 (also called γANP), which is produced by cleavage of the signal peptide (Fig. 3.11). When appropriate signals for hormone release are given, proANP 1-126 is further split into an NH 2 terminal fragment, proANP 1-98 (actually called NT-proANP) and the COOH-terminal peptide ANP 99-126 (ANP), which is generally considered to be the biologically active hormone \citep{11-13}.

The human BNP gene encodes for a preproBNP molecule of 134 amino acid residues with a signal peptide of 26 amino acids (Fig. 3.12). BNP is produced from a pro-hormone molecule of 108 amino acids, the proBNP 1-108 , usually indicated as proBNP. Before secre tion, proBNP is split by proteolytic enzymes into two peptides: the proBNP 1-76 (NH 2 terminal peptide fragment, usually indicated as NT-proBNP), which is biologically inactive, and the proBNP 77-108 (COOH-terminal peptide fragment), which is the active hormone (BNP) \citep{14}.

It is important to note that the preproBNP precursor is not detectable, and its existence is only a theoretical concept, deduced from the BNP cDNA sequence of the human (or other mammalian) gene \citep{14}. On the other hand, intact proBNP, NT-proBNP, and BNP can be identified in plasma by chromatography and immunoassay \citep{14-17}. Moreover, ANP and BNP can be produced and co-stored in the same granule in different stages of peptide maturation \citep{11-13}.

3.3 Regulation of Production/Secretion of ANP and BNP in Cardiac Tissue

Atrial natriuretic peptide and BNP are synthesized and secreted mainly by cardiomyocytes. However, it is generally thought that ANP is preferentially produced in the atria, while BNP is produced in the ventricles, particularly in patients with chronic cardiac diseases. Synthesis and secretion of the two different peptides may be differently regulated in atrial and ventricular myocytes, and, probably, even at different stages of life (neonatal or adult life) \citep{12-14}\citep{18-23}. Furthermore, there could be some differences in regulation of expression of CNH genes among mammalian species; for example, between rat and human in cis-acting regulatory elements responsible for BNP promoters \citep{22}.

These findings suggest some caution in the use of study results performed in experimental animals to interpret specific pathophysiological conditions in humans. Unfortunately, our present knowledge on regulation of CNH gene expression derives in great part from experimental studies in rodents \citep{18-23}.

Some recent data suggest that not only cardiomyocytes, but also fibroblasts may produce CNH in human heart \citep{24}; however, the clinical relevance of this finding has not been ascertained. According to studies performed in experimental animals, it has also been proposed that the endocrine response of the heart to pressure or volume load varies depending on whether the stimulus is acute, subacute, or chronic \citep{12}\citep{13}\citep{18}\citep{22}.

Atrial cardiomyocytes store pro-hormones (proANP and proBNP) in secretory granules, and split them into ANP and BNP before secretion.ANP and BNP are produced and stored in different stages of peptide maturation even though they are co-stored in the same granule. It was suggested that the prevalent peptide form in the atrial granules is unprocessed proANP, whereas BNP is mainly stored in atrial granules as the processed and mature form of the active peptide hormone \citep{12}\citep{13}\citep{18}.

Peptide hormones are secreted via several pathways including regulated secretion, constitutive secretion and constitutive-like secretion. Regulated secretion occurs upon stimulation by agonists whereby the hormone packaged is released from the storage granules, thus allowing endocrine cells to secrete hormones at a rate that exceeds its biosynthesis \citep{13}. The constitutive-like pathway is independent from both regulated and constitutive secretion and is insensitive to cycloheximide treatment \citep{13}. The constitutive pathway involves the passive diffusion of hormone in the absence of stimuli.  The presence of clathrin-coated vesicles in the atrial cardiocytes is indicative of the existence of this constitutive pathway \citep{13}.

CNH (especially ANP) should be predominantly secreted throughout a regulated pathway \citep{12}\citep{13}\citep{18}. There is also the possibility that a small amount of CNH is released via a constitutive pathway involving passive diffusion of secretory products \citep{12}\citep{13}\citep{18}.  Indeed, protein synthesis inhibition by cycloheximide significantly, though partially, decreases ANP secretion; this suggests that basal ANP secretion is to a degree, dependent on newly synthesized hormone \citep{13}. However, a component of basal secretion appears to depend upon stored hormone. Some studies have demonstrated that 60\% of the newly synthesized ANP is intended for storage, whereas the remaining 40\% is secreted under unstimulated conditions \citep{13}.

It is likely that circulating CNH (especially ANP) mostly derives from atria in healthy subjects \citep{12-14}\citep{18}. An acute atrial stretch may result in an immediate, sharp increase in the release rate of ANP at the expense of a rapidly depleting pool of newly synthesized hormone. The BNP pool that is utilized in response to stretch remains to be determined but in terms of peptide release, the ratio of ANP/BNP released is roughly the same as found in tissue storage, i.e., the response is in keeping with the lesser proportion of BNP present in mature atrial granules \citep{18}. The response of the CNH system to stretch may not be strictly reflected in vivo by an increase in plasma levels of ANP and BNP, probably because the amount of peptide released is masked by dilutional effects and different clearance rates of the two peptide hormones.

The BNP gene is expressed in both atrial and ventricular myocytes of normal and diseased heart \citep{12-14}\citep{18}. Ventricular myocytes in the normal heart of adult mammals do not usually show any evident secretory granules on electron microscopy \citep{13}\citep{14}.  However, some authors have identified secretory granules, similar to the atrial ones, in samples of ventricular myocardium collected during surgery or in endocardial biopsies studied by electron microscopy and immunohistocytochemistry in patients with cardiac disease \citep{14}\citep{25}\citep{26}. These studies suggest that normal ventricular myocardium may produce only a limited amount of BNP in response to an acute and appropriate stimulation, probably via a constitutive secretory pathway, while the amount of hor mone produced and secreted after chronic stimulation could be greatly increased via an upregulated secretory pathway. However, further studies are necessary in order to con firm the presence of secretory granules in human ventricular cardiomyocytes and in par ticular to evaluate their peptide content and pathophysiological relevance.  BNP mRNA levels increase predominantly by upregulated gene transcription and to a lesser degree, by post-transcriptional mechanisms \citep{22}. BNP transcription can increase with the activation of numerous positive cis-acting regulating elements or the inhibition of negative cis-acting elements on the 5’-flanking region on the BNP pro moter, as recently reviewed in detail \citep{18}\citep{22}. The type of the element depends on whether the stimulus is a mechanical or a neuro-humoral agonist \citep{22}\citep{23}. Due to their ability to cooperate with a diverse group of transcription factors and to react to a variety of dif ferent stimuli, some proteins of the GATA family (especially GATA-4 and GATA-6) emerge as crucial factors in regulating basal expression and inducible ANP and BNP pro moter activity \citep{18}\citep{23}. The transcription factors of the GATA family (from GATA-1 to GATA-6) are zinc-finger proteins that bind to the consensus sequence (A/T)GATA(A/G) via a DNA-binding domain containing two zinc fingers to activate target genes. GATA proteins play important roles in cell differentiation and homeostasis in all eukaryotes \citep{23}. It has been suggested that endothelin-1, angiotensin II and adrenergic agonists increase the expression of BNP mRNA in cultured isolated cardiomyocytes by activat ing the proximal GATA elements \citep{22}. The muscle-CATTCCT (M-CAT) consensus ele ment, the shear stress responsive element (SSRE), and the thyroid hormone response ele ment (TRE) could be other additional cis-acting factors regulating inducible BNP expres sion \citep{18}\citep{22}. Like GATA proteins, these elements can also be selectively activated by specific stimuli, especially by some neuro-hormones (such as b-adrenergic agonist and thyroid hormones) and cytokines (such as interleukin-1β [IL-1β] and tumor necrosis factor-α [TNF-α]) \citep{22}.

From a clinical point of view, it is important to note that chronic stimulation produces a greater amount of BNP than ANP, probably because the former is produced mainly by ventricular myocardium, which has a relatively greater mass than atria. However, ven tricular BNP gene expression can be selectively upregulated during the evolution of diseases affecting the ventricles, as demonstrated in an experimental model of heart failure (HF), i.e., the rapid ventricular pacing-induced congestive HF in dog \citep{27}. In fact, on average, the molar ratio of circulating BNP over ANP increases progressively with the severity of HF from a value of about 0.5 in healthy subjects up to about 3 in patients with NYHA functional class IV (Table 3.1, Fig. 3.13) \citep{28}.

These data explain why the BNP assay usually shows a better diagnostic accuracy in patients with cardiac disease than the ANP assay \citep{28}.

ANP and BNP are secreted from the heart into the circulation, thus providing for a baseline level of the hormones in blood \citep{13}. The response of the heart to pressure or volume load varies in relation to whether the challenge is acute, subacute or chronic \citep{12}\citep{13}.

Wall stretch is the most important stimulus for synthesis and secretion of ANP at the atrial level \citep{12}\citep{13}\citep{21-23}. The increased secretion of ANP following acute mechan ical atrial stretch is based on a phenomenon referred to as “stretch-secretion coupling”.  This effect is dependent upon a depletable ANP pool and is characterized by a phasic, short-term (i.e. minutes) burst of CNH secretion with no apparent effect on synthesis \citep{13}. The precise mechanism by which force is translated into biochemical stimulus has not been completely elucidated. However, several studies have documented the importance of outside-in signaling by extracellular matrix proteins (especially inte grin) in translating mechanical stress to changes in gene expression and the induction of ANP and BNP in hypertrophic myocardium \citep{22}.

Any physiological condition associated with an acute increase in venous return (pre load), such as physical exercise, rapid change from standing to supine position, or head out water immersion, causes a more rapid augmentation in ANP than in BNP plasma concentration. For instance, changes in ANP and BNP secretion have been well char acterized during and after the tachyarrhythmia induced in pigs by rapid atrial pacing (225 beats per minute). In this model, ANP plasma concentration shows a sharp initial peak followed by a decline, but remains significantly increased throughout a 24-hour post pacing period, while BNP increases significantly after an 8-hour pacing, and even more after a 24-hour pacing \citep{29}. Even acute changes in the effective plasma circulating vol ume, such as during a dialysis session in patients with chronic renal failure, cause greater variations in circulating levels of ANP than BNP \citep{30}.  Whereas chronic stimulated CNH secretion results in increased synthesis and secre tion in both atria and ventricles, there is an intermediate level stimulation of the endocrine heart whereby increased synthesis and secretion of CNH is evident only in the atria \citep{12}\citep{13}. This can be observed during the “mineralocorticoid escape”, a patho physiological condition characterized by a transient period of positive sodium balance resulting from chronic exposure to mineralocorticoid excess, such as aldosterone or deoxycorticosterone acetate followed by a vigorous natriuresis leading to a new steady state of sodium balance \citep{12}\citep{13}. The rise in intravascular volume and central venous pres sure leads to increased CNH production by the atria.

Wall distension is generally considered the main mechanical stimulus for CNH (especially BNP) production by ventricular tissue. This occurs in long-standing con ditions characterized by electrolyte and fluid retention, and therefore expansion of effective plasma volume, such as primary \citep{31} and secondary hyperaldosteronism, including cardiac, renal and liver failure \citep{12}\citep{13}\citep{18}\citep{30}. The changes in the pattern of gene expression are observed not only in the long-term hypertrophic process of ven tricular myocardium, but also at the onset of hemodynamic overload. Moreover, expres sion of the BNP gene takes place with many characteristics of an immediate-early gene. Indeed, hemodynamic overload in the left ventricle has been shown to result in an increase in the BNP gene expression within 1 h, associated with the expression of oncogenes c-fos and c-jun \citep{32}. A recent study suggested that the stretch-induced acti vation of BNP gene expression by increased left ventricular wall stress in an isolated perfused rat heart preparation is independent of transcriptional mechanisms and dependent on protein synthesis \citep{32}; other studies are necessary to confirm and clar ify these findings.

The presence of ventricular hypertrophy and fibrosis can stimulate hormone produc tion \citep{12}\citep{13}\citep{18-21}\citep{30}\citep{33-36}. However, recent studies have suggested that myocardial fibro sis rather than hypertrophy is associated with increased production of BNP \citep{24}\citep{35}\citep{36}.

More recently, several studies indicated that myocardial ischemia and hypoxia per se could also induce the synthesis/secretion of CNH by cardiomyocytes \citep{37-44}. The stud ies indicating that ANP gene is responsive to hypoxia have been reviewed recently in detail \citep{37}. Several data indicated that hypoxia directly stimulates ANP gene expression and its release in cardiac myocytes in vitro \citep{37}. The effect of hypoxia on BNP produc tion/secretion was also studied, demonstrating that surgical reduction of blood in an area of the anterior ventricular wall in pigs increased BNP mRNA by 3.5-fold in hypoxic compared with normoxic ventricular myocardium \citep{44}. Moreover, proBNP peptide accumulated in the medium of freshly harvested ventricular myocyte cultures, but was undetectable in ventricular myocardium, indicating rapid release of the newly synthe sized proBNP peptide \citep{44}. The direct stimulating effect of hypoxia on CNH gene expres sion is probably due to the activation of promoter activity; however, other potential mechanisms could modulate peptide hormone release from cardiomyocytes, includ ing the influx of extracellular Na + and Na + /Ca ++ exchange (due to the hypoxia-depend ent intracellular acidosis) as well as the activation of protein kinase C \citep{37}. Moreover, clinical studies reported that plasma levels of CNH (especially BNP and its related pep tides) were found to be closely related to aerobic exercise capacity in patients with HF \citep{45-47}. In particular, plasma NT-proBNP correlates better than indices of left ventric ular systolic function, such as ejection fraction, with peak oxygen consumption and exercise duration \citep{47}. These results may explain the elevated levels of BNP found even in patients with acute coronary syndrome (ACS), in the absence of a significant dilata tion of the ventricular chambers \citep{40}. This suggests a neuro-hormonal activation sec ondary to both reversible myocardial ischemia and necrosis \citep{41}.

There is increasing evidence from in vivo and ex vivo studies supporting the hypoth esis that the production/secretion of CNH is regulated by complex interactions with the neuro-hormonal and immune systems, especially in ventricular myocardium \citep{12}\citep{13}\citep{18}. Neuro-hormones, cytokines, and growth factors that can affect the produc tion/secretion of CNH are summarized in Table 3.2.

Endothelin and angiotensin II are considered the most powerful stimulators of pro duction/secretion of CNH \citep{12}\citep{13}\citep{18}\citep{22}; similarly, glucocorticoids, sex steroid hor mones, thyroid hormones, some growth factors and mediators of inflammation (such as some cytokines, especially TNF-α, interleukin-1, interleukin-6, and lipopolyliposaccha ride, LPS) share stimulating effects on the CNH system \citep{12}\citep{13}\citep{18}\citep{22}\citep{36}\citep{48-57} (Table 3.2). The interesting finding that CNH production is stimulated by cytokines and growth factors suggests a link between cardiac endocrine activity and remodelling or inflam matory processes in myocardial and smooth muscle cells. A large number of studies have recently contributed to support this hypothesis \citep{33-36}\citep{50-53}\citep{56-62}.

More complex, and still in part unknown, is the effect of adrenergic stimulation on CNH production. The α 1 -adrenergic agonist phenylephrine enhances the expression of some transcription factors, such as Egr-1 and c-myc, regulating (usually increasing) the natriuretic peptide gene expression in cultured neonatal rat cardiomyocytes \citep{12}\citep{13}\citep{18}\citep{63-66}. Conflicting results were obtained with β-agonists. In one study, the β agonist isoproterenol reduced the expression of BNP mRNA, but not that of ANP, an effect prevented by the β 1 -antagonist CGP20712A in isolated adult mouse cardiomy ocytes \citep{67}. In another study, isoproterenol stimulated the BNP mRNA expression in rat ventricular myocyte-enriched cultures \citep{68}. However, it was suggested that the stimu latory effects of both α- and β-adrenergic agonists on BNP gene inducible transcription are principally mediated by GATA elements \citep{22}\citep{23}.

Clinical studies performed in hypertensive patients have shown that monotherapy with a β-blocker, either β 1 -selective or not, is associated with an increase in the plasma concentration of ANP and/or BNP and their related peptides \citep{69-71}. In contrast, CNH response can be heterogeneous during β-blocker therapy in congestive HF \citep{28}\citep{72}, probably due to the various additive effects of other co-administered drugs. However, sustained treatment with β-blockers with improvement in cardiac function and exer cise capacity and reduction in filling pressure and cardiac volumes is usually associat ed with a fall in CNH levels in patients with HF \citep{28}\citep{73}\citep{74}.

As far as hormones more specifically acting on intermediate metabolism are con cerned, insulin (but not hyperglycemia) increased protein synthesis and ANP secre tion and gene expression in cultured rat cardiac myocytes \citep{75}. Moreover, in a model of genetic murine dilated cardiomyopathy, short-term RhGH treatment improved left ven tricular function and significantly reduced elevated mRNA expression of ANP and BNP gene expression in left ventricular tissue \citep{76}.

In conclusion, a huge number of experimental and clinical studies demonstrated that production and secretion of CNH (especially BNP) are not only related to hemo dynamic variations, but also subtly regulated by neuro-hormonal and immunological factors. Therefore, the variation of CNH circulating levels can be considered as a sen sitive index of the perturbation of the homeostatic systems.  3.4 Biological Action of CNH It is important to note that specific receptors for CNH have been found in all mam malian tissues, although at different concentrations, thus suggesting that CNH play an important biological role in several tissues. Most of these biological effects of CNH may be due to an autocrine/paracrine, rather than hormonal, action. In this section, we will discuss in detail only the best-known effects of CNH, and in particular those more strictly related to the endocrine function of the heart.

Cardiac natriuretic hormones have powerful physiological effects on the cardiovas cular system, body fluid, and electrolyte homeostasis \citep{13}\citep{28}\citep{30}\citep{77}\citep{78}. CNH share a direct diuretic, natriuretic and vasodilator effect and an inhibitory action on ventricu lar myocyte contraction \citep{79} as well as remodeling and inflammatory processes of myocardium and smooth muscle cells \citep{80-83} (Fig. 3.14). Thus, CNH exert a protective effect on endothelial function by decreasing shear stress, modulating coagulation and fib rinolysis pathways, and inhibiting platelet activation (Fig. 3.15). They can also inhibit vascular remodeling process as well as coronary restenosis post-angioplasty \citep{56}\citep{84-89}.

Furthermore, CNH share an inhibitory action on neuro-hormonal and immuno logical systems, and on some growth factors \citep{13}\citep{28}\citep{30}\citep{77}\citep{78}\citep{90-99}. In particular, the pivotal role of CNH (especially ANP) in modulating the immune response has been reviewed recently \citep{98}. The first evidence for a role of CNH in the immune system was given by the fact that peptide hormones and their receptors are expressed in various immune organs. Furthermore, several studies indicated that the CNH system in immune cells underlies specific regulatory mechanisms by affecting the innate as well as the adaptive immune response \citep{99}. In particular, ANP supports the first line of defense by increasing phagocytotic activity and production of reactive oxygen species of phago cytes. ANP affects the induced innate immune response by regulating the activation of macrophages at various stages. It also reduces production of pro-inflammatory medi ators by inhibition of iNOS and COX-2 as well as TNF-α synthesis. ANP also affects TNF-α action, i.e. it interferes with the inflammatory effects of TNF-α on the endothe lium. The peptide hormone counteracts TNF-α-induced endothelial permeability and adhesion and attraction of inflammatory cells. Finally, it affects thymopoesis and T cell maturation by acting on dendritic cells and regulates the balance between TH1 and TH2 responses \citep{99}.

The cited effects on the cardiovascular system and body fluid and electrolyte homeosta sis can be explained at least in part by the inhibition of control systems, including the sym pathetic nervous system, the renin-angiotensin-aldosterone system (RAAS), the vaso pressin/antidiuretic hormone system, the endothelin system, cytokines and growth factors \citep{90-99}.The endocrine action,shared by plasma ANP and BNP,can be enhanced by natriuretic peptides produced locally in target tissues (paracrine action). Indeed, endothelial cells syn thesize CNP, which in turn exerts a paracrine action on vessels \citep{57}\citep{84-88}. Moreover, renal tubular cells produce urodilatin, another member of the peptide natriuretic family, which has powerful diuretic and natriuretic properties \citep{100}.Genes for natriuretic peptides (includ ing ANP, BNP and CNP) are also expressed in the central nervous system, where they likely act as neurotransmitters and/or neuromodulators \citep{91-93}\citep{100-102}. In particular, it was demonstrated that intranasal ANP acts as central nervous inhibitor of the hypothalamus pituitary-adrenal stress system in humans \citep{103}. Finally, co-expression of CNH and of their receptors was observed in rat thymus cells and macrophages \citep{104}\citep{105},suggesting that CNH may have immunomodulatory and anti-inflammatory functions in mammals \citep{106}.

A recent detailed review \citep{107} has highlighted a possible major role for CNH in the development of certain systems, in particular skeleton, brain, and vessels. This review cites recent studies showing severe skeletal defects and impaired recovery after vascu lar and renal injury in CNH transgenic and knockout (KO) mice \citep{108}. In addition, CNH may have a role in the regulation of proliferation, survival, and neurite outgrowth of cultured neuronal and/or glial cells \citep{108}.

Changes in plasma ANP are also correlated with alcohol-associated psychological vari ables \citep{108}. Acute administration of alcohol stimulates the release of ANP independent ly of volume-loading effects. Patients whose ANP levels fell markedly during abstinence also reported more intense and frequent craving as well as more anxiety \citep{108}.  Several reports have shown that CNH stimulate the synthesis and release of testos terone in a dose-dependent manner in isolated and purified normal Leydig cells \citep{109 112}. It has been suggested that this effect on normal Leydig cell steroidogenesis does not involve classical mechanisms of cAMP-mediated regulation of steroidogenic activ ity by gonadotropins \citep{112}. The stimulated levels of testosterone production by ANP, BNP, and gonadotropins were comparable, whereas CNP has been found to be a weak stim ulator of testosterone production in Leydig cells \citep{112}. Moreover, testicular cells contain immunoreactive ANP-like materials and a high density of natriuretic peptide recep tor-A (NRP-A) \citep{112}. These findings suggest that CNH play paracrine and/or autocrine roles in testis and testicular cells. Furthermore, the presence of ANP and its receptors has been reported in ovarian cells, too. Increasing evidence strongly support that CNH are present and probably locally synthesized in ovarian cells of different mammalian species and also play an important physiological role in stimulating estradiol synthe sis and secretion in the female gonad \citep{112-115}. However, further studies are necessary in order to clarify completely the role played by CNH in the regulation of gonadal func tion and also to assess the inter-relationship between heart endocrine function and gonadal function in humans.

The huge amount of data reported above strongly supports the hypothesis that CNH are active components of the body integrative network that includes nervous, endocrine and immune systems. According to this hypothesis, the heart can no longer be seen as a passive automaton driven by nervous, endocrine or hemodynamic inputs, but as a leading actor on the stage. Thus, CNH, together with other neuro-hormonal factors, regulate cardiovascular hemodynamics and body fluid and electrolyte homeostasis, and probably modulate inflammatory response in some districts, including the car diovascular one. This hypothesis implies that there are two counteracting systems in the body: one has sodium-retaining, vasoconstrictive, thrombophylic, pro-inflamma tory and hypertrophic actions, while the second one promotes natriuresis and vasodi latation, and inhibits thrombosis, inflammation and hypertrophy. CNH are the main effectors of the latter system, and work in concert with NO, some prostaglandins, and other vasodilator peptides (such as bradykinin) \citep{116-120}. Under physiological condi tions, the effects of these two systems are well balanced via feedback mechanisms, and result in a beat-to-beat regulation of cardiac output and blood pressure in response to endogenous and exogenous stimuli. In patients with HF, the action of the first system is predominant, as a compensatory mechanism, initially, that progressively leads to detrimental effects.

The knowledge so far accumulated regarding CNH suggests that a continuous and intense information exchange flows from the endocrine heart system to nervous and immunological systems and to other organs (including kidney, endocrine glands, liver, adipose tissue, immuno-competent cells) and vice versa (Fig. 3.16). From a pathophys iological point of view, the close link between the CNH system and counter-regulatory systems could explain the increase in circulating levels of CNH in some non-cardiac-relat ed clinical conditions. Increased or decreased BNP levels were frequently reported in acute and chronic respiratory diseases \citep{121-129}, some endocrine and metabolic diseases \citep{130-141}, liver cirrhosis \citep{142-144}, renal failure \citep{100}\citep{144}, septic shock, chronic inflam matory diseases \citep{145-149}, subarachnoid hemorrhage \citep{150-153}, and some paraneo plastic syndromes \citep{154-156}. In addition, any myocardial damage leading to the release of sarcoplasma constituents (including CNH) in extracellular fluid, for instance that due to cardiotoxic agents \citep{157-161}, cardiac trauma or ischemic necrosis \citep{162}\citep{163}, also causes an increase in plasma concentration of CNH.

Furthermore, the inter-relationships between the CNH system and pro-inflammatory cytokines suggest that cardiac hormones play an important role in mechanisms respon sible for cardiac and vascular adaptation, maladaptation and remodeling in response to various physiological and pathological stimuli \citep{32}\citep{35}\citep{62}\citep{162}.  Elevated BNP levels in extra-cardiac diseases reveal an endocrine heart response to a “cardiovascular stress” (Fig. 3.17). Indeed, recent studies reported that plasma BNP concentration is an independent risk factor for mortality (cardiac and/or total) in pul monary embolism \citep{121}\citep{123}\citep{124} and hypertension \citep{127}, renal failure \citep{28}\citep{100}\citep{144}, sep tic shock \citep{145}, amyloidosis \citep{149}, and diabetes mellitus \citep{141} (see Chapter 6 for more details). According to this hypothesis, a BNP assay should be considered as a marker of cardiac stress (Fig. 3.17).

In conclusion, CNH share a powerful action on the cardiovascular system, including diuretic, natriuretic and vasodilator effects and an inhibitory action on ventricular myocyte contraction, as well as on remodeling and inflammatory processes of myocardi um and smooth muscle cells. Furthermore, CNH exert a protective effect on endothelial function by decreasing shear stress, modulating coagulation and fibrinolysis pathways, and inhibiting platelet activation. They can also inhibit the vascular remodeling process as well as coronary restenosis post-angioplasty. These effects can be explained, at least in part, by the inhibition of control systems, including the sympathetic nervous system, the RAAS, the vasopressin/antidiuretic hormone system, the endothelin system, cytokines and growth factors. Finally, the endocrine action of ANP and BNP is potentiated at the periphery (target tissues) by the paracrine action of other members of the peptide natri uretic family, such as CNP (in the vascular tissue) and urodilatin (in renal tissue).  Finally, some experimental studies performed in KO mice suggest a distinct patho physiological role for BNP in respect to ANP \citep{18}. While BNP KO mice are no different from control mice with regard to blood pressure, urine volume, and urinary electrolyte excretion, they have more extensive ventricular fibrosis, accompanied by increased transforming growth factor-b3 (TGF-b3) and collagen mRNA \citep{18}. These data suggest that BNP may function more as an autocrine/paracrine inhibitor of cell growth in the heart; while ANP may be considered as a traditional circulating hormone with pro nounced diuretic, natriuretic, and antihypertensive effects.

\subsection{Natriuretic Peptide Receptors and Intracellular Second Messenger Signaling
}
Cardiac natriuretic hormones share their biological action by means of specific recep tors (NPR), which are present within the cell membranes of target tissues. Three different subtypes of NPRs have so far been identified in mammalian tissues \citep{112}\citep{164}\citep{165}.

\subsubsection{Genes Encoding for NPRs
}
NPR1 is the gene coding the NPR-A receptor (natriuretic peptide receptor A/guany late cyclase A) and it is located on 1q21-q22 spanning 15,534 bp with 22 exons. The rel ative mRNA of 3,805 bp leads to a protein of 1,061 amino acids.  The NPR-B receptor (natriuretic peptide receptor B/guanylate cyclase B) is codified by the gene NPR2. This gene of 17,303 bp is on chromosome 9 (9p21-p12) and it is organized in 22 exons, which can give two types of mRNA. NPR2 Ia is an mRNA of 3,482 bp that has a 71 nucleotide insertion relative to isoform b, which results in a dif ferent, and shorter (995 aa), carboxy-terminus that may disrupt the guanylyl cyclase activity. NPR2 Ib (3,411 bp, 1,047 aa) does not include the alternate exon found in iso form a, and thus isoform b contains a longer carboxy-terminus. The natriuretic peptide receptor C gene, also named NPR3, is on 5p14-p13 and spans 74,698 bp (8 exons), giv ing an mRNA of 1,753 bp that is translated into a protein of 540 amino acids.

\subsubsection{Biological Function of NPRs
}
NPR-A and NPR-B are generally considered to mediate all known biological actions throughout the guanylate cyclase (GC) intracellular domain, while the third member of the natriuretic peptide receptor family, the NPR-C receptor, does not have a GC domain (Figs. 3.18, 3.19 and 3.20).  The GC receptors for ANP/BNP (NPR-GC-A) and CNP (NPR-GC-B) belong to a fam ily of seven isoforms of transmembrane enzymes (from GC-A to GC-G), which all con vert guanosine triphosphate into the second messenger cyclic 3’,5’-guanosine monophos phate (cGMP) \citep{164}.

Although partly homologous to soluble GC, the receptor for NO, the membrane GCs share a different and unique topology. The single transmembrane span domain divides the protein structure into an extracellular ligand binding domain and an intracellular region consisting of a protein kinase-homology domain, an amphipathic helical or hing region, and a cyclase-homology domain \citep{165} (Figs. 3.18, 3.19 and 3.20). The cyclase homology domain represents the catalytic cGMP synthesizing domain. The function of the intracellular region consisting of a protein kinase-homology domain is incom pletely understood. Although it probably binds ATP and contains many residues con served in the catalytic domain of protein kinases, kinase activity has not been detect ed \citep{165}. It represses the enzyme activity of the catalytic cGMP-synthesizing domain and at the same time is necessary for its ligand-dependent activation \citep{154}. The coiled-coil hing region is involved in receptor dimerization, which is also essential for the activa tion of the catalytic domain \citep{165}.

The cGMP produced modulates the activity of specific downstream regulatory pro teins, such as cGMP-regulated phosphodiesterases, ion channels and cGMP-dependent protein kinases type I (PKG I) and type II (PKG II) (Fig. 3.20). These proteins should be considered to be third messengers, which are differentially expressed in different cell types, ultimately modifying cellular functions \citep{166}\citep{167}. This specific action of CNH on target tissues depends essentially on two different mechanisms.

The physiological expression of NPR-A and NPR-B differs quite significantly in human tissues (Fig. 3.21). NPR-A is found in abundance in larger, conduit blood vessels, whereas the NPR-B is found predominantly in the central nervous system \citep{168}. Both receptors have been localized in adrenal glands and kidney \citep{168}. On the other hand, several studies indicate that phosphorylation of the kinase homology domain is a crit ical event in the regulation of NPRs \citep{169-171}.

The affinity for ANP, BNP and CNP also varies greatly among the different NPRs.  ANP shows a greater affinity for NPR-A and NPR-C, and CNP for NPR-B, while BNP shows a lower affinity for all NPRs compared to the other two peptides (Fig. 3.21).  Activation of the GC-linked NPRs is incompletely understood \citep{172}. NPR-A and NPR B are homo-oligomers in the absence and presence of their respective ligands, indicat ing that receptor activation does not simply result from ligand-dependent dimerization \citep{173}. However, ANP binding does cause a conformational change of each monomer closer together \citep{172-176}. The stoichiometry of the ligand-receptor complex is 1:2 \citep{177}.  Initial in vitro data suggested that direct phosphorylation of NPR-A by protein kinase C mediated its “desensitization” (i.e., the process by which an activated receptor is turned off) \citep{178}. However, subsequent studies conducted in live cells indicated that desensiti zation in response to prolonged natriuretic peptide exposure or activators of protein kinase C results in a net loss of phosphate from NPR-A and NPR-B \citep{171}\citep{179-182}.

Although ligand-dependent internalization and degradation of NPR-A has been intense ly studied by several groups for many years, a consensus understanding of the importance of this process in the regulation of NPRs has not emerged \citep{182}. Early studies conducted on PC-12 pheochromocytoma cells suggested that both NPR-A and NPR-C internalize ANP and that both receptors are recycled back to the cell surface \citep{184}. Other studies, using Leydig, Cos, and 293 cell lines, have reported that ANP binding to NPR-A stimulates its internalization, which results in the majority of the receptors being degraded with a smaller portion being recycled to the plasma membrane \citep{184-187}. In contrast, other stud ies performed in cultured glomerular mesangial and renomedullary interstitial cells from the rat or Chinese hamster ovary cells reported that NPR-A is a constitutively membrane resident protein that neither undergoes endocytosis nor mediates lysosomal hydrolysis of ANP \citep{188}\citep{189}. A more recent study using 293T cells suggested that NPR-A and NPR B are neither internalized nor degraded in response to receptor occupation \citep{173}. Fur thermore, this study did not support the hypothesis that down-regulation is responsible for NPR desensitization observed in response to various physiological or pathological stim uli \citep{182}.Further studies are necessary to clarify whether or not ANP binding to NPR-A stim ulates its internalization, and whether this process is tissue- and/or species-specific.

It is generally thought that the NPR-C is not linked to GC and so serves as a clearance receptor \citep{28}\citep{77}\citep{78}.NPR-C is present in higher concentration than NPR-A or NPR-B in sev eral tissues (especially vascular tissue),and it is known constitutively to internalize CNH \citep{172} (Fig.3.22).However,recent studies have found that CNH interact with NPR-C to suppress the cAMP concentration by inhibition of adenylyl cyclase \citep{190}\citep{191}. Specific binding to NPR-C increases inositol triphosphate and diacylglycerol concentrations by activating phospholi pase C activity or inhibits DNA synthesis stimulated by endothelin, platelet-derived growth factor and phorbol ester by inhibiting MAPK activity,as recently reviewed \citep{190}.The NPR-C mediated inhibition of adenylyl cyclase is mediated through Gi (inhibitory guanine nucleotide regulatory) proteins.According to this hypothesis,NPR-C,which is present in large amounts, especially on the endothelial cell wall,may mediate some paracrine effects of CNP on vascu lar tissue \citep{168}\citep{190}.However,further studies are necessary to elucidate the possible role of NPR C receptors as modulators of CNH action and/or degradation in peripheral tissues.

\subsection{Metabolic Pathways and Circulating Levels of CNH
}
Atrial natriuretic peptide and BNP are secreted directly from the heart. In the circula tion, CNHs are metabolized via two principal mechanisms: degradation by a mem brane-bound endopeptidase (NEP 24.11) and receptor-mediated cellular uptake via NPR-C \citep{14} (Fig. 3.22). Some biological characteristics of ANP, BNP and CNP (as well as of their precursors) are summarized in Table 3.3.

\subsubsection{ANP Metabolism
}
Atrial natriuretic peptides are a family of peptides derived from a common precursor, called preproANP, which in humans contains 151 amino acids and has a signal peptide sequence at its amino-terminal end (Fig. 3.11). The pro-hormone is stored in secretion granules of cardiomyocytes as a 126-amino-acid peptide, proANP 1-126 , which is pro duced by cleavage of the signal peptide. When appropriate signals for hormone release are given, proANP 1-126 is further split by some proteases (especially the serine protease corin) \citep{192} into N-terminal fragment NT-proANP and the COOH-terminal peptide ANP, which is generally considered to be the biologically active hormone, because it contains the cysteine ring (Figs. 3.1 and 3.11).

Studies from the group of Vesely et al. suggested that the NT-proANP can be metab olized in vivo in three peptide hormones with blood pressure-lowering, natriuretic, diuretic and/or kaliuretic properties \citep{100}. These peptide hormones, numbered by their amino acid sequences, beginning at the N-terminal end of the proANP pro-hormone, include: 1) the peptide proANP 1-30 , also called long-acting natriuretic peptide (LANP); 2) the peptide proANP 31-67 with vessel dilator properties; 3) the peptide proANP 79-98 with kaliuretic properties \citep{98}. However, these three peptides do not bind to the same NPRs of CNHs, because they do not have the cysteine ring. Further studies are neces sary to confirm and elucidate the biological action of these putative peptide hormones, as well as their in vivo metabolism.

There is some evidence that ANP is secreted according to a pulsatile pattern in humans \citep{193-197}. Upon secretion, ANP is rapidly distributed and degraded (the meta bolic clearance rate of ANP is on average about 2,000 ml/min in healthy subjects) with a plasma half-life of about 4-6 minutes in healthy adult subjects. In humans, about 50\% of the ANP secreted into the right atrium is extracted by the peripheral tissues during the first pass throughout the body \citep{198-201}. Furthermore, circulating ANP represents only a small fraction of the total body pool (no more than 1/15) in normal subjects and plasma ANP concentration shows rapid and wide fluctuations in healthy subjects, even at rest in the recumbent position \citep{198-201}. The turnover data suggest that circulating levels of ANP may not represent a close estimate of their disposal, and therefore of the activity of the CNH system, as implicitly accepted in physiological or clinical studies in which only the plasma concentration of the hormone is measured, without an estima tion of turnover rate. However, it was demonstrated that ANP clearance mechanisms are constant in the presence of rapid and large changes in endogenous ANP plasma levels induced by atrial and/or ventricular pacing, thus indicating that, at least for studies lasting only a few hours, changes in ANP circulating levels may provide a reliable esti mate of production rate variations \citep{201}.

\subsubsection{BNP Metabolism
}
The biological action, metabolic pathways, and turnover parameters of BNP are not as well known as those of ANP \citep{14}. However, it is commonly believed that the BNP turnover is less rapid than that of ANP with a plasma half-life of about 13-20 min utes; indeed, circulating levels of BNP are more stable than those of ANP in adult healthy subjects (Fig. 3.23). Bentzen et al. \citep{197} analyzed the secretion pattern of ANP and BNP in 12 patients with chronic HF and in 12 healthy adult subjects. ANP and BNP in plasma were determined by radioimmunoassay (RIA) at 2 min intervals during a 2-h period and were subsequently analyzed for pulsatile behavior using the method of Fourier transformation. All patients and healthy subjects had significant rhythmic oscillations in plasma ANP levels, and 11 patients with HF and 10 healthy subjects had significant rhythmic oscillations in plasma BNP levels \citep{197}. The ampli tude of the main frequency was considerably higher in patients than in healthy sub jects, but the main frequency did not differ significantly between patients and healthy subjects for either ANP or BNP. Patients with HF demonstrated pulsatile secretion of ANP and BNP with a much higher absolute amplitude, but with the same main fre quency as healthy subjects \citep{197}. Finally, rhythmic oscillations in plasma ANP lev els of healthy subjects showed significantly higher mean amplitude, but not fre quency, than those of BNP \citep{197}.

A very small amount of immunoreactive BNP has been found in urine \citep{202}\citep{203}, but the precise mechanism of renal excretion has not yet been fully clarified. In contrast to BNP, the biologically active peptide, other proBNP-derived inactive fragments also cir culate in plasma. These fragments are commonly referred to as “N-terminal proBNP” (NT-proBNP), but the molecular heterogeneity also includes the intact precursor, par ticularly in patients with HF \citep{14}\citep{204}. Cardiac secretion of proBNP and its N-termi nal fragments has been demonstrated by blood sampling from the coronary sinus \citep{205}.  Some data suggest that the major part of proBNP produced in myocardiocytes is appar ently processed prior to release \citep{14}; however, intact proBNP peptide was also found in plasma of patients with HF as well as healthy adult subjects \citep{14}\citep{205}\citep{206}.

A recent study, employing a new method for the total and equimolar assay of all proB NP-related peptides (i.e., intact proBNP precursor plus NT-proBNP concentrations), found comparable peripheral concentrations of BNP (measured by immunoradiomet ric assay) and proBNP-related peptides in patients with HF \citep{206}. Moreover, the BNP concentration (median 125 pmol/l) was higher than that of total proBNP (103 pmol/l) in the coronary sinus, suggesting that the cardiac secretion of these two peptides could be different \citep{206}.Alternatively, this finding could also reflect some difference in peripheral elimination of peptides because total proBNP concentration is significantly higher in the pulmonary artery than the aortic root in patients with right ventricular failure \citep{207}.

While NEP enzymes are mainly involved in natriuretic peptide inactivation in vivo, the degradation of BNP seen in vitro is most likely due to other enzymes, such as peptyl arginine aldehyde proteases, kallikrein, and serine proteases \citep{15}. However, the role of these enzymes in the degradation of BNP in vivo is unclear.

A recent study reported that both the BNP and total proBNP concentrations were increased more than 2-fold in the coronary sinus compared to the inferior caval vein (BNP-32: median 125 pmol/l, range 21-993 vs median 52 pmol/l, range 7-705; proBNP: median 103 pmol/l, range 16-691 vs 47 pmol/l, 8-500) \citep{206}. These findings are in accordance with previous studies suggesting that the cardiac gradient for BNP secre tion (as estimated by the difference between BNP concentration in coronary sinus and inferior caval vein) ranges from 1.6-fold to 2.9-fold \citep{204}\citep{208-210}. Taking these studies as a whole, ANP and BNP share a similar peripheral extraction value (of about 30-50\%). Further studies are necessary to elucidate the metabolism of BNP and in particular the predominant form of the circulating BNP-related peptides.

\subsection{CNH Genes and Cardiovascular Diseases
}
Since CNH have a potent diuretic antihypertensive action, and the impaired action of the peptides may cause hypertension, their genes may be candidates for cardiovascu lar disease, especially arterial hypertension. Furthermore, transgenic animals (espe cially mice), overexpressing CNH or knockout for ANP/BNP genes or their specific receptors, have been used to evaluate the pathophysiological role of the CNH system in cardiovascular diseases \citep{251}.

In transgenic mice with overexpression of ANP and BNP in liver, plasma ANP and BNP levels are from 10- to 100-fold higher than in control mice, with a blood pressure of 20-25 mmHg lower. These mice also have lighter hearts, but with the same cardiac out put and rate, than controls \citep{251}\citep{253-255}. The BNP-overexpressing mice show the same hemodynamic changes; on the other hand, ANP KO mice develop NaCl-sensitive hyper tension \citep{251}. Transgenic mice overexpressing the NPRA gene have also been created; these animals have a lower blood pressure than wild-type mice \citep{251}. The correspon ding KO mice show an increase in blood pressure compared with controls (on average 10 mmHg in heterozygous and 30 mmHg in homozygous animals), which is not affect ed by NaCl intake \citep{254}\citep{255}. These data suggest a different pathophysiological mech anism for hypertension between KO mice for the ANP gene and its specific receptor; this difference does not yet have an explanation \citep{251}. NPRC heterozygous KO mice do not show blood pressure variation, whereas homozygous mice show on average a decrease in blood pressure of about 8 mmHg \citep{251}.

The function of natriuretic peptides was also studied after induction of myocardial infarction in KO mice lacking the natriuretic peptide receptor guanylyl cyclase-A, the receptor for ANP and BNP \citep{89}. KO and wild-type mice were subjected to left coronary artery ligation and then followed-up for 4 weeks. KO mice showed significantly higher mortality because of a higher incidence of acute HF, which was associated with dimin ished water and sodium excretion and with higher cardiac levels of mRNAs encoding ANP, BNP, TGF-b1, and type I collagen. By 4 weeks after infarction, left ventricular remodeling, including myocardial hypertrophy and fibrosis, and impairment of left ventricular systolic function were significantly more severe in KO than wild-type mice \citep{89}. These data confirm that the CNH system has powerful anti-remodeling properties on ventricular cardiomyocytes.

In recent years, molecular genetic techniques have been introduced in etiological stud ies of polygenetic diseases, in linkage studies, in sib-pair linkage studies of various candi date genes, and in related studies \citep{251}\citep{252}. The association between some abnormalities in genes, coding for the CNH and their receptors, and some cardiovascular (in particular hypertension) and metabolic (such as diabetes mellitus) diseases has been tested in a large number of clinical studies (see also Chapter 6 for more details). To date, the results obtained are conflicting and seem to depend strictly on the ethnic population of the study.  The restriction fragment length polymorphism for the enzyme HpaII, located in intron 2 of NPPA (polymorphism also called Sma I), was reported to be more common in hypertensive African-Americans than in normotensive black controls \citep{257}; these data were then confirmed in two \citep{258}\citep{259}, but not a third \citep{260}, Caucasian popula tions. Furthermore, another study found that the HpaII polymorphism was not asso ciated with hypertension in the Chinese population of Hong Kong \citep{261}.

Regarding other NPPA polymorphisms, Japanese studies reported that both G1837A and T2238C polymorphisms are associated with essential hypertension \citep{262}, while only a marginally significant association was found with an ANP polymorphism locat ed in the 5’-untranslated region (C664G) \citep{263}.

Several allelic variants have also been described for genes coding for CNH recep tors (see the recent revew by Nakayama \citep{251} for a more detailed discussion of this topic). The clearance receptor for natriuretic peptides (NPR-C) is highly expressed in adipose tissue, and its bi-allelic (A/C) polymorphism was detected at position -55 in the conserved promoter element named P1. This variant of the NPR-C P1 promoter is associated with lower ANP levels and higher systolic blood pressure and mean blood pres sure in obese hypertensives: the C(-55) variant, in the presence of increased adiposity, might reduce plasma ANP through increased NPR-C-mediated ANP clearance, con tributing to higher blood pressure \citep{264}.

In the Japanese population an insertion/deletion (GCTGAGCC) polymorphism has been identified in the 5’-flanking region of the NPRA gene that is associated with essen tial hypertension and left ventricular hypertrophy \citep{265}. Another insertion/deletion polymorphism is on the 3’-untranslated region of the NPRA gene, on exon 22, and it seems to be associated with familial hypertension \citep{266}. However, these data should be confirmed in larger studies, including other ethnic populations.

3.9 An Integrated Neuro-Hormonal System Regulates Vascular Function

Endothelial cells release an array of vasoactive mediators that alter the tone and growth of the underlying smooth muscle and regulate the reactivity of circulating white blood cells, erythrocytes and platelets. These endogenous factors are usually called endothe lium-derived vasorelaxant mediators \citep{267}. Moreover, it appears that alterations in the capacity of the endothelium to release some mediators in response to pathophysiolog ical stimuli (the so-called endothelium dysfunction) are a major precipitating factor in many cardiovascular diseases. Perhaps the most important of these paracrine medi ators are prostacyclin (PGI 2 ) and nitric oxide (NO). More recently, a third endotheli um-derived vasorelaxant mediator has been described \citep{267}. This is termed endothe lium-derived hyperpolarizing factor (EDHF) because it elicits a characteristic smooth muscle hyperpolarization and relaxation. Much attention has focused on identifying EDHF(s), with diverse candidates, including cytochrome P450 metabolites, KC ions, anandamide and hydrogen peroxide \citep{267}. However, the role of each of these as EDHF remains unsubstantiated.

There is now compelling evidence that CNH (and especially CNP) act as EDHFs in some vascular beds \citep{267-271}. Indeed, numerous studies have demonstrated that ANP, BNP and CNP bind to NPR-A and NPR-B receptors on vascular smooth muscle cells (either freshly isolated or in culture), stimulate cGMP accumulation, and cause a dose dependent vasodilation \citep{268-271}. This increase in cGMP causes vasodilatation by reducing intracellular calcium levels, as occur when cGMP accumulation is stimulated by NO and its analogs \citep{268}.

It is theoretically conceivable that ANP and BNP act like hormones in vascular tis sue by reaching the smooth muscle cells from the circulation after secretion by the heart, while CNP shows a paracrine action, being secreted by endothelial cells \citep{57}\citep{84}\citep{87}\citep{88} (Fig. 3.15). However, Casco et al. \citep{272} demonstrated the existence of a complete CNH system (including the production and secretion of ANP, BNP and CNP) in ather osclerotic human coronary vessels by means of in situ hybridization and immunocy tochemistry methods. In particular, the expression of mRNAs of ANP, BNP and CNP, measured by RT PCR, tended to be increased in macroscopically diseased arteries com pared to normal vessels, although only the values for BNP expression were significant ly different \citep{272}. This study suggests that the CNH system is involved in the pathobi ology of intimal plaque formation as well as in vascular remodeling in humans.  Some studies indicated that there are complex interactions even among CNH them selves. Nazario et al. \citep{273} reported that ANP and BNP can stimulate CNP production through a guanylate cyclase receptor on endothelial cells. As a result, vasodilatory, and anti-mitogenic effects of ANP and BNP in the vasculature could occur in part through CNP production and subsequent action if these interactions occur in vivo. In other words, ANP/BNP and CNP paracrine system should share a synergic action on vascu lar tissues.

Several studies have demonstrated complex interactions betwen CNH and the other endothelium-derived vasorelaxant mediators \citep{267}. Indeed, evidence from cellular, ani mal, and human studies suggests that all CNH are able to stimulate NO production by endothelial NO synthase (eNOS); this effect is probably mediated by clearance recep tor NPR-C \citep{270}. Stimulation of this NPR-C receptor results in decreased cAMP levels by adenyl cyclase inhibition through an inhibitory guanine nucleotide-regulating pro tein \citep{270}. Furthermore, ANP treatment increases renal and cardiac NO synthesis in rats \citep{274}. On the other hand, NO, released from endothelial cells, negatively modulates ANP secretion from atrial myocytes, induced by mechanical stretch in perfused rat heart preparation \citep{275}. Furthermore, ANP expression is markedly upregulated in eNOS -/- mice, and exogenous ANP restores ventricular relaxation in wild-type mice treated with NOS inhibitors \citep{276}. These data suggest that the CNH and NO systems are linked by a negative feedback mechanism. Finally, CNH (and especially CNP) mimic many of the anti-atherogenic actions of PGI 2 and NO \citep{267}. This gives rise to the pos sibility that CNP might compensate for the loss of these mediators in cardiovascular pathologies to restore the vasodilator capacity of the endothelium, in addition to its anti-adhesive and anti-aggregatory influences.

CNH also strongly interact with the effectors of counter-regulatory systems at the vas cular tissue level \citep{13}\citep{28}\citep{30}\citep{77}\citep{78}\citep{90-99}. In particular, interactions between CNH and ET-1 also appear to be important physiologically; indeed, the vascular effects of CNH are directly opposite to those of ET-1 \citep{267}\citep{269}; in particular, ET-1-induced vasocon striction and myocyte hypertrophy is inhibited by CNH. While CNP has little natri uretic and diuretic action compared to ANP or BNP, it is capable of modulating the vascular effects of the local RAAS by opposing potent vasoconstriction to angiotensin II \citep{269}. CNP not only functionally antagonizes ET-1 and angiotensin II, but it also directly modulates ET-1 \citep{277} and angiotensin II \citep{278} synthesis. On the other hand, ET 1 induces an increase in the number of endothelial cells that secrete CNP \citep{279}. There fore, the parallel production and activity of vasodilator CNP and vasoconstrictors such as ET-1 and angiotensin II allows for tight local regulation of these vasoactive peptides and thus blood flow \citep{267}\citep{269}\citep{279}.

Furthermore, the inter-relationships between the CNH system and pro-inflammatory cytokines suggest that cardiac hormones play an important role in mechanisms respon sible for cardiac and vascular adaptation, maladaptation and remodeling in response to various physiological and pathological stimuli \citep{32}\citep{35}\citep{62}\citep{162}. The identification of CNP as an EDHF, combined with its expression in endothelial cells, indicates that CNP is suited to modulate the activity of circulating cells, particularly leukocytes and platelets.  Moreover, inflammatory stimuli such as IL-1b, TNF and lipopolysaccharide \citep{280} stim ulate the release of CNP from isolated endothelial cells. As a result, modulation of the biological activity of CNP is likely to have a profound influence on the development of an inflammatory response. Certainly, an anti-atherogenic activity of CNP fits with the cytoprotective, anti-inflammatory actions of NO and PGI 2 , the other major endotheli um-derived vasorelaxants \citep{267}\citep{269-271}\citep{280}.

From a clinical point of view, it is important to note that exogenous application of CNP in situations where endothelial NO production is compromised might be therapeutic in disorders that are associated with endothelial dysfunction. For example, overex pression of CNP by adenoviral-gene delivery in veins dramatically reduces the lumi nal narrowing (neointimal hyperplasia) that develops when it is grafted to the carotid artery, thereby retaining patency of the graft \citep{281}. CNH, including CNP, also suppress the production of pro-inflammatory cyclooxygenase 2 metabolites in isolated cells \citep{106}\citep{282}. Other studies demonstrated a direct effect of CNP on immune-cell recruitment in vivo \citep{267}\citep{271}. Therefore, like NO, endothelial CNP (like ANP and BNP) exerts a pro tective anti-inflammatory effect \citep{104-106}\citep{267}\citep{271}\citep{283}. This inhibitory effect of CNH on leukocytes indicates that these peptides modulate the expression of adhesion mol ecules on either the endothelium or leukocytes.

Several data support the thesis that CNH (and especially CNP) are important, endoge nous, anti-atherogenic mediators. CNP is a potent inhibitor of vascular smooth muscle migration and proliferation that is stimulated by oxidized LDL \citep{277}. CNP also inhibits the proliferation of vascular smooth muscle \citep{284}, and enhances endothelial cell regen eration in vitro and in vivo \citep{281}. The observation that CNP alters leukocyte-endothe lial interactions indicates that it might also affect platelet function. In accordance with this, thrombus formation is suppressed significantly in the presence of CNP \citep{280}, which indicates that inhibition of coagulation might contribute to the vasoprotective proper ties of this peptide. Observations that CNP blocks platelet aggregation, induced by throm bin, confirm that endothelium-derived CNP also exerts an anti-thrombotic effect \citep{267}.

All the above-mentioned studies demonstrate that CNH (and especially CNP) exert a protective effect on endothelial function by decreasing shear stress, modulating coagulation and fibrinolysis pathways, and inhibiting platelet activation (Fig. 3.15). They can also inhibit the vascular remodeling process as well as coronary restenosis post-angioplasty \citep{56}\citep{84-89}\citep{267}\citep{281}\citep{283}. These vasoprotective actions should be considered as a result of complex inter-relationships between the CNH system and both the synergic (including NO, PGI 2 , and other endothelium-derived vasoactive mediators) and the counter-regulatory systems (including endothelins, RAAS, cytokines, and growth factors).

3.10 Summary and Conclusion

Natriuretic peptides (including ANP, BNP, CNP, DNP and urodilatin) constitute a fam ily of peptide hormones and neurotransmitters, sharing a similar peptide chain, char acterized by a cysteine bridge (Fig. 3.1). The physiological relevance of these peptides is well demonstrated by their presence since the first dawning of life, from unicellular to pluricellular organisms, including plants and all animals. Furthermore, their genes have been repeatedly doubled during evolution, starting from an ancestral gene, thus sug gesting that these peptides are indispensable for life (Fig. 3.6).

CNH have powerful physiological effects on the cardiovascular system, body fluid, and electrolyte homeostasis. These effects can be explained at least in part by the inhibition of counter-regulatory systems, including the sympathetic nervous system, RAAS, the vaso pressin/antidiuretic hormone system, the endothelin system, cytokines and growth factors.  The endocrine action shared by plasma ANP and BNP can be enhanced by natri uretic peptides produced locally in target tissues (paracrine action). Indeed, endothe lial cells synthesize CNP, which exerts a paracrine action on vessels. Moreover, renal tubular cells produce urodilatin, another member of the peptide natriuretic family, which shows powerful diuretic and natriuretic properties. Genes for natriuretic pep tides (including ANP, BNP and CNP) are also expressed in the central nervous system, where they likely act as neurotransmitters and/or neuromodulators. Finally, co-expres sion of CNH and their receptors was observed in immunocompetent cells, suggesting that CNH may have immunomodulatory and anti-inflammatory functions in mam mals. Furthermore, CNH are expressed in almost all the body tissues as well as their specific receptors, including organs and tissues not discussed in this chapter, such as gut \citep{285}, skeletal \citep{106} and ocular \citep{286} tissues. In all tissues, CNH could also act as a local mediator or paracrine effector of tissue-specific functions.

These data, taken as a whole, strongly suggest that natriuretic peptides constitute a family sharing endocrine, paracrine and autocrine actions and neurotransmitter and immunomodulator functions. Therefore, it can be hypothesized that the CNH system is closely related to the other regulatory systems (nervous, endocrine and immuno logical) in a biological hierarchical network (Fig. 3.16) [287, 288].

\bibliography{thesis}

\end{document}
