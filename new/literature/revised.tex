\subsection{Chemical Structure and Gene Evolution of Cardiac Natriuretic Hormones
\subsection{Genes Encoding for Cardiac Natriuretic Hormones
In conclusion, a huge number of experimental and clinical studies demonstrated that production and secretion of CNH (especially BNP) are not only related to hemo dynamic variations, but also subtly regulated by neuro-hormonal and immunological factors. Therefore, the variation of CNH circulating levels can be considered as a sen sitive index of the perturbation of the homeostatic systems.  3.4 Biological Action of CNH It is important to note that specific receptors for CNH have been found in all mam malian tissues, although at different concentrations, thus suggesting that CNH play an important biological role in several tissues. Most of these biological effects of CNH may be due to an autocrine/paracrine, rather than hormonal, action. In this section, we will discuss in detail only the best-known effects of CNH, and in particular those more strictly related to the endocrine function of the heart.
\subsection{Natriuretic Peptide Receptors and Intracellular Second Messenger Signaling
\subsubsection{Genes Encoding for NPRs
\subsubsection{Biological Function of NPRs
\subsection{Metabolic Pathways and Circulating Levels of CNH
\subsubsection{ANP Metabolism
\subsubsection{BNP Metabolism
\subsection{CNH Genes and Cardiovascular Diseases
\section{Cardiac Natriuretic Hormones as Markers of Cardiovascular Disease: Methodological Aspects}
\subsection{ General Considerations}
\subsection{ Determination of ANP and NT-proANP}
\subsubsection{ Competitive Immunoassay Methods for ANP}
\subsection{. These data indicate that all assays are more specific for the intact ANP molecule}
\subsubsection{ Non-Competitive Immunoassay Methods for ANP}
\subsubsection{ Determination of NT-proANP}
\subsection{ Determination of BNP and NT-proBNP}
\subsubsection{ First-Generation (Manual) BNP Assays}
\subsection{ ml of plasma was necessary for the assay \citep{bib213} \citep{bib257}.}
\subsubsection{ Second-Generation (Automated) BNP Assays}
\subsubsection{ Determination of NT-proBNP}
\subsubsection{ Specificity of BNP/NT-proBNP Assays}
\subsubsection{ Quality Specifications for BNP/NT-proBNP Assays}
\subsection{ Determination of Other Natriuretic Peptides (CNP, DNP, and Urodilatin)}
\subsection{ Summary and Perspectives}
\subsection{ Circulating Levels of Cardiac Natriuretic Hormones:}
\subsubsection{ Influence of Age and Gender}
\subsubsection{ Comparison between the CNH Assay and that of CNH-Related Pro-Hormone Peptides}
\subsubsection{ Resistance to the Biological Action of CNH}
these durgs can potentiate the biological activity of CNH system by increasing the concentration of biologically active peptides \citep{bib339} \citep{bib340} \citep{bib341} \citep{bib342} (see Chapter 7, sections 7.4 and 7.5, for more
6, section 6.9, and Chapter 7, section 7.5, for more detailed clinical information).
\subsubsection{ Diagnostic Accuracy of CNH Assay in Plasma from Patients with Cardiac Diseases}
\subsubsection{ Biological Variation of Plasma BNP: a Problem or a Clinical Resource?}
\subsection{ CNH Assay as Diagnostic and Prognostic Tool in Cardiac Diseases}
\subsubsection{ Use of CNH Assay in the Screening and Classification}
\subsubsection{ Diagnostic Accuracy of CNH Assay in Asymptomatic,}
\subsubsection{ Diagnostic Accuracy of CNH Assay in Patients with Suspected HF}
\subsubsection{ Diagnostic Accuracy of CNH Assay in Patients with Acute Myocardial Infarction}
\subsubsection{ Diagnostic Accuracy of CNH Assay in Elderly People}
\subsubsection{ Detection of Drug Cardiotoxicity by means of BNP Assay}
\subsubsection{ Diagnostic Accuracy of CNH Assay in Coronary Artery Disease}
\subsection{ Comparison between the Diagnostic Accuracy}
\subsection{ Use of CNH Assay as Prognostic Marker in Cardiovascular Diseases}
\subsubsection{ Prognosis in HF}
\subsubsection{ Prognosis in ACS}
\subsubsection{ Prognostic Relevance of CNH Assay in the General Population}
\subsection{ CNH Assay in Management of Patients with HF}
\subsubsection{ Clinical Relevance of CNH Assay in Tailoring the Therapy of HF}
\subsubsection{ Cost-Effectiveness of CNH Assay in Management of Patients with Acute or Chronic HF}
\subsection{ Summary and Conclusion}
\section{Introduction}
\section{Review of Literature}
\subsection{Natriuretic Peptides}
\subsubsection{Atrial Natriuretic Peptide}
\subsubsection{B-Type Natriuretic Peptide}
\subsubsection{C-Type Natriuretic Peptide}
\subsection{Natriuretic Peptide Receptors}
\subsubsection{Natriuretic Peptide Receptor-A}
\subsubsection{Natriuretic Peptide Receptor-B}
\subsubsection{Natriuretic Peptide Receptor-C}
\subsection{Physiologic Effects of Natriuretic Peptides}
\subsubsection{Natriuretic Peptide Effects on Blood Pressure}
\subsubsection{Effects of Natriuretic Peptides on Cardiac Hypertrophy and Fibrosis}
\subsubsection{Effects of CNP and NPR-B on Bone Growth}
\subsection{Therapeutics of Natriuretic Peptides}
Measurement of serum BNP levels is used in the clinic as a diagnostic indicator for heart failure, and synthetic forms of both ANP and BNP have been approved in some countries for the treatment of heart failure \citep{Gardner2003a}. The extent of their usefulness, however, has come under question due to their limited renal actions, and trials are underway to determine the most effective use of these peptides. In this section, we will explore the history of both synthetic ANP and BNP as therapeutic agents. %TO-DO search for more recent data
\subsubsection{Synthetic ANP (Anaritide and Carperitide)}
\subsubsection{Synthetic BNP (Nesiritide)}
\subsection{BNP and NT-proBNP}
\subsubsection{Differences in Physiology}
\subsubsection{BNP and NT-proBNP in clinical practice}
\subsubsection{Variables influencing BNP and NT-proBNP levels: potential limitations?}
% \subsection{Coronary Artery Disease}
% \subsection{Coronary Artery Bypass Grafting}
% \subsubsection{on-pump vs OPCAB}
\section{Material and Methods}
\subsubsection{Exclusion Criteria}
\subsubsection{Definitions}
\subsubsection{Primary and Secondary Outcomes}
\subsubsection{Data Analysis and Statistical Methods}
\section{Results}
\section{Discussion}
We are aware of four previous studies that tried to answer the same question of the current thesis; is preoperative natriuretic peptides of prognostic value when it comes to cardiac surgery patients.  The tables shown in this section shows the differences and similarities between those studies in design; cohort characteristics, peptide used and frequency and timing of samples, variables observed and duration, and their results.
\section{Conclusion}
\section{Summary}
