\documentclass[14pt,a4paper,onecolumn]{extarticle}
\usepackage[utf8]{inputenc}
\usepackage{amsmath}
\usepackage{amsfonts}
\usepackage{amssymb}
\usepackage{graphicx}
\usepackage{natbib}
\usepackage{multirow}
\usepackage{pdflscape}

\bibliographystyle{plainnat}
\setlength{\parskip}{1.2em}
\setlength{\parindent}{0pt}
% \graphicspath{{../images/}}

\author{Ibrahim AbuBakr ElSeddiq}
\title{The predictive value of NTproBNP on postoperative outcome in patients undergoing offpump CABG}

\begin{document}

\maketitle
\clearpage



\subsection{Chemical Structure and Gene Evolution of Cardiac Natriuretic Hormones
}
All natriuretic peptides share a similar structural conformation, characterized by a peptide ring with a cysteine bridge (Fig. 3.1). This ring is well preserved throughout the phylogenetic evolution, since it constitutes the portion of the peptide hormone that binds to its specific receptor (Fig. 3.2). Conversely, the two terminal amino acid chains (i.e., NH 2 - and COOH-terminus) show a high degree of variability among the natriuretic peptides, in terms of both length and amino acidic composition \citep{1}.

Natriuretic hormone-like peptides seem to be present in the plant kingdom as well as in the animal kingdom \citep{1} \citep{2} \citep{3}; thus indicating that this hormonal system, which has been shown to regulate solute transport in vertebrates, has evolved early in evolution \citep{1} \citep{2} \citep{3} \citep{4} \citep{5} \citep{6}.  ANP-, BNP- and CNP-like peptides have been identified in tetrapods, ranging from amphibians to mammals \citep{1}. Natriuretic peptides also exist in fish as a family of structurally related iso-hormones, including ANP, CNP and ventricular natriuretic peptide (VNP) (Fig. 3.2) \citep{1}. A recent study has indicated that BNP is also present in some fish \citep{6}. In vertebrates, ANP, BNP and VNP are endocrine hormones secreted from the heart, while CNP is principally a paracrine factor in the brain and periphery. In elasmobranches, only CNP is present in the heart and brain, and it functions as a circulating hormone as well as a paracrine factor \citep{1}. Immunoreactive ANP-like pep tide has also been detected in the unicellular Paramecium and in various species of plants \citep{1}.

CNP is the most structurally conserved member of the natriuretic peptide family (Fig. 3.3) \citep{1} \citep{4}. The ANP sequence is conserved in mammals; however, the identity across different groups (e.g. between mammals and frogs) is rather low (about 50\%) (Fig. 3.4). BNP is highly variable even in mammalian species (Fig. 3.5) \citep{1} \citep{4}.  It is noteworthy that the complete pro-hormone sequences of ANP (proANP) and CNP (proCNP) are more conserved within mammals, but they are highly variable across different classes of vertebrates, except at the C-terminal mature ANP and CNP sequences.  Thus, the N-terminal part of pro-hormones ANP (NT-proANP) and CNP may not have common biological function throughout vertebrate species \citep{1}. The NT-proBNP is not well conserved even in mammals, as observed for BNP as well \citep{1}.

Studies in fish, based on nucleotide and amino acid sequence similarity, suggest that the natriuretic peptide family of iso-hormones may have evolved from a neuromodulatory CNP-like brain peptide (Fig. 3.6) \citep{4}. However, caution should be exercised in identification and comparison of vertebrate hormones from phylogenetically distant organisms \citep{1}.

\subsection{Genes Encoding for Cardiac Natriuretic Hormones
}
The ANP and BNP coding genes, named NPPA and NPPB, are tightly linked on human chromosome 1 and mouse chromosome 4 \citep{7} \citep{8}. Arden et al. \citep{7} confirmed the assign ment to the 1 p 36 region by FISH and Southern blot analysis. Moreover, pulsed-field gel electrophoresis placed NPPA and NPPB within 50 kb of each other.

The Gene Cards Data Base reports that NPPA maps on 1p36.2 and that it is less than 10 kb more telomeric than the BNP coding gene. NPPA spans 2,075 bp and constitutes three exons and two introns (Fig. 3.7). The splicing produces an mRNA of 840 bp that is translated in the ANP precursor of 153 amino acids.

NPPB is also organized in three exons and two introns that span 1,466 bp. The mRNA is 691 bp and the relative primary product is a peptide of 134 amino acids (Fig. 3.7).  The NPPA and NPPB genes are expressed in almost all tissues but, for both, the heart is the organ in which the expression is higher (Figs. 3.8 and 3.9).

On the basis of PCR-analyzed microsatellite length polymorphisms among recom binant inbred strains of mice, Ogawa et al. \citep{10} found that the CNP gene is located on mouse chromosome 1 (Fig. 3.10). Using somatic hybrid cell methodology, the human CNP gene was assigned to chromosome 2 \citep{10}. Extrapolating from studies of homology of synteny, they suggested that the gene endocing CNP may lie in the 2q24-qter region.  This gene constitutes 824 bp, organized in two exons separated by an intron, and its mRNA (of 381 bp) gives a C natriuretic peptide precursor of 126 amino acids (Fig. 3.10).

The natriuretic peptide genes encode for the precursor sequences of these hormones, named pre-pro-hormones, which are then split into pro-hormones by proteolytic cleav age of an N-terminal hydrophobic signal peptide. This cleavage occurs cotranslation ally during protein synthesis in the rough endoplasmic reticulum, before the synthesis of the C-terminal part of the pro-hormone sequence is completed \citep{11} \citep{12} \citep{13}.  The pro-hormone of ANP is stored as a 126-amino acid peptide, proANP 1-126 (also called γANP), which is produced by cleavage of the signal peptide (Fig. 3.11). When appropriate signals for hormone release are given, proANP 1-126 is further split into an NH 2 terminal fragment, proANP 1-98 (actually called NT-proANP) and the COOH-terminal peptide ANP 99-126 (ANP), which is generally considered to be the biologically active hormone \citep{11} \citep{12} \citep{13}.

The human BNP gene encodes for a preproBNP molecule of 134 amino acid residues with a signal peptide of 26 amino acids (Fig. 3.12). BNP is produced from a pro-hormone molecule of 108 amino acids, the proBNP 1-108 , usually indicated as proBNP. Before secre tion, proBNP is split by proteolytic enzymes into two peptides: the proBNP 1-76 (NH 2 terminal peptide fragment, usually indicated as NT-proBNP), which is biologically inactive, and the proBNP 77-108 (COOH-terminal peptide fragment), which is the active hormone (BNP) \citep{14}.

It is important to note that the preproBNP precursor is not detectable, and its existence is only a theoretical concept, deduced from the BNP cDNA sequence of the human (or other mammalian) gene \citep{14}. On the other hand, intact proBNP, NT-proBNP, and BNP can be identified in plasma by chromatography and immunoassay \citep{14} \citep{15} \citep{16} \citep{17}. Moreover, ANP and BNP can be produced and co-stored in the same granule in different stages of peptide maturation \citep{11} \citep{12} \citep{13}.

3.3 Regulation of Production/Secretion of ANP and BNP in Cardiac Tissue

Atrial natriuretic peptide and BNP are synthesized and secreted mainly by cardiomyocytes. However, it is generally thought that ANP is preferentially produced in the atria, while BNP is produced in the ventricles, particularly in patients with chronic cardiac diseases. Synthesis and secretion of the two different peptides may be differently regulated in atrial and ventricular myocytes, and, probably, even at different stages of life (neonatal or adult life) \citep{12} \citep{13} \citep{14} \citep{18} \citep{19} \citep{20} \citep{21} \citep{22} \citep{23}. Furthermore, there could be some differences in regulation of expression of CNH genes among mammalian species; for example, between rat and human in cis-acting regulatory elements responsible for BNP promoters \citep{22}.

These findings suggest some caution in the use of study results performed in experimental animals to interpret specific pathophysiological conditions in humans. Unfortunately, our present knowledge on regulation of CNH gene expression derives in great part from experimental studies in rodents \citep{18} \citep{19} \citep{20} \citep{21} \citep{22} \citep{23}.

Some recent data suggest that not only cardiomyocytes, but also fibroblasts may produce CNH in human heart \citep{24}; however, the clinical relevance of this finding has not been ascertained. According to studies performed in experimental animals, it has also been proposed that the endocrine response of the heart to pressure or volume load varies depending on whether the stimulus is acute, subacute, or chronic \citep{12} \citep{13} \citep{18} \citep{22}.

Atrial cardiomyocytes store pro-hormones (proANP and proBNP) in secretory granules, and split them into ANP and BNP before secretion.ANP and BNP are produced and stored in different stages of peptide maturation even though they are co-stored in the same granule. It was suggested that the prevalent peptide form in the atrial granules is unprocessed proANP, whereas BNP is mainly stored in atrial granules as the processed and mature form of the active peptide hormone \citep{12} \citep{13} \citep{18}.

Peptide hormones are secreted via several pathways including regulated secretion, constitutive secretion and constitutive-like secretion. Regulated secretion occurs upon stimulation by agonists whereby the hormone packaged is released from the storage granules, thus allowing endocrine cells to secrete hormones at a rate that exceeds its biosynthesis \citep{13}. The constitutive-like pathway is independent from both regulated and constitutive secretion and is insensitive to cycloheximide treatment \citep{13}. The constitutive pathway involves the passive diffusion of hormone in the absence of stimuli.  The presence of clathrin-coated vesicles in the atrial cardiocytes is indicative of the existence of this constitutive pathway \citep{13}.

CNH (especially ANP) should be predominantly secreted throughout a regulated pathway \citep{12} \citep{13} \citep{18}. There is also the possibility that a small amount of CNH is released via a constitutive pathway involving passive diffusion of secretory products \citep{12} \citep{13} \citep{18}.  Indeed, protein synthesis inhibition by cycloheximide significantly, though partially, decreases ANP secretion; this suggests that basal ANP secretion is to a degree, dependent on newly synthesized hormone \citep{13}. However, a component of basal secretion appears to depend upon stored hormone. Some studies have demonstrated that 60\% of the newly synthesized ANP is intended for storage, whereas the remaining 40\% is secreted under unstimulated conditions \citep{13}.

It is likely that circulating CNH (especially ANP) mostly derives from atria in healthy subjects \citep{12} \citep{13} \citep{14} \citep{18}. An acute atrial stretch may result in an immediate, sharp increase in the release rate of ANP at the expense of a rapidly depleting pool of newly synthesized hormone. The BNP pool that is utilized in response to stretch remains to be determined but in terms of peptide release, the ratio of ANP/BNP released is roughly the same as found in tissue storage, i.e., the response is in keeping with the lesser proportion of BNP present in mature atrial granules \citep{18}. The response of the CNH system to stretch may not be strictly reflected in vivo by an increase in plasma levels of ANP and BNP, probably because the amount of peptide released is masked by dilutional effects and different clearance rates of the two peptide hormones.

The BNP gene is expressed in both atrial and ventricular myocytes of normal and diseased heart \citep{12} \citep{13} \citep{14} \citep{18}. Ventricular myocytes in the normal heart of adult mammals do not usually show any evident secretory granules on electron microscopy \citep{13} \citep{14}.  However, some authors have identified secretory granules, similar to the atrial ones, in samples of ventricular myocardium collected during surgery or in endocardial biopsies studied by electron microscopy and immunohistocytochemistry in patients with cardiac disease \citep{14} \citep{25} \citep{26}. These studies suggest that normal ventricular myocardium may produce only a limited amount of BNP in response to an acute and appropriate stimulation, probably via a constitutive secretory pathway, while the amount of hor mone produced and secreted after chronic stimulation could be greatly increased via an upregulated secretory pathway. However, further studies are necessary in order to con firm the presence of secretory granules in human ventricular cardiomyocytes and in par ticular to evaluate their peptide content and pathophysiological relevance.  BNP mRNA levels increase predominantly by upregulated gene transcription and to a lesser degree, by post-transcriptional mechanisms \citep{22}. BNP transcription can increase with the activation of numerous positive cis-acting regulating elements or the inhibition of negative cis-acting elements on the 5’-flanking region on the BNP pro moter, as recently reviewed in detail \citep{18} \citep{22}. The type of the element depends on whether the stimulus is a mechanical or a neuro-humoral agonist \citep{22} \citep{23}. Due to their ability to cooperate with a diverse group of transcription factors and to react to a variety of dif ferent stimuli, some proteins of the GATA family (especially GATA-4 and GATA-6) emerge as crucial factors in regulating basal expression and inducible ANP and BNP pro moter activity \citep{18} \citep{23}. The transcription factors of the GATA family (from GATA-1 to GATA-6) are zinc-finger proteins that bind to the consensus sequence (A/T)GATA(A/G) via a DNA-binding domain containing two zinc fingers to activate target genes. GATA proteins play important roles in cell differentiation and homeostasis in all eukaryotes \citep{23}. It has been suggested that endothelin-1, angiotensin II and adrenergic agonists increase the expression of BNP mRNA in cultured isolated cardiomyocytes by activat ing the proximal GATA elements \citep{22}. The muscle-CATTCCT (M-CAT) consensus ele ment, the shear stress responsive element (SSRE), and the thyroid hormone response ele ment (TRE) could be other additional cis-acting factors regulating inducible BNP expres sion \citep{18} \citep{22}. Like GATA proteins, these elements can also be selectively activated by specific stimuli, especially by some neuro-hormones (such as b-adrenergic agonist and thyroid hormones) and cytokines (such as interleukin-1\beta [IL-1\beta] and tumor necrosis factor-\alpha [TNF-\alpha]) \citep{22}.

From a clinical point of view, it is important to note that chronic stimulation produces a greater amount of BNP than ANP, probably because the former is produced mainly by ventricular myocardium, which has a relatively greater mass than atria. However, ven tricular BNP gene expression can be selectively upregulated during the evolution of diseases affecting the ventricles, as demonstrated in an experimental model of heart failure (HF), i.e., the rapid ventricular pacing-induced congestive HF in dog \citep{27}. In fact, on average, the molar ratio of circulating BNP over ANP increases progressively with the severity of HF from a value of about 0.5 in healthy subjects up to about 3 in patients with NYHA functional class IV (Table 3.1, Fig. 3.13) \citep{28}.

These data explain why the BNP assay usually shows a better diagnostic accuracy in patients with cardiac disease than the ANP assay \citep{28}.

ANP and BNP are secreted from the heart into the circulation, thus providing for a baseline level of the hormones in blood \citep{13}. The response of the heart to pressure or volume load varies in relation to whether the challenge is acute, subacute or chronic \citep{12} \citep{13}.

Wall stretch is the most important stimulus for synthesis and secretion of ANP at the atrial level \citep{12} \citep{13} \citep{21} \citep{22} \citep{23}. The increased secretion of ANP following acute mechan ical atrial stretch is based on a phenomenon referred to as “stretch-secretion coupling”.  This effect is dependent upon a depletable ANP pool and is characterized by a phasic, short-term (i.e. minutes) burst of CNH secretion with no apparent effect on synthesis \citep{13}. The precise mechanism by which force is translated into biochemical stimulus has not been completely elucidated. However, several studies have documented the importance of outside-in signaling by extracellular matrix proteins (especially inte grin) in translating mechanical stress to changes in gene expression and the induction of ANP and BNP in hypertrophic myocardium \citep{22}.

Any physiological condition associated with an acute increase in venous return (pre load), such as physical exercise, rapid change from standing to supine position, or head out water immersion, causes a more rapid augmentation in ANP than in BNP plasma concentration. For instance, changes in ANP and BNP secretion have been well char acterized during and after the tachyarrhythmia induced in pigs by rapid atrial pacing (225 beats per minute). In this model, ANP plasma concentration shows a sharp initial peak followed by a decline, but remains significantly increased throughout a 24-hour post pacing period, while BNP increases significantly after an 8-hour pacing, and even more after a 24-hour pacing \citep{29}. Even acute changes in the effective plasma circulating vol ume, such as during a dialysis session in patients with chronic renal failure, cause greater variations in circulating levels of ANP than BNP \citep{30}.  Whereas chronic stimulated CNH secretion results in increased synthesis and secre tion in both atria and ventricles, there is an intermediate level stimulation of the endocrine heart whereby increased synthesis and secretion of CNH is evident only in the atria \citep{12} \citep{13}. This can be observed during the “mineralocorticoid escape”, a patho physiological condition characterized by a transient period of positive sodium balance resulting from chronic exposure to mineralocorticoid excess, such as aldosterone or deoxycorticosterone acetate followed by a vigorous natriuresis leading to a new steady state of sodium balance \citep{12} \citep{13}. The rise in intravascular volume and central venous pres sure leads to increased CNH production by the atria.

Wall distension is generally considered the main mechanical stimulus for CNH (especially BNP) production by ventricular tissue. This occurs in long-standing con ditions characterized by electrolyte and fluid retention, and therefore expansion of effective plasma volume, such as primary \citep{31} and secondary hyperaldosteronism, including cardiac, renal and liver failure \citep{12} \citep{13} \citep{18} \citep{30}. The changes in the pattern of gene expression are observed not only in the long-term hypertrophic process of ven tricular myocardium, but also at the onset of hemodynamic overload. Moreover, expres sion of the BNP gene takes place with many characteristics of an immediate-early gene. Indeed, hemodynamic overload in the left ventricle has been shown to result in an increase in the BNP gene expression within 1 h, associated with the expression of oncogenes c-fos and c-jun \citep{32}. A recent study suggested that the stretch-induced acti vation of BNP gene expression by increased left ventricular wall stress in an isolated perfused rat heart preparation is independent of transcriptional mechanisms and dependent on protein synthesis \citep{32}; other studies are necessary to confirm and clar ify these findings.

The presence of ventricular hypertrophy and fibrosis can stimulate hormone produc tion \citep{12} \citep{13} \citep{18} \citep{19} \citep{20} \citep{21} \citep{30} \citep{33} \citep{34} \citep{35} \citep{36}. However, recent studies have suggested that myocardial fibro sis rather than hypertrophy is associated with increased production of BNP \citep{24} \citep{35} \citep{36}.

More recently, several studies indicated that myocardial ischemia and hypoxia per se could also induce the synthesis/secretion of CNH by cardiomyocytes \citep{37} \citep{38} \citep{39} \citep{40} \citep{41} \citep{42} \citep{43} \citep{44}. The stud ies indicating that ANP gene is responsive to hypoxia have been reviewed recently in detail \citep{37}. Several data indicated that hypoxia directly stimulates ANP gene expression and its release in cardiac myocytes in vitro \citep{37}. The effect of hypoxia on BNP produc tion/secretion was also studied, demonstrating that surgical reduction of blood in an area of the anterior ventricular wall in pigs increased BNP mRNA by 3.5-fold in hypoxic compared with normoxic ventricular myocardium \citep{44}. Moreover, proBNP peptide accumulated in the medium of freshly harvested ventricular myocyte cultures, but was undetectable in ventricular myocardium, indicating rapid release of the newly synthe sized proBNP peptide \citep{44}. The direct stimulating effect of hypoxia on CNH gene expres sion is probably due to the activation of promoter activity; however, other potential mechanisms could modulate peptide hormone release from cardiomyocytes, includ ing the influx of extracellular Na + and Na + /Ca ++ exchange (due to the hypoxia-depend ent intracellular acidosis) as well as the activation of protein kinase C \citep{37}. Moreover, clinical studies reported that plasma levels of CNH (especially BNP and its related pep tides) were found to be closely related to aerobic exercise capacity in patients with HF \citep{45} \citep{46} \citep{47}. In particular, plasma NT-proBNP correlates better than indices of left ventric ular systolic function, such as ejection fraction, with peak oxygen consumption and exercise duration \citep{47}. These results may explain the elevated levels of BNP found even in patients with acute coronary syndrome (ACS), in the absence of a significant dilata tion of the ventricular chambers \citep{40}. This suggests a neuro-hormonal activation sec ondary to both reversible myocardial ischemia and necrosis \citep{41}.

There is increasing evidence from in vivo and ex vivo studies supporting the hypoth esis that the production/secretion of CNH is regulated by complex interactions with the neuro-hormonal and immune systems, especially in ventricular myocardium \citep{12} \citep{13} \citep{18}. Neuro-hormones, cytokines, and growth factors that can affect the produc tion/secretion of CNH are summarized in Table 3.2.

Endothelin and angiotensin II are considered the most powerful stimulators of pro duction/secretion of CNH \citep{12} \citep{13} \citep{18} \citep{22}; similarly, glucocorticoids, sex steroid hor mones, thyroid hormones, some growth factors and mediators of inflammation (such as some cytokines, especially TNF-\alpha, interleukin-1, interleukin-6, and lipopolyliposaccha ride, LPS) share stimulating effects on the CNH system \citep{12} \citep{13} \citep{18} \citep{22} \citep{36} \citep{48} \citep{49} \citep{50} \citep{51} \citep{52} \citep{53} \citep{54} \citep{55} \citep{56} \citep{57} (Table 3.2). The interesting finding that CNH production is stimulated by cytokines and growth factors suggests a link between cardiac endocrine activity and remodelling or inflam matory processes in myocardial and smooth muscle cells. A large number of studies have recently contributed to support this hypothesis \citep{33} \citep{34} \citep{35} \citep{36} \citep{50} \citep{51} \citep{52} \citep{53} \citep{56} \citep{57} \citep{58} \citep{59} \citep{60} \citep{61} \citep{62}.

More complex, and still in part unknown, is the effect of adrenergic stimulation on CNH production. The \alpha 1 -adrenergic agonist phenylephrine enhances the expression of some transcription factors, such as Egr-1 and c-myc, regulating (usually increasing) the natriuretic peptide gene expression in cultured neonatal rat cardiomyocytes \citep{12} \citep{13} \citep{18} \citep{63} \citep{64} \citep{65} \citep{66}. Conflicting results were obtained with \beta-agonists. In one study, the \beta agonist isoproterenol reduced the expression of BNP mRNA, but not that of ANP, an effect prevented by the \beta 1 -antagonist CGP20712A in isolated adult mouse cardiomy ocytes \citep{67}. In another study, isoproterenol stimulated the BNP mRNA expression in rat ventricular myocyte-enriched cultures \citep{68}. However, it was suggested that the stimu latory effects of both \alpha- and \beta-adrenergic agonists on BNP gene inducible transcription are principally mediated by GATA elements \citep{22} \citep{23}.

Clinical studies performed in hypertensive patients have shown that monotherapy with a \beta-blocker, either \beta 1 -selective or not, is associated with an increase in the plasma concentration of ANP and/or BNP and their related peptides \citep{69} \citep{70} \citep{71}. In contrast, CNH response can be heterogeneous during \beta-blocker therapy in congestive HF \citep{28} \citep{72}, probably due to the various additive effects of other co-administered drugs. However, sustained treatment with \beta-blockers with improvement in cardiac function and exer cise capacity and reduction in filling pressure and cardiac volumes is usually associat ed with a fall in CNH levels in patients with HF \citep{28} \citep{73} \citep{74}.

As far as hormones more specifically acting on intermediate metabolism are con cerned, insulin (but not hyperglycemia) increased protein synthesis and ANP secre tion and gene expression in cultured rat cardiac myocytes \citep{75}. Moreover, in a model of genetic murine dilated cardiomyopathy, short-term RhGH treatment improved left ven tricular function and significantly reduced elevated mRNA expression of ANP and BNP gene expression in left ventricular tissue \citep{76}.

In conclusion, a huge number of experimental and clinical studies demonstrated that production and secretion of CNH (especially BNP) are not only related to hemo dynamic variations, but also subtly regulated by neuro-hormonal and immunological factors. Therefore, the variation of CNH circulating levels can be considered as a sen sitive index of the perturbation of the homeostatic systems.  3.4 Biological Action of CNH It is important to note that specific receptors for CNH have been found in all mam malian tissues, although at different concentrations, thus suggesting that CNH play an important biological role in several tissues. Most of these biological effects of CNH may be due to an autocrine/paracrine, rather than hormonal, action. In this section, we will discuss in detail only the best-known effects of CNH, and in particular those more strictly related to the endocrine function of the heart.

Cardiac natriuretic hormones have powerful physiological effects on the cardiovas cular system, body fluid, and electrolyte homeostasis \citep{13} \citep{28} \citep{30} \citep{77} \citep{78}. CNH share a direct diuretic, natriuretic and vasodilator effect and an inhibitory action on ventricu lar myocyte contraction \citep{79} as well as remodeling and inflammatory processes of myocardium and smooth muscle cells \citep{80} \citep{81} \citep{82} \citep{83} (Fig. 3.14). Thus, CNH exert a protective effect on endothelial function by decreasing shear stress, modulating coagulation and fib rinolysis pathways, and inhibiting platelet activation (Fig. 3.15). They can also inhibit vascular remodeling process as well as coronary restenosis post-angioplasty \citep{56} \citep{84} \citep{85} \citep{86} \citep{87} \citep{88} \citep{89}.

Furthermore, CNH share an inhibitory action on neuro-hormonal and immuno logical systems, and on some growth factors \citep{13} \citep{28} \citep{30} \citep{77} \citep{78} \citep{90} \citep{91} \citep{92} \citep{93} \citep{94} \citep{95} \citep{96} \citep{97} \citep{98} \citep{99}. In particular, the pivotal role of CNH (especially ANP) in modulating the immune response has been reviewed recently \citep{98}. The first evidence for a role of CNH in the immune system was given by the fact that peptide hormones and their receptors are expressed in various immune organs. Furthermore, several studies indicated that the CNH system in immune cells underlies specific regulatory mechanisms by affecting the innate as well as the adaptive immune response \citep{99}. In particular, ANP supports the first line of defense by increasing phagocytotic activity and production of reactive oxygen species of phago cytes. ANP affects the induced innate immune response by regulating the activation of macrophages at various stages. It also reduces production of pro-inflammatory medi ators by inhibition of iNOS and COX-2 as well as TNF-\alpha synthesis. ANP also affects TNF-\alpha action, i.e. it interferes with the inflammatory effects of TNF-\alpha on the endothe lium. The peptide hormone counteracts TNF-\alpha-induced endothelial permeability and adhesion and attraction of inflammatory cells. Finally, it affects thymopoesis and T cell maturation by acting on dendritic cells and regulates the balance between TH1 and TH2 responses \citep{99}.

The cited effects on the cardiovascular system and body fluid and electrolyte homeosta sis can be explained at least in part by the inhibition of control systems, including the sym pathetic nervous system, the renin-angiotensin-aldosterone system (RAAS), the vaso pressin/antidiuretic hormone system, the endothelin system, cytokines and growth factors \citep{90} \citep{91} \citep{92} \citep{93} \citep{94} \citep{95} \citep{96} \citep{97} \citep{98} \citep{99}.The endocrine action,shared by plasma ANP and BNP,can be enhanced by natriuretic peptides produced locally in target tissues (paracrine action). Indeed, endothelial cells syn thesize CNP, which in turn exerts a paracrine action on vessels \citep{57} \citep{84} \citep{85} \citep{86} \citep{87} \citep{88}. Moreover, renal tubular cells produce urodilatin, another member of the peptide natriuretic family, which has powerful diuretic and natriuretic properties \citep{100}.Genes for natriuretic peptides (includ ing ANP, BNP and CNP) are also expressed in the central nervous system, where they likely act as neurotransmitters and/or neuromodulators \citep{91} \citep{92} \citep{93} \citep{100} \citep{101} \citep{102}. In particular, it was demonstrated that intranasal ANP acts as central nervous inhibitor of the hypothalamus pituitary-adrenal stress system in humans \citep{103}. Finally, co-expression of CNH and of their receptors was observed in rat thymus cells and macrophages \citep{104} \citep{105},suggesting that CNH may have immunomodulatory and anti-inflammatory functions in mammals \citep{106}.

A recent detailed review \citep{107} has highlighted a possible major role for CNH in the development of certain systems, in particular skeleton, brain, and vessels. This review cites recent studies showing severe skeletal defects and impaired recovery after vascu lar and renal injury in CNH transgenic and knockout (KO) mice \citep{108}. In addition, CNH may have a role in the regulation of proliferation, survival, and neurite outgrowth of cultured neuronal and/or glial cells \citep{108}.

Changes in plasma ANP are also correlated with alcohol-associated psychological vari ables \citep{108}. Acute administration of alcohol stimulates the release of ANP independent ly of volume-loading effects. Patients whose ANP levels fell markedly during abstinence also reported more intense and frequent craving as well as more anxiety \citep{108}.  Several reports have shown that CNH stimulate the synthesis and release of testos terone in a dose-dependent manner in isolated and purified normal Leydig cells \citep{109 112}. It has been suggested that this effect on normal Leydig cell steroidogenesis does not involve classical mechanisms of cAMP-mediated regulation of steroidogenic activ ity by gonadotropins \citep{112}. The stimulated levels of testosterone production by ANP, BNP, and gonadotropins were comparable, whereas CNP has been found to be a weak stim ulator of testosterone production in Leydig cells \citep{112}. Moreover, testicular cells contain immunoreactive ANP-like materials and a high density of natriuretic peptide recep tor-A (NRP-A) \citep{112}. These findings suggest that CNH play paracrine and/or autocrine roles in testis and testicular cells. Furthermore, the presence of ANP and its receptors has been reported in ovarian cells, too. Increasing evidence strongly support that CNH are present and probably locally synthesized in ovarian cells of different mammalian species and also play an important physiological role in stimulating estradiol synthe sis and secretion in the female gonad \citep{112} \citep{113} \citep{114} \citep{115}. However, further studies are necessary in order to clarify completely the role played by CNH in the regulation of gonadal func tion and also to assess the inter-relationship between heart endocrine function and gonadal function in humans.

The huge amount of data reported above strongly supports the hypothesis that CNH are active components of the body integrative network that includes nervous, endocrine and immune systems. According to this hypothesis, the heart can no longer be seen as a passive automaton driven by nervous, endocrine or hemodynamic inputs, but as a leading actor on the stage. Thus, CNH, together with other neuro-hormonal factors, regulate cardiovascular hemodynamics and body fluid and electrolyte homeostasis, and probably modulate inflammatory response in some districts, including the car diovascular one. This hypothesis implies that there are two counteracting systems in the body: one has sodium-retaining, vasoconstrictive, thrombophylic, pro-inflamma tory and hypertrophic actions, while the second one promotes natriuresis and vasodi latation, and inhibits thrombosis, inflammation and hypertrophy. CNH are the main effectors of the latter system, and work in concert with NO, some prostaglandins, and other vasodilator peptides (such as bradykinin) \citep{116} \citep{117} \citep{118} \citep{119} \citep{120}. Under physiological condi tions, the effects of these two systems are well balanced via feedback mechanisms, and result in a beat-to-beat regulation of cardiac output and blood pressure in response to endogenous and exogenous stimuli. In patients with HF, the action of the first system is predominant, as a compensatory mechanism, initially, that progressively leads to detrimental effects.

The knowledge so far accumulated regarding CNH suggests that a continuous and intense information exchange flows from the endocrine heart system to nervous and immunological systems and to other organs (including kidney, endocrine glands, liver, adipose tissue, immuno-competent cells) and vice versa (Fig. 3.16). From a pathophys iological point of view, the close link between the CNH system and counter-regulatory systems could explain the increase in circulating levels of CNH in some non-cardiac-relat ed clinical conditions. Increased or decreased BNP levels were frequently reported in acute and chronic respiratory diseases \citep{121} \citep{122} \citep{123} \citep{124} \citep{125} \citep{126} \citep{127} \citep{128} \citep{129}, some endocrine and metabolic diseases \citep{130} \citep{131} \citep{132} \citep{133} \citep{134} \citep{135} \citep{136} \citep{137} \citep{138} \citep{139} \citep{140} \citep{141}, liver cirrhosis \citep{142} \citep{143} \citep{144}, renal failure \citep{100} \citep{144}, septic shock, chronic inflam matory diseases \citep{145} \citep{146} \citep{147} \citep{148} \citep{149}, subarachnoid hemorrhage \citep{150} \citep{151} \citep{152} \citep{153}, and some paraneo plastic syndromes \citep{154} \citep{155} \citep{156}. In addition, any myocardial damage leading to the release of sarcoplasma constituents (including CNH) in extracellular fluid, for instance that due to cardiotoxic agents \citep{157} \citep{158} \citep{159} \citep{160} \citep{161}, cardiac trauma or ischemic necrosis \citep{162} \citep{163}, also causes an increase in plasma concentration of CNH.

Furthermore, the inter-relationships between the CNH system and pro-inflammatory cytokines suggest that cardiac hormones play an important role in mechanisms respon sible for cardiac and vascular adaptation, maladaptation and remodeling in response to various physiological and pathological stimuli \citep{32} \citep{35} \citep{62} \citep{162}.  Elevated BNP levels in extra-cardiac diseases reveal an endocrine heart response to a “cardiovascular stress” (Fig. 3.17). Indeed, recent studies reported that plasma BNP concentration is an independent risk factor for mortality (cardiac and/or total) in pul monary embolism \citep{121} \citep{123} \citep{124} and hypertension \citep{127}, renal failure \citep{28} \citep{100} \citep{144}, sep tic shock \citep{145}, amyloidosis \citep{149}, and diabetes mellitus \citep{141} (see Chapter 6 for more details). According to this hypothesis, a BNP assay should be considered as a marker of cardiac stress (Fig. 3.17).

In conclusion, CNH share a powerful action on the cardiovascular system, including diuretic, natriuretic and vasodilator effects and an inhibitory action on ventricular myocyte contraction, as well as on remodeling and inflammatory processes of myocardi um and smooth muscle cells. Furthermore, CNH exert a protective effect on endothelial function by decreasing shear stress, modulating coagulation and fibrinolysis pathways, and inhibiting platelet activation. They can also inhibit the vascular remodeling process as well as coronary restenosis post-angioplasty. These effects can be explained, at least in part, by the inhibition of control systems, including the sympathetic nervous system, the RAAS, the vasopressin/antidiuretic hormone system, the endothelin system, cytokines and growth factors. Finally, the endocrine action of ANP and BNP is potentiated at the periphery (target tissues) by the paracrine action of other members of the peptide natri uretic family, such as CNP (in the vascular tissue) and urodilatin (in renal tissue).  Finally, some experimental studies performed in KO mice suggest a distinct patho physiological role for BNP in respect to ANP \citep{18}. While BNP KO mice are no different from control mice with regard to blood pressure, urine volume, and urinary electrolyte excretion, they have more extensive ventricular fibrosis, accompanied by increased transforming growth factor-b3 (TGF-b3) and collagen mRNA \citep{18}. These data suggest that BNP may function more as an autocrine/paracrine inhibitor of cell growth in the heart; while ANP may be considered as a traditional circulating hormone with pro nounced diuretic, natriuretic, and antihypertensive effects.

\subsection{Natriuretic Peptide Receptors and Intracellular Second Messenger Signaling
}
Cardiac natriuretic hormones share their biological action by means of specific recep tors (NPR), which are present within the cell membranes of target tissues. Three different subtypes of NPRs have so far been identified in mammalian tissues \citep{112} \citep{164} \citep{165}.

\subsubsection{Genes Encoding for NPRs
}
NPR1 is the gene coding the NPR-A receptor (natriuretic peptide receptor A/guany late cyclase A) and it is located on 1q21-q22 spanning 15,534 bp with 22 exons. The rel ative mRNA of 3,805 bp leads to a protein of 1,061 amino acids.  The NPR-B receptor (natriuretic peptide receptor B/guanylate cyclase B) is codified by the gene NPR2. This gene of 17,303 bp is on chromosome 9 (9p21-p12) and it is organized in 22 exons, which can give two types of mRNA. NPR2 Ia is an mRNA of 3,482 bp that has a 71 nucleotide insertion relative to isoform b, which results in a dif ferent, and shorter (995 aa), carboxy-terminus that may disrupt the guanylyl cyclase activity. NPR2 Ib (3,411 bp, 1,047 aa) does not include the alternate exon found in iso form a, and thus isoform b contains a longer carboxy-terminus. The natriuretic peptide receptor C gene, also named NPR3, is on 5p14-p13 and spans 74,698 bp (8 exons), giv ing an mRNA of 1,753 bp that is translated into a protein of 540 amino acids.

\subsubsection{Biological Function of NPRs
}
NPR-A and NPR-B are generally considered to mediate all known biological actions throughout the guanylate cyclase (GC) intracellular domain, while the third member of the natriuretic peptide receptor family, the NPR-C receptor, does not have a GC domain (Figs. 3.18, 3.19 and 3.20).  The GC receptors for ANP/BNP (NPR-GC-A) and CNP (NPR-GC-B) belong to a fam ily of seven isoforms of transmembrane enzymes (from GC-A to GC-G), which all con vert guanosine triphosphate into the second messenger cyclic 3’,5’-guanosine monophos phate (cGMP) \citep{164}.

Although partly homologous to soluble GC, the receptor for NO, the membrane GCs share a different and unique topology. The single transmembrane span domain divides the protein structure into an extracellular ligand binding domain and an intracellular region consisting of a protein kinase-homology domain, an amphipathic helical or hing region, and a cyclase-homology domain \citep{165} (Figs. 3.18, 3.19 and 3.20). The cyclase homology domain represents the catalytic cGMP synthesizing domain. The function of the intracellular region consisting of a protein kinase-homology domain is incom pletely understood. Although it probably binds ATP and contains many residues con served in the catalytic domain of protein kinases, kinase activity has not been detect ed \citep{165}. It represses the enzyme activity of the catalytic cGMP-synthesizing domain and at the same time is necessary for its ligand-dependent activation \citep{154}. The coiled-coil hing region is involved in receptor dimerization, which is also essential for the activa tion of the catalytic domain \citep{165}.

The cGMP produced modulates the activity of specific downstream regulatory pro teins, such as cGMP-regulated phosphodiesterases, ion channels and cGMP-dependent protein kinases type I (PKG I) and type II (PKG II) (Fig. 3.20). These proteins should be considered to be third messengers, which are differentially expressed in different cell types, ultimately modifying cellular functions \citep{166} \citep{167}. This specific action of CNH on target tissues depends essentially on two different mechanisms.

The physiological expression of NPR-A and NPR-B differs quite significantly in human tissues (Fig. 3.21). NPR-A is found in abundance in larger, conduit blood vessels, whereas the NPR-B is found predominantly in the central nervous system \citep{168}. Both receptors have been localized in adrenal glands and kidney \citep{168}. On the other hand, several studies indicate that phosphorylation of the kinase homology domain is a crit ical event in the regulation of NPRs \citep{169} \citep{170} \citep{171}.

The affinity for ANP, BNP and CNP also varies greatly among the different NPRs.  ANP shows a greater affinity for NPR-A and NPR-C, and CNP for NPR-B, while BNP shows a lower affinity for all NPRs compared to the other two peptides (Fig. 3.21).  Activation of the GC-linked NPRs is incompletely understood \citep{172}. NPR-A and NPR B are homo-oligomers in the absence and presence of their respective ligands, indicat ing that receptor activation does not simply result from ligand-dependent dimerization \citep{173}. However, ANP binding does cause a conformational change of each monomer closer together \citep{172} \citep{173} \citep{174} \citep{175} \citep{176}. The stoichiometry of the ligand-receptor complex is 1:2 \citep{177}.  Initial in vitro data suggested that direct phosphorylation of NPR-A by protein kinase C mediated its “desensitization” (i.e., the process by which an activated receptor is turned off) \citep{178}. However, subsequent studies conducted in live cells indicated that desensiti zation in response to prolonged natriuretic peptide exposure or activators of protein kinase C results in a net loss of phosphate from NPR-A and NPR-B \citep{171} \citep{179} \citep{180} \citep{181} \citep{182}.

Although ligand-dependent internalization and degradation of NPR-A has been intense ly studied by several groups for many years, a consensus understanding of the importance of this process in the regulation of NPRs has not emerged \citep{182}. Early studies conducted on PC-12 pheochromocytoma cells suggested that both NPR-A and NPR-C internalize ANP and that both receptors are recycled back to the cell surface \citep{184}. Other studies, using Leydig, Cos, and 293 cell lines, have reported that ANP binding to NPR-A stimulates its internalization, which results in the majority of the receptors being degraded with a smaller portion being recycled to the plasma membrane \citep{184} \citep{185} \citep{186} \citep{187}. In contrast, other stud ies performed in cultured glomerular mesangial and renomedullary interstitial cells from the rat or Chinese hamster ovary cells reported that NPR-A is a constitutively membrane resident protein that neither undergoes endocytosis nor mediates lysosomal hydrolysis of ANP \citep{188} \citep{189}. A more recent study using 293T cells suggested that NPR-A and NPR B are neither internalized nor degraded in response to receptor occupation \citep{173}. Fur thermore, this study did not support the hypothesis that down-regulation is responsible for NPR desensitization observed in response to various physiological or pathological stim uli \citep{182}.Further studies are necessary to clarify whether or not ANP binding to NPR-A stim ulates its internalization, and whether this process is tissue- and/or species-specific.

It is generally thought that the NPR-C is not linked to GC and so serves as a clearance receptor \citep{28} \citep{77} \citep{78}.NPR-C is present in higher concentration than NPR-A or NPR-B in sev eral tissues (especially vascular tissue),and it is known constitutively to internalize CNH \citep{172} (Fig.3.22).However,recent studies have found that CNH interact with NPR-C to suppress the cAMP concentration by inhibition of adenylyl cyclase \citep{190} \citep{191}. Specific binding to NPR-C increases inositol triphosphate and diacylglycerol concentrations by activating phospholi pase C activity or inhibits DNA synthesis stimulated by endothelin, platelet-derived growth factor and phorbol ester by inhibiting MAPK activity,as recently reviewed \citep{190}.The NPR-C mediated inhibition of adenylyl cyclase is mediated through Gi (inhibitory guanine nucleotide regulatory) proteins.According to this hypothesis,NPR-C,which is present in large amounts, especially on the endothelial cell wall,may mediate some paracrine effects of CNP on vascu lar tissue \citep{168} \citep{190}.However,further studies are necessary to elucidate the possible role of NPR C receptors as modulators of CNH action and/or degradation in peripheral tissues.

\subsection{Metabolic Pathways and Circulating Levels of CNH
}
Atrial natriuretic peptide and BNP are secreted directly from the heart. In the circula tion, CNHs are metabolized via two principal mechanisms: degradation by a mem brane-bound endopeptidase (NEP 24.11) and receptor-mediated cellular uptake via NPR-C \citep{14} (Fig. 3.22). Some biological characteristics of ANP, BNP and CNP (as well as of their precursors) are summarized in Table 3.3.

\subsubsection{ANP Metabolism
}
Atrial natriuretic peptides are a family of peptides derived from a common precursor, called preproANP, which in humans contains 151 amino acids and has a signal peptide sequence at its amino-terminal end (Fig. 3.11). The pro-hormone is stored in secretion granules of cardiomyocytes as a 126-amino-acid peptide, proANP 1-126 , which is pro duced by cleavage of the signal peptide. When appropriate signals for hormone release are given, proANP 1-126 is further split by some proteases (especially the serine protease corin) \citep{192} into N-terminal fragment NT-proANP and the COOH-terminal peptide ANP, which is generally considered to be the biologically active hormone, because it contains the cysteine ring (Figs. 3.1 and 3.11).

Studies from the group of Vesely et al. suggested that the NT-proANP can be metab olized in vivo in three peptide hormones with blood pressure-lowering, natriuretic, diuretic and/or kaliuretic properties \citep{100}. These peptide hormones, numbered by their amino acid sequences, beginning at the N-terminal end of the proANP pro-hormone, include: 1) the peptide proANP 1-30 , also called long-acting natriuretic peptide (LANP); 2) the peptide proANP 31-67 with vessel dilator properties; 3) the peptide proANP 79-98 with kaliuretic properties \citep{98}. However, these three peptides do not bind to the same NPRs of CNHs, because they do not have the cysteine ring. Further studies are neces sary to confirm and elucidate the biological action of these putative peptide hormones, as well as their in vivo metabolism.

There is some evidence that ANP is secreted according to a pulsatile pattern in humans \citep{193} \citep{194} \citep{195} \citep{196} \citep{197}. Upon secretion, ANP is rapidly distributed and degraded (the meta bolic clearance rate of ANP is on average about 2,000 ml/min in healthy subjects) with a plasma half-life of about 4-6 minutes in healthy adult subjects. In humans, about 50\% of the ANP secreted into the right atrium is extracted by the peripheral tissues during the first pass throughout the body \citep{198} \citep{199} \citep{200} \citep{201}. Furthermore, circulating ANP represents only a small fraction of the total body pool (no more than 1/15) in normal subjects and plasma ANP concentration shows rapid and wide fluctuations in healthy subjects, even at rest in the recumbent position \citep{198} \citep{199} \citep{200} \citep{201}. The turnover data suggest that circulating levels of ANP may not represent a close estimate of their disposal, and therefore of the activity of the CNH system, as implicitly accepted in physiological or clinical studies in which only the plasma concentration of the hormone is measured, without an estima tion of turnover rate. However, it was demonstrated that ANP clearance mechanisms are constant in the presence of rapid and large changes in endogenous ANP plasma levels induced by atrial and/or ventricular pacing, thus indicating that, at least for studies lasting only a few hours, changes in ANP circulating levels may provide a reliable esti mate of production rate variations \citep{201}.

\subsubsection{BNP Metabolism
}
The biological action, metabolic pathways, and turnover parameters of BNP are not as well known as those of ANP \citep{14}. However, it is commonly believed that the BNP turnover is less rapid than that of ANP with a plasma half-life of about 13-20 min utes; indeed, circulating levels of BNP are more stable than those of ANP in adult healthy subjects (Fig. 3.23). Bentzen et al. \citep{197} analyzed the secretion pattern of ANP and BNP in 12 patients with chronic HF and in 12 healthy adult subjects. ANP and BNP in plasma were determined by radioimmunoassay (RIA) at 2 min intervals during a 2-h period and were subsequently analyzed for pulsatile behavior using the method of Fourier transformation. All patients and healthy subjects had significant rhythmic oscillations in plasma ANP levels, and 11 patients with HF and 10 healthy subjects had significant rhythmic oscillations in plasma BNP levels \citep{197}. The ampli tude of the main frequency was considerably higher in patients than in healthy sub jects, but the main frequency did not differ significantly between patients and healthy subjects for either ANP or BNP. Patients with HF demonstrated pulsatile secretion of ANP and BNP with a much higher absolute amplitude, but with the same main fre quency as healthy subjects \citep{197}. Finally, rhythmic oscillations in plasma ANP lev els of healthy subjects showed significantly higher mean amplitude, but not fre quency, than those of BNP \citep{197}.

A very small amount of immunoreactive BNP has been found in urine \citep{202} \citep{203}, but the precise mechanism of renal excretion has not yet been fully clarified. In contrast to BNP, the biologically active peptide, other proBNP-derived inactive fragments also cir culate in plasma. These fragments are commonly referred to as “N-terminal proBNP” (NT-proBNP), but the molecular heterogeneity also includes the intact precursor, par ticularly in patients with HF \citep{14} \citep{204}. Cardiac secretion of proBNP and its N-termi nal fragments has been demonstrated by blood sampling from the coronary sinus \citep{205}.  Some data suggest that the major part of proBNP produced in myocardiocytes is appar ently processed prior to release \citep{14}; however, intact proBNP peptide was also found in plasma of patients with HF as well as healthy adult subjects \citep{14} \citep{205} \citep{206}.

A recent study, employing a new method for the total and equimolar assay of all proB NP-related peptides (i.e., intact proBNP precursor plus NT-proBNP concentrations), found comparable peripheral concentrations of BNP (measured by immunoradiomet ric assay) and proBNP-related peptides in patients with HF \citep{206}. Moreover, the BNP concentration (median 125 pmol/l) was higher than that of total proBNP (103 pmol/l) in the coronary sinus, suggesting that the cardiac secretion of these two peptides could be different \citep{206}.Alternatively, this finding could also reflect some difference in peripheral elimination of peptides because total proBNP concentration is significantly higher in the pulmonary artery than the aortic root in patients with right ventricular failure \citep{207}.

While NEP enzymes are mainly involved in natriuretic peptide inactivation in vivo, the degradation of BNP seen in vitro is most likely due to other enzymes, such as peptyl arginine aldehyde proteases, kallikrein, and serine proteases \citep{15}. However, the role of these enzymes in the degradation of BNP in vivo is unclear.

A recent study reported that both the BNP and total proBNP concentrations were increased more than 2-fold in the coronary sinus compared to the inferior caval vein (BNP-32: median 125 pmol/l, range 21-993 vs median 52 pmol/l, range 7-705; proBNP: median 103 pmol/l, range 16-691 vs 47 pmol/l, 8-500) \citep{206}. These findings are in accordance with previous studies suggesting that the cardiac gradient for BNP secre tion (as estimated by the difference between BNP concentration in coronary sinus and inferior caval vein) ranges from 1.6-fold to 2.9-fold \citep{204} \citep{208} \citep{209} \citep{210}. Taking these studies as a whole, ANP and BNP share a similar peripheral extraction value (of about 30-50\%). Further studies are necessary to elucidate the metabolism of BNP and in particular the predominant form of the circulating BNP-related peptides.

\subsection{CNH Genes and Cardiovascular Diseases
}
Since CNH have a potent diuretic antihypertensive action, and the impaired action of the peptides may cause hypertension, their genes may be candidates for cardiovascu lar disease, especially arterial hypertension. Furthermore, transgenic animals (espe cially mice), overexpressing CNH or knockout for ANP/BNP genes or their specific receptors, have been used to evaluate the pathophysiological role of the CNH system in cardiovascular diseases \citep{251}.

In transgenic mice with overexpression of ANP and BNP in liver, plasma ANP and BNP levels are from 10- to 100-fold higher than in control mice, with a blood pressure of 20-25 mmHg lower. These mice also have lighter hearts, but with the same cardiac out put and rate, than controls \citep{251} \citep{253} \citep{254} \citep{255}. The BNP-overexpressing mice show the same hemodynamic changes; on the other hand, ANP KO mice develop NaCl-sensitive hyper tension \citep{251}. Transgenic mice overexpressing the NPRA gene have also been created; these animals have a lower blood pressure than wild-type mice \citep{251}. The correspon ding KO mice show an increase in blood pressure compared with controls (on average 10 mmHg in heterozygous and 30 mmHg in homozygous animals), which is not affect ed by NaCl intake \citep{254} \citep{255}. These data suggest a different pathophysiological mech anism for hypertension between KO mice for the ANP gene and its specific receptor; this difference does not yet have an explanation \citep{251}. NPRC heterozygous KO mice do not show blood pressure variation, whereas homozygous mice show on average a decrease in blood pressure of about 8 mmHg \citep{251}.

The function of natriuretic peptides was also studied after induction of myocardial infarction in KO mice lacking the natriuretic peptide receptor guanylyl cyclase-A, the receptor for ANP and BNP \citep{89}. KO and wild-type mice were subjected to left coronary artery ligation and then followed-up for 4 weeks. KO mice showed significantly higher mortality because of a higher incidence of acute HF, which was associated with dimin ished water and sodium excretion and with higher cardiac levels of mRNAs encoding ANP, BNP, TGF-b1, and type I collagen. By 4 weeks after infarction, left ventricular remodeling, including myocardial hypertrophy and fibrosis, and impairment of left ventricular systolic function were significantly more severe in KO than wild-type mice \citep{89}. These data confirm that the CNH system has powerful anti-remodeling properties on ventricular cardiomyocytes.

In recent years, molecular genetic techniques have been introduced in etiological stud ies of polygenetic diseases, in linkage studies, in sib-pair linkage studies of various candi date genes, and in related studies \citep{251} \citep{252}. The association between some abnormalities in genes, coding for the CNH and their receptors, and some cardiovascular (in particular hypertension) and metabolic (such as diabetes mellitus) diseases has been tested in a large number of clinical studies (see also Chapter 6 for more details). To date, the results obtained are conflicting and seem to depend strictly on the ethnic population of the study.  The restriction fragment length polymorphism for the enzyme HpaII, located in intron 2 of NPPA (polymorphism also called Sma I), was reported to be more common in hypertensive African-Americans than in normotensive black controls \citep{257}; these data were then confirmed in two \citep{258} \citep{259}, but not a third \citep{260}, Caucasian popula tions. Furthermore, another study found that the HpaII polymorphism was not asso ciated with hypertension in the Chinese population of Hong Kong \citep{261}.

Regarding other NPPA polymorphisms, Japanese studies reported that both G1837A and T2238C polymorphisms are associated with essential hypertension \citep{262}, while only a marginally significant association was found with an ANP polymorphism locat ed in the 5’-untranslated region (C664G) \citep{263}.

Several allelic variants have also been described for genes coding for CNH recep tors (see the recent revew by Nakayama \citep{251} for a more detailed discussion of this topic). The clearance receptor for natriuretic peptides (NPR-C) is highly expressed in adipose tissue, and its bi-allelic (A/C) polymorphism was detected at position -55 in the conserved promoter element named P1. This variant of the NPR-C P1 promoter is associated with lower ANP levels and higher systolic blood pressure and mean blood pres sure in obese hypertensives: the C(-55) variant, in the presence of increased adiposity, might reduce plasma ANP through increased NPR-C-mediated ANP clearance, con tributing to higher blood pressure \citep{264}.

In the Japanese population an insertion/deletion (GCTGAGCC) polymorphism has been identified in the 5’-flanking region of the NPRA gene that is associated with essen tial hypertension and left ventricular hypertrophy \citep{265}. Another insertion/deletion polymorphism is on the 3’-untranslated region of the NPRA gene, on exon 22, and it seems to be associated with familial hypertension \citep{266}. However, these data should be confirmed in larger studies, including other ethnic populations.

3.9 An Integrated Neuro-Hormonal System Regulates Vascular Function

Endothelial cells release an array of vasoactive mediators that alter the tone and growth of the underlying smooth muscle and regulate the reactivity of circulating white blood cells, erythrocytes and platelets. These endogenous factors are usually called endothe lium-derived vasorelaxant mediators \citep{267}. Moreover, it appears that alterations in the capacity of the endothelium to release some mediators in response to pathophysiolog ical stimuli (the so-called endothelium dysfunction) are a major precipitating factor in many cardiovascular diseases. Perhaps the most important of these paracrine medi ators are prostacyclin (PGI 2 ) and nitric oxide (NO). More recently, a third endotheli um-derived vasorelaxant mediator has been described \citep{267}. This is termed endothe lium-derived hyperpolarizing factor (EDHF) because it elicits a characteristic smooth muscle hyperpolarization and relaxation. Much attention has focused on identifying EDHF(s), with diverse candidates, including cytochrome P450 metabolites, KC ions, anandamide and hydrogen peroxide \citep{267}. However, the role of each of these as EDHF remains unsubstantiated.

There is now compelling evidence that CNH (and especially CNP) act as EDHFs in some vascular beds \citep{267}  \citep{269} \citep{270} \citep{271}. Indeed, numerous studies have demonstrated that ANP, BNP and CNP bind to NPR-A and NPR-B receptors on vascular smooth muscle cells (either freshly isolated or in culture), stimulate cGMP accumulation, and cause a dose dependent vasodilation  \citep{269} \citep{270} \citep{271}. This increase in cGMP causes vasodilatation by reducing intracellular calcium levels, as occur when cGMP accumulation is stimulated by NO and its analogs .

It is theoretically conceivable that ANP and BNP act like hormones in vascular tis sue by reaching the smooth muscle cells from the circulation after secretion by the heart, while CNP shows a paracrine action, being secreted by endothelial cells \citep{57} \citep{84} \citep{87} \citep{88} (Fig. 3.15). However, Casco et al. \citep{272} demonstrated the existence of a complete CNH system (including the production and secretion of ANP, BNP and CNP) in ather osclerotic human coronary vessels by means of in situ hybridization and immunocy tochemistry methods. In particular, the expression of mRNAs of ANP, BNP and CNP, measured by RT PCR, tended to be increased in macroscopically diseased arteries com pared to normal vessels, although only the values for BNP expression were significant ly different \citep{272}. This study suggests that the CNH system is involved in the pathobi ology of intimal plaque formation as well as in vascular remodeling in humans.  Some studies indicated that there are complex interactions even among CNH them selves. Nazario et al. \citep{273} reported that ANP and BNP can stimulate CNP production through a guanylate cyclase receptor on endothelial cells. As a result, vasodilatory, and anti-mitogenic effects of ANP and BNP in the vasculature could occur in part through CNP production and subsequent action if these interactions occur in vivo. In other words, ANP/BNP and CNP paracrine system should share a synergic action on vascu lar tissues.

Several studies have demonstrated complex interactions betwen CNH and the other endothelium-derived vasorelaxant mediators \citep{267}. Indeed, evidence from cellular, ani mal, and human studies suggests that all CNH are able to stimulate NO production by endothelial NO synthase (eNOS); this effect is probably mediated by clearance recep tor NPR-C \citep{270}. Stimulation of this NPR-C receptor results in decreased cAMP levels by adenyl cyclase inhibition through an inhibitory guanine nucleotide-regulating pro tein \citep{270}. Furthermore, ANP treatment increases renal and cardiac NO synthesis in rats \citep{274}. On the other hand, NO, released from endothelial cells, negatively modulates ANP secretion from atrial myocytes, induced by mechanical stretch in perfused rat heart preparation \citep{275}. Furthermore, ANP expression is markedly upregulated in eNOS -/- mice, and exogenous ANP restores ventricular relaxation in wild-type mice treated with NOS inhibitors \citep{276}. These data suggest that the CNH and NO systems are linked by a negative feedback mechanism. Finally, CNH (and especially CNP) mimic many of the anti-atherogenic actions of PGI 2 and NO \citep{267}. This gives rise to the pos sibility that CNP might compensate for the loss of these mediators in cardiovascular pathologies to restore the vasodilator capacity of the endothelium, in addition to its anti-adhesive and anti-aggregatory influences.

CNH also strongly interact with the effectors of counter-regulatory systems at the vas cular tissue level \citep{13} \citep{28} \citep{30} \citep{77} \citep{78} \citep{90} \citep{91} \citep{92} \citep{93} \citep{94} \citep{95} \citep{96} \citep{97} \citep{98} \citep{99}. In particular, interactions between CNH and ET-1 also appear to be important physiologically; indeed, the vascular effects of CNH are directly opposite to those of ET-1 \citep{267} \citep{269}; in particular, ET-1-induced vasocon striction and myocyte hypertrophy is inhibited by CNH. While CNP has little natri uretic and diuretic action compared to ANP or BNP, it is capable of modulating the vascular effects of the local RAAS by opposing potent vasoconstriction to angiotensin II \citep{269}. CNP not only functionally antagonizes ET-1 and angiotensin II, but it also directly modulates ET-1 \citep{277} and angiotensin II \citep{278} synthesis. On the other hand, ET 1 induces an increase in the number of endothelial cells that secrete CNP \citep{279}. There fore, the parallel production and activity of vasodilator CNP and vasoconstrictors such as ET-1 and angiotensin II allows for tight local regulation of these vasoactive peptides and thus blood flow \citep{267} \citep{269} \citep{279}.

Furthermore, the inter-relationships between the CNH system and pro-inflammatory cytokines suggest that cardiac hormones play an important role in mechanisms respon sible for cardiac and vascular adaptation, maladaptation and remodeling in response to various physiological and pathological stimuli \citep{32} \citep{35} \citep{62} \citep{162}. The identification of CNP as an EDHF, combined with its expression in endothelial cells, indicates that CNP is suited to modulate the activity of circulating cells, particularly leukocytes and platelets.  Moreover, inflammatory stimuli such as IL-1b, TNF and lipopolysaccharide \citep{280} stim ulate the release of CNP from isolated endothelial cells. As a result, modulation of the biological activity of CNP is likely to have a profound influence on the development of an inflammatory response. Certainly, an anti-atherogenic activity of CNP fits with the cytoprotective, anti-inflammatory actions of NO and PGI 2 , the other major endotheli um-derived vasorelaxants \citep{267} \citep{269} \citep{270} \citep{271} \citep{280}.

From a clinical point of view, it is important to note that exogenous application of CNP in situations where endothelial NO production is compromised might be therapeutic in disorders that are associated with endothelial dysfunction. For example, overex pression of CNP by adenoviral-gene delivery in veins dramatically reduces the lumi nal narrowing (neointimal hyperplasia) that develops when it is grafted to the carotid artery, thereby retaining patency of the graft \citep{281}. CNH, including CNP, also suppress the production of pro-inflammatory cyclooxygenase 2 metabolites in isolated cells \citep{106} \citep{282}. Other studies demonstrated a direct effect of CNP on immune-cell recruitment in vivo \citep{267} \citep{271}. Therefore, like NO, endothelial CNP (like ANP and BNP) exerts a pro tective anti-inflammatory effect \citep{104} \citep{105} \citep{106} \citep{267} \citep{271} \citep{283}. This inhibitory effect of CNH on leukocytes indicates that these peptides modulate the expression of adhesion mol ecules on either the endothelium or leukocytes.

Several data support the thesis that CNH (and especially CNP) are important, endoge nous, anti-atherogenic mediators. CNP is a potent inhibitor of vascular smooth muscle migration and proliferation that is stimulated by oxidized LDL \citep{277}. CNP also inhibits the proliferation of vascular smooth muscle \citep{284}, and enhances endothelial cell regen eration in vitro and in vivo \citep{281}. The observation that CNP alters leukocyte-endothe lial interactions indicates that it might also affect platelet function. In accordance with this, thrombus formation is suppressed significantly in the presence of CNP \citep{280}, which indicates that inhibition of coagulation might contribute to the vasoprotective proper ties of this peptide. Observations that CNP blocks platelet aggregation, induced by throm bin, confirm that endothelium-derived CNP also exerts an anti-thrombotic effect \citep{267}.

All the above-mentioned studies demonstrate that CNH (and especially CNP) exert a protective effect on endothelial function by decreasing shear stress, modulating coagulation and fibrinolysis pathways, and inhibiting platelet activation (Fig. 3.15). They can also inhibit the vascular remodeling process as well as coronary restenosis post-angioplasty \citep{56} \citep{84} \citep{85} \citep{86} \citep{87} \citep{88} \citep{89} \citep{267} \citep{281} \citep{283}. These vasoprotective actions should be considered as a result of complex inter-relationships between the CNH system and both the synergic (including NO, PGI 2 , and other endothelium-derived vasoactive mediators) and the counter-regulatory systems (including endothelins, RAAS, cytokines, and growth factors).

3.10 Summary and Conclusion

Natriuretic peptides (including ANP, BNP, CNP, DNP and urodilatin) constitute a fam ily of peptide hormones and neurotransmitters, sharing a similar peptide chain, char acterized by a cysteine bridge (Fig. 3.1). The physiological relevance of these peptides is well demonstrated by their presence since the first dawning of life, from unicellular to pluricellular organisms, including plants and all animals. Furthermore, their genes have been repeatedly doubled during evolution, starting from an ancestral gene, thus sug gesting that these peptides are indispensable for life (Fig. 3.6).

CNH have powerful physiological effects on the cardiovascular system, body fluid, and electrolyte homeostasis. These effects can be explained at least in part by the inhibition of counter-regulatory systems, including the sympathetic nervous system, RAAS, the vaso pressin/antidiuretic hormone system, the endothelin system, cytokines and growth factors.  The endocrine action shared by plasma ANP and BNP can be enhanced by natri uretic peptides produced locally in target tissues (paracrine action). Indeed, endothe lial cells synthesize CNP, which exerts a paracrine action on vessels. Moreover, renal tubular cells produce urodilatin, another member of the peptide natriuretic family, which shows powerful diuretic and natriuretic properties. Genes for natriuretic pep tides (including ANP, BNP and CNP) are also expressed in the central nervous system, where they likely act as neurotransmitters and/or neuromodulators. Finally, co-expres sion of CNH and their receptors was observed in immunocompetent cells, suggesting that CNH may have immunomodulatory and anti-inflammatory functions in mam mals. Furthermore, CNH are expressed in almost all the body tissues as well as their specific receptors, including organs and tissues not discussed in this chapter, such as gut \citep{285}, skeletal \citep{106} and ocular \citep{286} tissues. In all tissues, CNH could also act as a local mediator or paracrine effector of tissue-specific functions.

These data, taken as a whole, strongly suggest that natriuretic peptides constitute a family sharing endocrine, paracrine and autocrine actions and neurotransmitter and immunomodulator functions. Therefore, it can be hypothesized that the CNH system is closely related to the other regulatory systems (nervous, endocrine and immuno logical) in a biological hierarchical network (Fig. 3.16) \citep{287} \citep{288}.

\section{Cardiac Natriuretic Hormones as Markers of Cardiovascular Disease: Methodological Aspects}

\subsection{ General Considerations}

Cardiac natriuretic hormones (CNH) constitute a complex family of related peptides with
similar peptide chains as well as degradation pathways (see Chap. 3 for more details).
CNH derive from common precursors, pre-pro-hormones (i.e., preproANP and preproBNP). Pro-hormone peptides are further split into an inactive longer NT-proANP or NTproBNP and the biologically active hormones, ANP or BNP, which are secreted in the
blood in equimolar amounts. However,ANP and BNP have a shorter plasma half-life and
consequently lower plasma concentrations compared to NT-proANP and NT-proBNP
(Table 4.1). For these reasons, setting up an immunoassay for N-terminal peptide fragments of proANP and proBNP should be easier than that for ANP and BNP, because
the requested analytical sensitivity is not too low \citep{bib21}. However, immunoassays for NTproANP and NT-proBNP may be affected by problems related to the different assay
specificities: as shown in Table 4.1, different results are produced by different methods
with a large bias \citep{bib21} \citep{bib22} \citep{bib23}. The different analytical performances could have some relevance in the diagnostic accuracy of different assay methods in discriminating between
subjects with or without cardiac disease \citep{bib21} \citep{bib22} \citep{bib23} \citep{bib24} \citep{bib25} \citep{bib26} \citep{bib27} \citep{bib28} \citep{bib29} \citep{bib210} \citep{bib211}.

Table 4.2 summarizes the respective advantages of the assay of biologically active
peptide hormones (ANP and BNP) compared to those of the assay of NT-proANP and
NT-proBNP. The assay of the inactive NT-propeptides better fits the definition of a disease biomarker than that measuring circulating concentrations of ANP or BNP, which,
on the other hand, may be considered a more reliable index of the activation (hormonal) status of the CNH system.

Taking into account the biochemical and physiological characteristics of different
peptides, it is theoretically conceivable that ANP is a better marker of acute overload
and/or rapid cardiovascular hemodynamic changes than BNP and NT-proANP or NTproBNP. It is well known that the circulating concentrations of ANP are more affected
by body position and more decreased by a hemodialysis session in patients with chronic renal failure than those of BNP, while plasma concentrations of NT-proANP or NTproBNP are not significantly changed \citep{bib21} \citep{bib24}. Furthermore, ANP increases more than
NT-proANP during rapid ventricular pacing \citep{bib212}.

The first methods commercially available for ANP and BNP determination, set up
before 1990, were competitive immunoassays (i.e., radioimmunoassay [RIA] and enzyme
immunoassay [EIA] methods). These methods usually required a preliminary chromatographic step because of their poor sensitivity and specificity \citep{bib21}; this purification
step markedly decreased assay precision and practicability. A second generation of
immunoassays for ANP- and BNP-related peptides became commercially available during the 1990s; these methods were non-competitive immunoassays, e.g. immunoradiometric assays (IRMAs) \citep{bib213} \citep{bib214}. Compared to competitive methods, IRMAs for CNH
determination showed better sensitivity, specificity, and did not require a preliminary chromatographic step. However, the long turnaround time did not allow their use in the clinical setting, especially in emergency situations \citep{bib21} \citep{bib22} \citep{bib23} \citep{bib24}. More recently, a new generation of
immunoassays for CNH became commercially available, including some point-of-care
testing (POCT) methods \citep{bib215}. These methods are non-competitive sandwich immunoassays, which use non-radioactive materials as labels for the antigen/antibody reaction \citep{bib23} \citep{bib24}. The availability of these methods has allowed a wider diffusion of CNH assays (mainly BNP and NT-proBNP assays) in clinical practice, including emergencies.

In summary, several methods for CNH determination have been proposed measuring similar or identical peptides, showing, however, different analytical performance, reference values, clinical results, and, possibly, diagnostic accuracy \citep{bib23} \citep{bib24}. In addition, there
is no general agreement about the CNH terminology used by different researchers and
manufacturers; this may increase confusion and cause misleading interpretation. A
standardization of terminology and methods is, therefore, required in order to evaluate and compare the diagnostic accuracy of the different CNH assays \citep{bib24}. In this chapter, the most important analytical characteristics and performances of CNH immunoassays are described.

\subsection{ Determination of ANP and NT-proANP}

\subsubsection{ Competitive Immunoassay Methods for ANP}

Before 2000, tissue or circulating concentrations of ANP were usually measured by competitive immunoassay methods using radiolabeled tracers (RIA), although enzyme
immunoassay methods were also set up \citep{bib216} \citep{bib217} \citep{bib218} \citep{bib219} \citep{bib220} \citep{bib221} \citep{bib222} \citep{bib223} \citep{bib224} \citep{bib225} \citep{bib226} \citep{bib227} \citep{bib228} \citep{bib229} \citep{bib230} \citep{bib231} \citep{bib232}. As ANP determination by competitive
immunoassay methods can be affected by many analytical problems, several analytical
options were proposed \citep{bib21} \citep{bib22} \citep{bib229} \citep{bib230}. First, a suitable ANP assay should be targeted only
at the biologically active peptide. For this reason, the employed antibodies should be highly specific for the “biologically active” part of the peptide, i.e., the amino acid sequence
containing the ring structure formed by the disulfide bridge. As an example, cross-reactivities of four different antisera used in commercial ANP assays are reported in Table
\subsection{. These data indicate that all assays are more specific for the intact ANP molecule}
than for the other tested ANP fragments, in which the disulfide bridge is disrupted.
However, large differences in specificities of the antisera are also present. The main
consequence of this finding is that large differences in ANP values should actually be
expected when measured by different assays (Table 4.4) \citep{bib21} \citep{bib22} \citep{bib229} \citep{bib230}.

Theoretically, the major drawback results from the presence of several ANP-related
peptides in plasma samples (or tissues; i.e., all the endogenous precursors or metabolites of the hormone and other cardiac or non-cardiac peptide hormones structurally
related to ANP), which can interfere with a competitive immunoassay. Moreover, some
authors have suggested that platelets can interfere with a direct RIA by producing an
enhanced plasma peptide concentration due to the presence of high-affinity receptors
for cardiac natriuretic peptides on platelets \citep{bib224}. As suggested, the “classical” centrifugation (1000 x g for 10 min at 4°C) of blood samples could actually produce plateletenriched plasma with a platelet count exceeding that in the original whole-blood sample. For these reasons, preliminary extraction and chromatographic purification of
plasma (or tissue) samples is suggested with the aim of eliminating these interferences
and increasing accuracy (i.e., specificity) \citep{bib224} \citep{bib229} \citep{bib230}.As a matter of fact, RIAs with a preliminary extraction/purification step gave in general more accurate results than the
same RIAs without this step \citep{bib224} \citep{bib229}.

High sensitivity for detection of low ANP concentrations is also required for a plasma ANP assay, due to low circulating concentrations of the hormone in healthy subjects (about 5-50 ng/l, corresponding to about 1.6-16.2 pmol/l) (Table 4.4). Chromatographic extraction of large volumes of plasma (≥3 ml) has therefore been suggested to
increase the precision of the measurement in this concentration range \citep{bib224} \citep{bib229} \citep{bib230}.

As mentioned above, a preliminary step for the extraction/purification of the peptides from the plasma or tissue samples is generally requested when RIAs for ANP are
used. However, this step greatly decreases assay precision and practicability, thus generating conflicting results \citep{bib230}. The main reasons are: 1) the quantitative recovery of
native hormone added to plasma samples after extraction with Sep-Pak C18 cartridges
is generally poor for all the most used procedures (~50-70\%); 2) the chromatographic
pattern of 125 I-radiolabeled ANP, generally used for the routine assessment of extraction
recovery, may be different from that of native hormone \citep{bib22} \citep{bib229} \citep{bib230}, so that the recovery
of tracer could not be a suitable estimate of true recovery of native ANP present in plasma samples; 3) the extraction recovery was not verified for all the individual samples
assayed, but only a mean run recovery was measured in general. These factors make it
difficult to standardize a chromatographic procedure for the preliminary extraction/purification step of ANP present in plasma or tissue \citep{bib21} \citep{bib22} \citep{bib230}.

Finally, since a tracer with the highest specific activity must be used to increase the
assay sensitivity of the RIA system, the peptide is generally labeled with two atoms of
radioiodine (namely 125 I) per molecule by adding these isotopes directly to the ring of
the last C-terminal amino acid (tyrosine). Unfortunately, this type of di-iodinated tracer is more rapidly degraded in vitro than the mono-iodinated tracer \citep{bib22} \citep{bib233}. Furthermore,
radiolabeled ANP, labeled with one or two 125 I atoms in the tyrosine ring, is not stable
in aqueous solutions, this representing another important issue for ANP determination by RIA \citep{bib233}.

Because of the different analytical characteristics and performances of RIA methods,
it is not surprising that a great variability in ANP measured values was observed in an
external quality assessment of five Italian laboratories with long experience in performing
ANP assays: between-assay coefficients of variations (CV) >50\% were indeed found \citep{bib230}.

\subsubsection{ Non-Competitive Immunoassay Methods for ANP}
Non-competitive IRMA methods for the measurement of plasma ANP have been set up
to overcome the problems observed with competitive assays \citep{bib234} \citep{bib235} \citep{bib236} \citep{bib237}. All these methods
are “two-site” (sandwich) IRMAs, which use two specific monoclonal antibodies or antisera prepared against two sterically remote epitopes of the ANP molecule. In some methods, using a solid phase as a separation system, one of the two antibodies is radiolabeled
with 125 I and the other is coated onto wells of a microliter plate \citep{bib234} or beads \citep{bib236} \citep{bib237}. Other
IRMAs use a liquid-phase system in which both the specific anti-ANP antibodies (i.e., radiolabeled and non-radiolabeled) are in solution and the bound fraction is separated from
the free fraction using a precipitating antiserum with centrifugation \citep{bib235}.

IRMAs have some advantages over RIAs. First, IRMAs do not use radiolabeled ANP as
tracer but the more stable radiolabeled antibody; consequently, these methods do not
present the tracer instability-related problems that can affect some ANP RIAs. Second,
because ANP IRMA methods are in general highly sensitive and not significantly affected by non-specific (so-called “matrix effects”) as well as specific interferences, these methods do not generally require preliminary extraction and purification of the plasma sample, which can reduce assay precision and practicability as well as increase the cost. Finally, IRMAs use a lower plasma volume (generally 0.1-0.3 ml) for the assay than RIAs \citep{bib237}.
Figure 4.1 displays the imprecision profiles of three RIAs and an IRMA for ANP
measurement \citep{bib229} \citep{bib230} \citep{bib237}. The IRMA method showed better precision than RIAs, especially in the low range concentrations. The detection limit of RIAs ranged between 10
and 20 ng/l (3.2-6.5 pmol/l) \citep{bib229} \citep{bib230}, while the detection limit of IRMA was ~2 ng/l (0.65
pmol/l) \citep{bib237}. The working range (arbitrarily defined as the interval of ANP values measured with an imprecision [as CV] <15\%) of IRMA was greater (10-2,000 ng/l; 3.2-650
pmol/l) than that of RIA (on average 20-1,000 ng/l; 6.5-325 pmol/l) \citep{bib230} \citep{bib237}.

It is noteworthy that IRMAs generally have reference intervals superimposable on
those obtained with the most accurate RIAs, using a preliminary extraction/purification
step (Table 4.4).

The IRMA system, which uses two monoclonal antibodies, is more specific for the
intact ANP than RIAs, as indicated by the results reported in Table 4.3. However, it is
theoretically conceivable that the intact peptide proANP 1-126 , containing the biologically active hormone ANP in its C-terminal 99-126 amino acid residues, may also be measured by this IRMA (as well as by all RIAs). This would be a considerable drawback if the
assayed concentrations of ANP and intact proANP 1-126 are similar, which is evidently the
case when cardiac tissue extracts are assayed. Under these experimental conditions, if the
aim of the study is to measure separately both the biologically active ANP and the intact
precursor proANP 1-126 , IRMA methods (as well as all RIAs) require a preliminary extraction/chromatographic step to separate the two molecules accurately before the assay.

On the other hand, an intrinsic problem of the non-competitive “two-site” immunometric methods is the so-called “hook” effect at high concentrations of analyte. Indeed,
this effect was found in one IRMA for ANP concentrations >250 pmol/l \citep{bib235}, but not in
another IRMA for ANP concentrations of up to 2,000 ng/l (650 pmol/l) \citep{bib237}. The “hook”
effect can probably be avoided by using a monoclonal antibody coated onto the solid
phase showing high binding capacity \citep{bib234}.

The only disadvantage of commercially available IRMA compared to RIA is the
greater cost. For laboratories that have an opportunity to set up an RIA method without using expensive commercial products (e.g., to prepare specific antibodies and tracers directly), purchasing a commercial IRMA kit may increase the cost by approximately three times. On the other hand, for laboratories in which a commercial RIA kit
is already being used, the higher cost of an IRMA kit may be largely counterbalanced
by the increase in accuracy, precision, and, especially, practicability.

At the present time, no non-competitive, fully automated immunoassay system for
ANP is commercially available.

\subsubsection{ Determination of NT-proANP}

NT-proANP is generally measured by direct RIA methods, i.e., without a preliminary
extraction/purification step \citep{bib21} \citep{bib22} \citep{bib220} \citep{bib238} \citep{bib239} \citep{bib240}. NT-proANP RIAs are more practicable and
cheaper than ANP IRMAs, but they can be affected by some analytical problems. The
intact NT-proANP is a long peptide, implying that an anti-NT-proANP antiserum may
recognize only a few epitopes of the peptide \citep{bib21} \citep{bib22} \citep{bib239}. Consequently, if low-molecular
mass fragments of NT-proANP are present within the circulation, RIA using different
antisera or monoclonal antibodies can show significantly different results \citep{bib21} \citep{bib22}.

In their extensive studies,Vesely and coworkers have suggested that NT-proANP can
be degraded in vivo with the production of at least three different biologically active peptides (i.e., NT-proANP 1-30 , proANP 31-67 , and proANP 79-98 ) \citep{bib241} \citep{bib242} \citep{bib243} \citep{bib244} \citep{bib245} \citep{bib246}. These peptides are
unable to bind the specific ANP/BNP receptors, but they show peculiar biological effects
(such as blood pressure lowering, diuretic, natriuretic, and/or kaliuretic properties)
and a considerably longer plasma half-life (hours versus minutes) than that of ANP
and BNP. Therefore, these peptides should be considered as distinct peptide hormones
too \citep{bib246}. Further studies are, however, necessary to confirm the physiological relevance
of these peptides. Of course, to accurately incasure these NT-proANP related peptides,
specific methods for each peptide are necessary \citep{bib21}.

A two-site IRMA for the NT-proANP 1-98 , which uses a first monoclonal antibody
against the peptide NT-proANP 1-25 and a second antibody against the peptide NTproANP 43-66 , has also been set up \citep{bib247}. As expected, this IRMA was more sensitive, precise, and specific for the intact NT-proANP than RIA \citep{bib247}.

\subsection{ Determination of BNP and NT-proBNP}

\subsubsection{ First-Generation (Manual) BNP Assays}

The first methods set up for BNP determination were RIAs \citep{bib248} \citep{bib249} \citep{bib250} \citep{bib251} \citep{bib252} \citep{bib253} \citep{bib254} \citep{bib255}. Because BNP shows
circulating values and peptide structure similar to those of ANP, BNP RIAs were also
characterized by the same analytical problems (i.e., low degree of precision, sensitivity, accuracy, and practicability). As a consequence, a preliminary extraction/purification
step was generally requested and a relatively large volume of plasma sample (3-10 ml)
needed for the determination of BNP by RIA \citep{bib253}. As for ANP, the recovery of BNP after
the extraction/purification procedure with Sep-Pack C18 cartridges (generally used in
RIA methods) was poor (~70\% on average) \citep{bib250} \citep{bib253}.

To overcome these problems, a two-site, solid-phase (sandwich) IRMA method has
been proposed for the determination of the biologically active form of BNP \citep{bib213} \citep{bib256} \citep{bib257}.
The IRMA uses two different monoclonal antibodies that recognize the C-terminal
region and the intramolecular ring structure of BNP, respectively. One of these antibodies is radiolabeled with 125 I and the other is coated on the beads in solid phase. The
detection limit of this IRMA was ~2 ng/l (0.6 pmol/l), with a working range (BNP values measured with an imprecision [CV] <15\%) of 7-2,000 ng/l (2-578 pmol/l). Only
\subsection{ ml of plasma was necessary for the assay \citep{bib213} \citep{bib257}.}

\subsubsection{ Second-Generation (Automated) BNP Assays}

In the past few years a new generation of fully automated immunoassays for BNP has
become commercially available, including POCT methods. These assays are non-competitive sandwich-type immunoassays that use non-radioactive materials as labels for
antigen/antibody reaction and two monoclonal antibodies or a combination of monoclonal and polyclonal antibodies for BNP binding (Fig. 4.2) \citep{bib27} \citep{bib28} \citep{bib215} \citep{bib258} \citep{bib259} \citep{bib260} \citep{bib261} \citep{bib262} \citep{bib263} \citep{bib264}. Usually one
antibody binds to the ring structure and the other antibody either to the C- or N-terminal end of BNP, respectively (Table 4.5).

\subsubsection{ Determination of NT-proBNP}

The first assay for the measurement of NT-proBNP was reported by Hunt et al.\citep{bib265} \citep{bib266}.Later
on,other groups have developed “in-house”methods \citep{bib267} \citep{bib268}.All these methods were direct
(i.e., without extraction) competitive assays. Commercial EIA kits for NT-proBNP related
peptides are also available and their analytical performances have been evaluated \citep{bib25} \citep{bib211} \citep{bib269}.
A two-site sandwich immunoassay for NT-proBNP using detection by electrochemiluminescence (electrochemiluminescence immunoassay [ECLIA]) and fully automated instrumentation has recently been developed and its analytical performance
extensively evaluated \citep{bib269} \citep{bib270} \citep{bib271} \citep{bib272} \citep{bib273} \citep{bib274} \citep{bib275}. Two polyclonal antibodies directed against amino acid
residues 1-21 and 39-50 of the NT-proBNP molecule are used (Table 4.5). The peptide
concentration is measured on the Elecsys systems (Roche Diagnostics). Quite recently,
methods using the same antibody combination and calibrator material have been developed by Dade Behring and Ortho Clinical Diagnostics for their automatic platforms. Several others are under development.

As discussed below,the heterogeneity of the NT-proBNP fragments in blood has a major
influence on antibody detection and assay specificity (Fig. 4.3) \citep{bib21} \citep{bib24} \citep{bib276}. Therefore, it is not
surprising that different NT-proBNP assays may produce different results (Table 4.6) \citep{bib23}.

\subsubsection{ Specificity of BNP/NT-proBNP Assays}

It is known that BNP derives from the preproBNP, which contains a signal peptide
sequence at the N-terminal end. After the signal peptide is cleaved, proBNP is further
split proteolytically into NT-proBNP and the biologically active peptide hormone, BNP.
However, the site at which proteolytic cleavage of proBNP takes place is still being debated. Most of it occurs within or on the surface of cardiomyocytes before the secretion in
blood, but small amounts of the intact proBNP are also found in the circulation, indicating that some proteolysis may occur in the circulation. Thus, with the exception of
preproBNP, all the metabolically BNP-related molecules are likely to be present in plasma, which is the biological sample generally used for measurements (Fig. 4.4) \citep{bib276} \citep{bib277} \citep{bib278} \citep{bib279}.

This fact was shown most clearly in the study of Hunt et al. \citep{bib279}, which compared the
amounts of proBNP-derived molecular forms found in cardiac tissue and in plasma.
Qualitatively, the same peaks were seen after HPLC fractionation of an atrial extract
as were seen in plasma from a patient with heart failure, documenting the presence of
proBNP, NT-proBNP, and BNP in both samples. The presence of intact proBNP in human
plasma, in addition to BNP and NT-proBNP fragments, is important because of the
potential that it may create problems regarding analytical specificity in the measurement
of these peptides in plasma.

The N-terminal region of proBNP contains a leucine zipper-like sequence motif that
may induce peptide oligomerization in plasma under physiological conditions, producing either a trimer or tetramer of proBNP \citep{bib280}. These oligomerized molecules may
expose or obscure epitopes recognized by the antibodies used in commercial assays.
However, a more recent study indicated that human synthetic NT-proBNP exists as a
monomer with an unordered random coil in neutral pH physiological salt solution \citep{bib281}.

Experimental data also support the cleavage of BNP by plasma proteases. Proteolysis of the C-terminal structure by kallikrein occurs after activation of the coagulation
contact activation system by a negatively charged surface. This can occur in vivo on the
intraluminal surface of a damaged vessel and/or in vitro on the glass wall of blood collection tubes \citep{bib282} \citep{bib283}. Proteolytic cleavage of the two N-terminal amino acid residues,
serine and proline, may occur immediately after blood collection or still within the circulation, making the N-terminal residue of BNP very sensitive to degradation \citep{bib278}. It may
be critical to take these enzymatic cleavages, particularly the one at the N-terminus,
into account when choosing epitopes for antibody production and immunoassay design.
The disulfide-bond-mediated ring structure appears to be stable in blood samples.
However, this does not mean that degradation can never occur even in this site \citep{bib283}.
Finally, a recent study has shown that circulating NT-proBNP is heterogeneous and that
most immunoreactive NT-proBNP is significantly smaller in size than NT-proBNP 1-76
because of truncation at both termini \citep{bib284}. This fragmentation is more pronounced in
serum than in plasma.

In the light of all these data,the complexity of the measurement of BNP and related peptides is clear.Assays with critical epitope requirements may differ in their reactivity with circulating peptides, so that commercial assays, nominally measuring the same analyte, may
be differently affected by cross-reactivity problems (Table 4.7). Plasma BNP may be overestimated if the sandwich against BNP is formed by an antibody directed against the ring
structure of the molecule and the second antibody is directed to the C-terminal end. In this
case, plasma intact proBNP is also detected by the assay \citep{bib278}. On the other hand, the combination of an antibody directed against the ring structure with an antibody against the
N-terminal part of the molecule appears to be specific for the BNP,being,however,more prone
to the peptide degradation by plasma proteases which can affect the N-terminus position
\citep{bib263} \citep{bib277} \citep{bib278}. This may result in a significant instability of the sample \citep{bib270}.

As regards NT-proBNP measurement, the recognized antibody epitopes are also crucial for immunoassay specificity. It is likely, although frequently not reported, that the antibodies used in many so-called “NT-proBNP” assays may also detect many of the circulating NT-proBNP split products, in addition to measuring proBNP. Further, several clinical studies reported in the literature have been performed by home-made non-commercial methods, which are not available to other research groups. To obviate the NT-proBNP heterogeneity in plasma, Goetze et al. developed a processing-independent analysis
(PIA) for quantification of total proBNP, i.e. intact proBNP and its fragments in plasma
\citep{bib285}. Calibrators were prepared from synthetic tyrosine-extended proBNP 1-10 and an
antibody directed against amino acid sequence 1-10 of human proBNP was used. Before
measurement, plasma was treated with a proteinase (trypsin) that cleaved all proBNP peptides present in plasma to the 1-21 fragment. In this way, intact proBNP and all biologically derived fragments in the sample are cleaved into the same analyte and assayed
together as equimolar amounts by RIA. The clinical utility of this approach remains to
be determined, but it would overcome the issue of differential detection of circulating
proBNP-derived fragments by different assays.

Different assays, theoretically measuring the same analyte, may thus produce significantly different results, not only in terms of proportional bias, but also displaying significant intercept values, meaning different analytical specificities \citep{bib27} \citep{bib28} \citep{bib262} \citep{bib269}. Generally speaking, it would be correct to assert that there are currently two types of commercially available assays. The former group, which uses a sandwich with an antibody
against the ring portion and a second antibody against the N-terminal part of BNP
peptide, is specific for the measurement of biologically active hormone, BNP. The latter, which is more heterogeneous, comprises methods that measure proBNP and all (or
some) of its metabolic products, including BNP and NT-proBNP, with variable specificity.
Nevertheless, when using these cardiac natriuretic peptides as biomarkers of cardiac dysfunction (and not for the evaluation of a biologically active hormone and its possible
alterations), both analytical approaches can be acceptable, once their clinical usefulness has been proven. No two assays are, however, analytically equivalent at present.
The BNP/NT-proBNP values are significantly dependent on the type of assay used, as
a result of the specificity of the employed antibodies and of different calibration materials. As results are heavily method-dependent, it should clearly be stated that reference intervals and decision limits derived from clinical studies are only valid for the
particular assay used and must not be extrapolated to other assays (Table 4.8).

Therefore, there is a need for assay standardization that, through the selection of
appropriate reference materials and an adequate reference measurement procedure
useful for their certification, and especially through the definition of the analyte to
be measured, allows the same results to be obtained even when the measurements are
performed with different methods \citep{bib286}. In the meantime, in order to employ correctly the measurement of these markers in clinical practice, it is mandatory to use
method-dependent reference limits and cut-off values without extrapolations from
one method to another \citep{bib287}.

\subsubsection{ Quality Specifications for BNP/NT-proBNP Assays}

With the clinical meaning of these markers of cardiac function being firmly documented, it now seems important to focus more on the biology of these peptides and
related analytical issues. Only after appropriate analytical quality specifications are
addressed, will the many issues pertaining to methodological differences that lead to nonharmonized concentration values and clinical interpretation of natriuretic peptide concentrations be reconciled. The responsibility of defining and implementing these issues
should be shared among laboratories, industry, clinicians, and regulatory agencies on
an international front.

A report was recently published by the Committee on Standardization of Markers of
Cardiac Damage of the International Federation of Clinical Chemistry and Laboratory Medicine (IFCC), concerning quality specifications for BNP assays \citep{bib288}. The recommendations proposed in this report are mainly intended for use by manufacturers of commercial assays and regulatory agencies. However, several points are also relevant for
clinical trial groups and research investigators, as well as for clinical laboratories interested in measuring BNP or NT-proBNP. Both analytical and pre-analytical factors were
addressed, as shown in Table 4.9.

First, definitive information about the synthesis and catabolism of BNP-related peptides is needed to determine the characteristics of the calibrator materials to be used in
standardizing clinical assays. Their composition should closely resemble that of the
molecule showing the greatest utility in a given clinical situation. For this type of definition, an evidence-based international agreement is required.

Second, assay specifications including antibody specificity and cross-reactivity characteristics to structurally related (e.g., metabolized and degraded) molecules need to be
clearly delineated.

Third, assay imprecision and interferences have to be entirely described. Because of
a consistently high biological variation for both BNP and NT-proBNP, very low assay
imprecision may be unnecessary (see Chapter 5, for a more detailed discussion of this point). A desirable total imprecision (as CV) of <15\% at BNP/NT-proBNP concentrations within the reference interval has been recommended \citep{bib288}. Acceptable imprecision was observed with different commercial automated assays, the Abbott
AxSYM method having the largest CV range (Fig. 4.5). In general, no problems have been
reported with the most important endogenous interferents (hemolysis, hyperbilirubinemia, hypertriglyceridemia) (Table 4.10) \citep{bib28} \citep{bib270} \citep{bib289} \citep{bib290} \citep{bib291} \citep{bib292}. Grossly hemolyzed samples
might, however, be a problem in the Beckman Access BNP assay \citep{bib290}.

Finally, there are important issues related to the type of sample to be used for BNP/NTproBNP measurements and the in vitro stability of these analytes. For BNP assays, the
EDTA plasma is the only suitable specimen \citep{bib293}. Conversely, it appears that for measuring NT-proBNP by the ECLIA method, serum is the sample of choice \citep{bib272}. With this
assay, EDTA plasma gave a consistent negative bias (8\% on average) compared with
matched serum samples, although studies did not indicate the variability among samples.

Blood samples should not be collected in glass tubes when using an immunoassay
employing in the sandwich an antibody against the C-terminus for BNP measurement.
It has been demonstrated that the above antibody is highly susceptible to the effect of
kallikrein, a plasma protease activated by contact with the wall of the glass tube, that
degrades the C-terminal portion of BNP (and proBNP), making impossible the identification of the molecule by the immunoassay using the above-reported antibody in the
sandwich \citep{bib282} \citep{bib283} \citep{bib294}. The stability of the blood sample can therefore be obtained by
the use of plastic collection tubes. The in vitro stability is not a problem for NT-proBNP, while BNP is more unstable as proteolytic phenomena deprive it of the two N-terminal amino acids (Table 4.11). In general, BNP should be measured within 4 hours of
collection if the sample is stored at room temperature. If the testing cannot be performed within 4 hours, the plasma should be separated, a kallikrein- or serine-specific protease inhibitor added, and the specimen stored refrigerated at 2-8°C for up to 24
hours or frozen at -70°C if stored for longer periods.

\subsection{ Determination of Other Natriuretic Peptides (CNP, DNP, and Urodilatin)}

The methods set up for determination of CNP \citep{bib298}, DNP \citep{bib299}, and urodilatin \citep{bib2100} \citep{bib2101}
were all RIAs. Because these peptides show circulating values and peptide structure
similar to those of ANP and BNP, these RIAs were also characterized by the same analytical problems (i.e., low degree of precision, sensitivity, accuracy, and practicability).
As a consequence, a preliminary extraction/purification step was generally requested and
a relatively large volume of plasma sample (i.e., 1-2 ml for CNP assay) needed for the
determination \citep{bib295}. The assay recovery after the extraction/purification procedure with
Sep-Pack C18 cartridges is usually not complete; for example, it is only about 80-85\%
for the CNP assay \citep{bib298}. As expected for manual methods with a preliminary extraction
step, assay precision is relatively poor. Indeed, the total between-assay variability of the
CNP assay is 25.2\% \citep{bib298}.

Due to their poor experimental practicability and long TAT, these immunoassay
methods are usually used only in clinical or pathophysiological studies rather than in
routine clinical practice. It is important to point out that the lack of suitable immunoassays could be an explanation for the relatively small amount of clinical data available for
CNP, DNP and urodilatin compared with ANP and BNP.

Some specific points should be taken into account about the measurement of CNP,
DNP an urodilatin, respectively.

DNP was first identified in venom of the green mamba snake \citep{bib2103}, and then in
mammal plasma and tissues, but its origin, biochemical and pathophysiological characteristics, and so the clinical relevance of its assay, are still unclear \citep{bib2102} \citep{bib2104}.
Urodilatin is identical in structure to the circulating 28 amino acid human ANP (99126), with addition of four amino acids (Thr-Ala-Pro-Arg) at the NH 2 -terminus (Fig. 4.6).
Urodilatin is synthesized by the same gene that synthesizes ANP, but in the kidney, as
opposed to all other tissues that have been investigated, the ANP pro-hormone is processed
differently, resulting in urodilatin rather than ANP being formed \citep{bib2100} \citep{bib2101} \citep{bib2102}. Urodilatin
seems to be produced only in renal tissue and is not present in plasma (or at very low
concentration), but only in urine \citep{bib2100} \citep{bib2101} \citep{bib2102}; consequently, only urine or renal tissue samples must be used for urodilatin assay. The clinical relevance of urodilatin assay should be
limited only to renal diseases \citep{bib2102}. From a theoretical point of view, the set-up of a suitable urodilatin immunoassay presents some difficulties due to the very close structural
similarity between ANP and urodilatin (Fig. 4.6). In order to distinguish better between
these two peptides, an antibody against the first N-terminus four amino acids of urodilatin should be used \citep{bib2100}. However, immunoreactivity and biological activity may be
poorly correlated if a competitive immunoassay based on this antibody is used. Indeed,
the biological acivity of CNH depends on the integrity of the cysteine bridge, so that only
the use of a non-competitive immunoassay using two antibodies (one against the N-terminus and the other against the part of the peptide including the cysteine bridge) should
allow a better correlation between immunoreactivity and biological activity.
More information is available about the biochemical and pathophysiological characteristics of CNP, as well as the clinical relevance of its assay in plasma or tissue
samples.

CNP is a peptide that was initially identified in the central nervous system \citep{bib2105}.
CNP is expressed predominantly in the endothelium, but the central nervous system,
myocardium, and gastrointestinal and genitourinary tracts can also express the peptide
\citep{bib2106} \citep{bib2107}. CNP has both structural and physiological similarities to other known natriuretic peptides, and a shorter circulatory half-life, and is the most highly conserved
form of the natriuretic peptides between species \citep{bib2106} \citep{bib2107} \citep{bib2108} \citep{bib2109} \citep{bib2110}. Studies in fish, based on
nucleotide and amino acid sequence similarity, suggest that the natriuretic peptide
family of iso-hormones may have evolved from a neuromodulatory CNP-like brain peptide (Fig. 3.6) \citep{bib2109} \citep{bib2110}.

Two mature forms of the peptide exist, being derived from a 126 amino acid prepro-hormone. Both contain the 17 amino acid ring common to all members of the natriuretic peptide family (Fig. 4.7). The peptide with higher molecular weight, CNP-53, predominates in tissues, whereas CNP-22 is found mainly in plasma and in cerebrospinal
fluid. It has been suggested that CNP-53 may function, at least in part, as a storage form
of the peptide, while CNP-22 circulates in the plasma \citep{bib2106} \citep{bib2108} \citep{bib2111} \citep{bib2112} \citep{bib2113}.

Most data on the biological effects of the peptide relate to the 22 amino acid form and
its biological effects are currently being investigated at different sites in various species.
Moreover, the endothelial site of production of CNP peptide and the proximal location
of its receptor in vascular smooth muscle suggest that this vascular natriuretic peptide
system may play a role in concert with other local systems (nitric oxide, prostaglandins)
in the control of vascular tone, counteracting the vasoconstrictor systems (endothelin,
angiotensin II; see Chap. 3 for more details) \citep{bib2107} \citep{bib2114} \citep{bib2115}. Several studies demonstrated that CNP exerts a protective effect on endothelial function by decreasing shear
stress, modulating coagulation and fibrinolysis pathways, and inhibiting platelet activation (Fig. 3.15) \citep{bib2114} \citep{bib2115}. CNP can also inhibit the vascular remodeling process as well
as coronary restenosis post-angioplasty and the adverse effects of ischemia/reperfusion injury \citep{bib2107} \citep{bib2113} \citep{bib2114} \citep{bib2115} \citep{bib2116} \citep{bib2117} \citep{bib2118} \citep{bib2119} \citep{bib2120}.
These studies suggest that the assay of CNP and the N-terminus fragment of propeptide (NT-proCNP) should have some clinical relevance in all pathophysiological
conditions characterized by endothelium dysfunction. Indeed, increased plasma levels of CNP/NT-proCNP have already been identified in a number of pathological conditions (such as hypoxia, sepsis, chronic liver, renal or heart failure) likely characterized by endothelial dysfunction \citep{bib2106} \citep{bib2114} \citep{bib2121} \citep{bib2122} \citep{bib2123} \citep{bib2124} \citep{bib2125} \citep{bib2126} \citep{bib2127} \citep{bib2128} \citep{bib2129}. However, clinical interpretation
of results of CNP assays may be difficult because this peptide has paracrine rather
than hormonal action \citep{bib2112} \citep{bib2113} \citep{bib2114} \citep{bib2115}; as a result, variations of circulating levels of CNP
may be poorly (or even not at all) correlated to biological effects actually performed
by the peptide in vascular tissue. Furthermore, blood collection, pre-analytical purification step and analytical assay performance may strongly affect the results of CNP/NTproCNP assay \citep{bib298}.

\subsection{ Summary and Perspectives}

The important role of CNH (especially BNP and NT-proBNP) assay in screening for
heart disease, detection of left ventricular systolic and/or diastolic dysfunction, and differential diagnosis of dyspnea, prognostic stratification of patients with congestive heart
failure has been confirmed even more recently \citep{bib24} \citep{bib287} \citep{bib288} \citep{bib2130} \citep{bib2131}. In particular, the
BNP/NT-proBNP assay was included in the first step of the algorithm for the diagnosis
of acute heart failure along with the electrocardiogram and chest X-ray examination,
as recently confirmed by the Task Force of the European Society of Cardiology for the diagnosis and treatment of chronic heart failure \citep{bib2132}.According to these recommendations,
in order to introduce the routine measurement of BNP or NT-proBNP in clinical practice, reliable immunoassay methods should be commercially available.

CNH can be measured with competitive or non-competitive immunoassays. Noncompetitive immunoassays generally share better degree of sensitivity, precision and
specificity than the respective competitive immunoassays. In the last years a new generation of fully automated immunoassays, for BNP and NT-proBNP, including some
POCT methods, became commercially available. These assays are non-competitive sandwich-type immunoassays which uses non-radioactive materials as labels for antigen/antibody reaction and two monoclonal antibodies or a combination of monoclonal and
polyclonal antibodies for peptide binding \citep{bib21} \citep{bib27} \citep{bib28} \citep{bib215} \citep{bib258} \citep{bib259} \citep{bib260} \citep{bib261} \citep{bib262} \citep{bib263} \citep{bib264}. In particular, for BNP assay,
usually one antibody binds to the ring structure and the other antibody either to the Cor N-terminal end, respectively (Tables 4, 5). Although these fully automated commercial assays produce results in a short time and with excellent analytical imprecision, the
BNP/NT-proBNP values continue to significantly depend on the type of assay used, as a
result of the specificity of the employed antibodies and of different analytical standardization deriving from the different calibration materials \citep{bib286} \citep{bib287} \citep{bib288}. The mentioned
discrepancies among the concentrations obtained with different methods are directly
reflected over reference and decisional limits, which often change according to the method
used \citep{bib262} \citep{bib287} \citep{bib288} (Tables 4 - 8). Therefore, there is a need for standardization that, through
the selection of appropriate reference materials and adequate reference measurement
procedures useful for their certification, and especially through the definition of the
analyte to be measured, would allow to obtain comparable results, even when the measurements are performed with different methods \citep{bib21} \citep{bib286} \citep{bib287} \citep{bib2288}.


Clinical Considerations and Applications in Cardiac Diseases


\subsection{ Circulating Levels of Cardiac Natriuretic Hormones:}

Physiological Considerations and Clinical Interpretation
CNH are powerful hormones with important physiological effects. Consequently, by considering their assay only as a marker for cardiac disease may result in a misinterpretation
or underestimation of their biological action and of their pathophysiological role in cardiovascular as well as other diseases.In the first part of Chapter 5 ,
some physiological conditions in which the interpretations of increased levels of CNH
may be difficult or provoke misunderstanding will be discussed in detail. In particular,
the influence of age and sex hormones on circulating levels of CNH will be reviewed .As far as the clinical interpretation of variations of circulating levels of CNH is
concerned, some important points will be stressed: 1) the pathophysiological and clinical
consequence of the progressive resistance to biological actions of CNH in patients with
heart failure ; 2) the inter-relationship between hemodynamic mechanisms
and activity of neuro-endocrine system in determining the variation of circulating levels
of CNH ; 3) the clinical relevance of variation of CNH levels .
The discussion of these points will lay the pathophysiological foundations for better understanding of the second part of this Chapter, which concerns the diagnostic and prognostic role of CNH assay in patients with cardiovascular disease .

\subsubsection{ Influence of Age and Gender}

The circulating levels of CNH are regulated or modified by several physiological factors
(such as circadian variations, age, gender, exercise, body posture, and water immersion),
eating habits (especially sodium intake), clinical conditions (Table 5.1), and drugs (including corticosteroids, sex steroid hormones, thyroid hormones, diuretics, angiotensin-converting enzyme [ACE] inhibitors, and adrenergic agonists and antagonists) \citep{bib31} \citep{bib32} \citep{bib33} \citep{bib34} \citep{bib35} \citep{bib36}.

The wide variations of circulating levels of CNH in adult healthy subjects in relation to aging and gender could have a particular clinical relevance \citep{bib37} \citep{bib38} \citep{bib39} \citep{bib310} (Figs. 5.1 and 5.2, Table 5.2). Indeed, Vasan et al. \citep{bib39} recently demonstrated that the diagnostic accuracy of CNH assay for community screening is gender-dependent.
In order to explain these variations, the possible influence of sex steroid hormones
on the CNH system, as well as the modification of the cardiovascular system with aging,
should be taken into account \citep{bib311} \citep{bib312} \citep{bib313} \citep{bib314}. According to these mechanisms, the higher CNH
values of women during the fertile adult period could be explained by the physiological stimulation of female sex steroid hormones. In particular, the BNP concentration is
on average 36\% higher in women than in men aged less than 50 years \citep{bib37} (Figs. 5.1 and 5.2, Table 5.2). The increase in CNH with aging may be due to the para-physiological
decline in myocardial function and other organs (including kidney), typical of senescence \citep{bib315}. In this case, the CNH assay may be considered as a biochemical marker of
increased risk of cardiac morbidity in old age \citep{bib316}. Moreover, the increase in CNH with
aging may be due to a decrease in their clearance rate. Indeed, an age modulation of
maximum binding capacity of clearance (C-type) receptors for CNH was reported in
platelets of elderly persons \citep{bib317}.


\subsubsection{ Comparison between the CNH Assay and that of CNH-Related Pro-Hormone Peptides}

All CNH derive from pre-pro-hormones (i.e., preproANP and preproBNP), containing
a signal peptide sequence at the amino-terminal end. The pro-hormones (i.e., proANP
and proBNP) are produced by cleavage of signal peptide, and then are further split into
inactive longer NH 2 -terminal fragments (i.e., NT-proANP or NT-proBNP), and a biologically active shorter COOH-terminal peptide (i.e., ANP or BNP), which are secreted
in the blood in equimolar amounts (Figs. 3.11 and 3.12). However, ANP and BNP have
a shorter plasma half-life and consequently lower plasma concentration, compared to
NT-proANP and NT-proBNP (Table 4.1) \citep{bib31} \citep{bib32} \citep{bib33} \citep{bib34} \citep{bib35} \citep{bib36} \citep{bib37} \citep{bib318}.

Studies on structure-activity relationships have shown the importance for the binding to the specific receptors of the central ring structure of CNH, formed by a disulfide bridge between the two cysteine residues. For this reason, only ANP and BNP, which
present the disulfide bridge in the peptide chain, share the typical hormonal activity of
CNH, while the NT-proANP and NT-proBNP do not \citep{bib31} \citep{bib32} \citep{bib33} \citep{bib34} \citep{bib35} \citep{bib36} \citep{bib37}.

Theoretically, setting up an immunoassay for NT-proANP and NT-proBNP should
be easier because their plasma concentrations are higher than ANP and BNP \citep{bib318}.

On the other hand, NT-proANP and NT-proBNP immunoassays may be affected by several analytical problems, mainly concerning the different assay specificities; consequently, very different results are produced by different methods with a large bias \citep{bib32} \citep{bib35} \citep{bib36} \citep{bib318} (Table 4.1). The different analytical performance might affect the diagnostic
accuracy of the assays, in discriminating between subjects with or without cardiac
disease \citep{bib32} \citep{bib35} \citep{bib36} .
The respective advantages of measuring biologically active peptide hormones (ANP
and BNP), or inactive peptides (NT-proANP and NT-proBNP) are summarized in Table. The assay of the inactive propeptides better fits the definition of disease marker than the assay of circulating levels of ANP or BNP, which, on the other hand, may be considered a more reliable index of the activation status of the CNH system.
Considering the biochemical and physiological characteristics of the different peptides, it is conceivable that ANP is a better marker of acute overload and/or rapid cardiovascular hemodynamic changes than BNP and, especially, than NT-proANP or
NT-proBNP \citep{bib32} \citep{bib35}. For example, circulating levels of ANP are known to be more affected by body position and decreased to a greater extent by a hemodialysis session in
patients with chronic renal failure than those of BNP, while plasma NT-proANP is
unchanged \citep{bib322}. Furthermore, ANP increases more than NT-proANP during rapid
ventricular pacing \citep{bib323}.


\subsubsection{ Resistance to the Biological Action of CNH}

Deficiencies in the activity of the CNH system could explain altered electrolyte and
fluid homeostasis occurring in chronic heart failure (HF) \citep{bib324} \citep{bib325} \citep{bib326}. However, the hypothesis proposing HF as a syndrome of CNH deficiency was challenged when the CNH
system was more carefully investigated in experimental animals and humans \citep{bib324} \citep{bib325} \citep{bib326}.

Patients with chronic HF show increased CNH plasma levels compared to normal subjects (Table 3.1, Fig. 3.13). These findings have been recently defined the “endocrine
paradox” in HF \citep{bib36}, i.e., extremely high circulating levels of hormones with powerful natriuretic activity in patients with congestive HF, who show physical signs of fluid retention
and vasoconstriction due to a relatively poor biological activity of the CNH system.
A blunted natriuretic response after pharmacological doses of ANP and BNP has
been observed in experimental models and in patients with chronic HF, suggesting a
resistance to the biological effects of CNH, principally natriuresis \citep{bib324} \citep{bib325} \citep{bib326} \citep{bib327} \citep{bib328} \citep{bib329} \citep{bib330} \citep{bib331}. This resistance syndrome was also demonstrated by in vivo turnover studies using radioactive
tracers in patients with HF \citep{bib332} \citep{bib333}.

Resistance to the biological action of CNH could, theoretically, have three different
causes (Table 5.3). First, circulating CNH could be, at least in part, inactive. Furthermore, a great fraction of CNH could be inactivated by plasma and tissue proteases
before they bind to specific receptors. These two conditions account for all possible
mechanisms acting at the pre-receptor level. Second, down-regulation of specific receptors could explain a reduced CNH activity. Finally, some mechanisms can act at postreceptor level, counteracting the biological effects of CNH.
Mechanisms acting at pre-receptor level: Some peptides, derived in vivo or in vitro from
degradation of intact proBNP, are biologically inactive, although they can be measured
by immunoassay methods \citep{bib35} \citep{bib36} \citep{bib318}. Since the circulating levels of intact proBNP and of
its derived peptides increase progressively with severity of HF, immunoassay methods
can greatly overestimate the true biological activity of CNH in patients with severe HF
\citep{bib36}. Unfortunately, at present, it is not possible to estimate the inaccuracy of CNH
immunoassays because these methods use different, not standardized antibodies and
calibrators, leading to highly different clinical results \citep{bib35} \citep{bib36} \citep{bib318} \citep{bib334} \citep{bib335}.

A resistance to the biological action of CNH may be theoretically due to an increase
in degradation (turnover) of circulating biologically active peptides. CNH are degraded in vivo and in vitro by several types of proteolytic enzymes, including serin-proteases, peptidyl arginine aldehyde proteases, kallikrein-like proteases, and neutral
endopeptidases (NEP) \citep{bib35} \citep{bib36} \citep{bib318} \citep{bib334} \citep{bib335} \citep{bib336} \citep{bib337} \citep{bib338}.

Individual differences in the ability of heart tissue to mature the precursor of CNH peptides, or of peripheral tissues to degrade them, may help to explain why there are some differences in the clinical presentation among patients with HF with similar clinical severity
and ventricular function \citep{bib36}.However,further studies are necessary to confirm this hypothesis.From a clinical point of view,it is important to note that some drugs sharing an inhibitory action on both NEP and ACE (so called vasopeptidase inhibitors) may have some beneficial effects in patients with arterial hypertension and/or HF because the administration of
these durgs can potentiate the biological activity of CNH system by increasing the concentration of biologically active peptides \citep{bib339} \citep{bib340} \citep{bib341} \citep{bib342} (see Chapter 7, sections 7.4 and 7.5, for more
details).It is important to note that renal function can affect the biological action of CNH in
different ways \citep{bib35} .CNH are small peptides freely filtrated
by renal glomerulus; the kidneys are probably responsible for about 50\% of metabolic clerance rate of plasma ANP and BNP and in this way renal diseases can affect the circulating levels of CNH.Indeed,a decreased renal function greatly increases the plasma CNH concentration
and consequently more peptide hormones are available for other target tissues (such as
brain,vascular tissue,adrenal gland and so on) \citep{bib35}.However,luminal perfusion with ANP has
been shown to reduce sodium efflux from the inner medullar collecting duct, suggesting
that this hormone has also luminal sites of action \citep{bib325}.As a consequence, a reduction in the
filtration can potentially induce renal hypo-responsiveness to CNH. To date, however,ANP
has been detected only on tubular basolateral membranes \citep{bib325}. Thus, the mechanisms of
CNH luminal action need to be elucidated before conclusions are drawn about the functional significance of reduced natriuretic peptide filtrations in the renal hypo-responsiveness to ANP and other natriuretic peptides.Finally,a resistance to the biological action of CNH
may be theoretically also due to an increased renal filtration (clearance) of active peptides;
at this moment, however, there are no consistent data to confirm this hypothesis.
Mechanisms acting at receptor level: Some studies suggest that the resistance to biological effects of CNH in HF may be due,in part,to variations in the relative amount of the three
different types of natriuretic peptide-specific receptors.In particular,there could be an upregulation of type C receptors (NPR-C) with a parallel down-regulation of type A and B receptors (NPR-A and NPR-B) \citep{bib343} \citep{bib344} \citep{bib345} \citep{bib346} \citep{bib347}. NPR-A and NPR-B mediate all known hormonal actions
of CNH, therefore their down-regulation should induce a deactivation of the CNH system.
The upregulation of NPR-C receptors that strongly contribute to the clearance of biologically active peptides could further increase the resistance to CNH in patients with HF \citep{bib343}.

These findings are well in accordance with the results of in vivo kinetic studies obtained
using radioactive tracers in patients with HF \citep{bib332} \citep{bib333}. Moreover, a recent study confirmed
that mRNA expression levels of ANP, BNP and the NPR-C receptor were markedly
increased in human failing hearts \citep{bib346}. Reversal of cardiomyocyte hypertrophy during
left ventricular assist device support was accompanied by normalization of ANP, BNP and
NPR-C mRNA levels and a significant recovery of responsiveness to ANP \citep{bib347}.

However, a very recent study \citep{bib348} found that neither NPR-A nor NPR-B were internalized or degraded in response to natriuretic peptide binding in 293T cultured cells.
Another well-characterized deactivation mechanism is the process by which an activated
receptor is turned off, commonly referred to as “desensitization” \citep{bib348} \citep{bib349} \citep{bib350}. Phosphorylation
of the intracellular kinase homology domain of NRP-A and NPR-B is required for hormone-dependent activation of the receptor, while dephosphorylation at this site causes desensitization. Deactivation of the CNH system via desensitization of NRP-A and
NPR-B can occur in response to various pathophysiological stimuli \citep{bib348} \citep{bib349} \citep{bib350} .
Further studies are necessary to clarify what is the most important mechanism of
deactivation of CNH system acting in vivo at receptor level in patients with heart failure: the down-regulation (of NPR-A and NPR-B), the upregulation (of NPR-C), or the
desensitization (of NPR-A and NPR-B).
A peripheral resistance to the biological effects of CNH may play an important role in
other clinical conditions, besides HF. For example, NPR-C is also present on cellular membranes of adipose tissue. It was suggested that the increase in NPR-C receptors observed
in obese subjects can in turn increase the peripheral degradation of CNH and consequently
blunt the action of the CNH system \citep{bib351} \citep{bib352}. Indeed, recent studies have documented
decreased circulating levels of CNH (especially BNP) in obese subjects, compared to nonobese subjects matched for age and gender \citep{bib351} \citep{bib354}. This reduced activity of the CNH system may increase the risk of developing arterial hypertension and other cardiovascular
diseases due to the non-contrasted and therefore prevailing effects of the counter-regulatory system with sodium-retentive and vasoconstrictive properties \citep{bib352} \citep{bib353} \citep{bib354} (see Chapter
6, section 6.9, and Chapter 7, section 7.5, for more detailed clinical information).
However, recent studies found that the NPR-C receptor could be coupled to a G-protein that inhibits cyclic AMP synthesis. These receptors, which are present in great
amount especially on the endothelial cell wall, may mediate some paracrine effects of
CNP on vascular tissue \citep{bib355} \citep{bib356} \citep{bib357}. Therefore, further studies will be necessary to elucidate the possible role of NPR-C receptors as modulators of CNH action and/or degradation in peripheral tissues.
Mechanisms acting at post-receptor level: A large number of studies demonstrated that
the activation of the neuro-hormonal system accelerates the left ventricular functional impairment in patients with HF \citep{bib31} \citep{bib32} \citep{bib33} \citep{bib34} \citep{bib35} \citep{bib36} \citep{bib358} \citep{bib359} \citep{bib360}. Drugs that contrast the detrimental
effects of the neuro-hormonal system activation play a key role for the current pharmacological treatment of HF. Some of these, such as ACE inhibitors, angiotensin II
receptor blockers, β-blockers, and spironolactone decrease the circulating levels of CNH
\citep{bib35} \citep{bib361} \citep{bib362} \citep{bib363} \citep{bib364} \citep{bib365},“normalize” their kinetics, and increase their biological activity \citep{bib324} \citep{bib326}. Furthermore, they enhance the natriuretic effect of ANP or BNP analogs administered to
patients. In other words, the treatment with this type of pharmacological agents decreases the systemic resistance to the biological effects of CNH \citep{bib324} \citep{bib326}.

Patients with HF show a progressive and parallel increase in CNH levels and in some
neuro-hormones and cytokines. This increase can be closely related to disease severity, as assessed by functional NYHA class (Table 3.1, Fig. 3.13) \citep{bib360}. Plasma BNP values,
normalized by mean values found in healthy subjects, are significantly higher than
other normalized neuro-hormone and cytokine values in HF (Fig. 5.3, Table 5.4) \citep{bib360};
these data well support the so-called “endocrine paradox” of HF \citep{bib36}. On average, the
response of the CNH system to the increasing challenge of disease severity may not be
linear (Fig. 5.4). The curve reported in Figure 5.4 suggests that the CNH system responds
with a sharp increase in BNP plasma concentration in the early phase of HF (NYHA
class I-II patients), followed, with the clinical progression of the disease, by a blunted
increase (NYHA class III), and finally by a plateau (NYHA class IV) (Table 3.1, Fig. 5.4).
These findings are consistent with the results from in vivo kinetic studies, indicating
that ANP turnover (i.e., metabolic clearance rate and production rate) and natriuresis
are both increased in patients in the early phase of HF (NYHA class I), as compared to
patients with congestive HF \citep{bib324} \citep{bib326}. This suggests that resistance to the biological
action of CNH is characteristic only of the congestive stage of HF. During the asymptomatic, early phase of the disease, the CNH system is able to compensate for the overactivity of the counter-regulatory system (Fig. 5.4).
In conclusion, several mechanisms occurring at the pre-receptorial, receptorial and
post-receptorial levels may play a role in inducing peripheral resistance to the biological
effects of CNH in HF. However, the overwhelming activation of the counter-regulatory
system should be considered the predominant pathophysiological mechanism. The data
discussed so far also suggest that monitoring the degree of systemic resistance to the biological effects of CNH could be clinically useful in the follow-up of patients with HF.

\subsubsection{ Diagnostic Accuracy of CNH Assay in Plasma from Patients with Cardiac Diseases}

The CNH system activation is modulated not only by hemodynamic factors, but also by
the activity of the counter-regulatory neuro-hormonal system. The response of the
CNH system to chronic pathophysiological stimuli may be log-like (Figs. 3.13 and 5.4).
Consequently, it is likely that very small changes in hemodynamics, not assessable by
echocardiographic examination, may produce significant (and measurable) variations
in plasma concentrations of CNH. Moreover, recent studies have indicated that very
small changes in some neuro-hormones and cytokines can produce wider variations in
BNP circulating levels \citep{bib360} (Fig. 5.3).
It is well known that changes in hemodynamic parameters (such as left ventricular
ejection fraction, EF) and plasma CNH levels (expressed in a log scale) are closely related in patients with cardiovascular diseases (Fig. 5.5) \citep{bib32} \citep{bib33} \citep{bib34} \citep{bib35} \citep{bib319} \citep{bib358} \citep{bib359} \citep{bib360} \citep{bib361} \citep{bib362} \citep{bib363} \citep{bib364} \citep{bib365}. However, correlations between plasma CNH levels and echocardiographically measured parameters,
such as left ventricular EF, myocardial mass and chamber volumes, are usually less close
in the general population (large community-based sample, including healthy subjects
with or without individuals with asymptomatic myocardial dysfunction) \citep{bib38} \citep{bib39} \citep{bib366} \citep{bib367},
as well shown dy data reported in Figure 5.6.
It should be emphasized that the recognition of HF syndrome is not equivalent to the
clinical diagnosis of cardiomyopathy or to the assessment of left ventricular dysfunction, these latter terms describing possible structural reasons for the development of HF
\citep{bib368} \citep{bib369} \citep{bib370}. Instead, HF is a clinical syndrome that is characterized by specific symptoms
(namely dyspnea and fatigue) and signs (namely fluid retention) \citep{bib368}. There is no diagnostic test for HF, because it is largely a clinical diagnosis based on a careful history
and physical examination \citep{bib368}. Therefore, there are no objective criteria for the identification and/or clinical stratification of patients with suspected HF; several “reference
(gold) standards” \citep{bib371}, based on clinical, laboratory and instrumental examinations,
can be used to evaluate the diagnostic accuracy of the CNH assay \citep{bib35}. For this evaluation, some studies took into consideration only echocardiographic assessment of ventricular dysfunction, while in others either results of echocardiography or all clinical data
have been used for the diagnosis of HF \citep{bib35} \citep{bib372}.

Unfortunately, using echocardiography as the unique reference standard may lead to
misinterpretations when evaluating the diagnostic accuracy of the CNH assay. A systematic review of clinical studies in patient populations with a prevalence of HF ranging from 3.8\% to 51\% to determine the diagnostic accuracy of BNP assays found that
the sensitivity (ranging from 90 to 97\%) was much less variable (by 4.4-fold) than the
specificity (ranging from 53 to 84\%) \citep{bib35}. Furthermore, in a meta-analysis \citep{bib372}, diagnostic accuracy of BNP assays greatly varied according not only to the group of patients
studied, but also to the reference standard used (left ventricular EF <40\% or <55\%,
diagnosis of diastolic dysfunction, diagnosis of systolic + diastolic dysfunction, or integrative clinical criteria). These data indicate that the choice of a suitable and accurate
reference standard for evaluation of the diagnostic accuracy of the BNP assay in patients
with HF may be a problem that is actually underestimated in the literature. For a proper definition of diagnostic accuracy, tested individuals should be grouped into those
with and without disease, by means of an independent clinical judgment, considering
both the CNH assay and echocardiographic data.
Careful echocardiographic examinations usually show slightly better or even similar diagnostic accuracy than the BNP assay in patients with cardiac diseases \citep{bib373} \citep{bib374}. On
the other hand, the CNH assay may have some advantages compared to echocardiographic examinations alone in specific clinical settings. For example, Williams et al.
suggested that in patients with chronic HF, the NT-proBNP assay reflects functional
cardiac impairment and decreased exercise capacity (measured by peak exercise oxygen consumption) better than the left ventricular EF \citep{bib375}.

Several studies indicate that BNP and NT-proBNP are powerful and independent
risk markers of cardiovascular events (especially mortality) not only in patients with HF,
but also in those with acute coronary syndrome, as reported in recent and systematic
reviews \citep{bib35} \citep{bib376} \citep{bib377} \citep{bib378}. Some studies also suggested that the cardiovascular risk increases progressively to CNH concentration \citep{bib377} \citep{bib378} \citep{bib379}; that is, there is no threshold that actually identifies patients with null risk.
Diagnostic sensitivity of BNP/NT-proBNP assays in detecting left ventricular systolic dysfunction could be suboptimal in asymptomatic or low-risk individuals, especially in women \citep{bib39}. Specificity of BNP assays in patients with HF ranged from 53 to
84\% and positive predictive values from 3 to 85\% in several studies \citep{bib35}. These data indicate that CNH assays can produce a relatively large number of false-positive results.
Consequently, many individuals, actually without HF (about 15-60\% of those with positive CNH tests), might undergo expensive and/or harmful investigations to rule out
the disease, or even be inappropriately labeled as cardiac patients \citep{bib35}.

Several false-positive results can be observed in patients with various clinical diagnoses, as reported in Table 5.1. In these patients, increased plasma BNP may predict,
even in the presence of a normal echocardiographic examination, an increased risk of
mortality or major cardiac events, including pulmonary embolism \citep{bib380} \citep{bib381} \citep{bib382} and hypertension \citep{bib383}, renal failure \citep{bib384} \citep{bib385}, septic shock \citep{bib386}, some chronic inflammatory diseases
(such as amyloidosis and sarcoidosis) \citep{bib387} \citep{bib388}, and diabetes mellitus \citep{bib389}.

On the other hand, false-negative results could be found in patients on treatment
with anti-adrenergic agents, diuretics and/or ACE inhibitors, all drugs that can reduce
CNH levels \citep{bib35}. As shown in Figure 5.7, a large number of patients with only mild HF
(NYHA classes I and II) may have values slightly above or even under the 99th percentile of distribution values of BNP concentration in healthy subjects (about 50 ng/l,
measured by an IRMA method). In these patients, successful treatment and consequent
improvement in cardiac function and exercise capacity, and reduction in filling pressure
and cardiac volumes, is usually associated with a marked fall in CNH levels: thus, a larger number of patients could have BNP values within the reference range values \citep{bib35} \citep{bib390}.

However, at matched echocardiographic alterations, patients in whom BNP levels drop
in response to therapy have a reduced rate of major cardiac events or mortality, compared to untreated hypertensive patients, who could have similar echocardiographic
abnormalities. This represents the rationale for using the CNH assay for therapy decision making and for monitoring HF patients \citep{bib35} \citep{bib361} \citep{bib362} \citep{bib363} \citep{bib364} \citep{bib365}.

In populations with a higher prevalence of cardiac diseases, including only individuals with a clinical suspicion of HF, the diagnostic sensitivity of BNP can improve up to
95\%, or even more, as long as appropriate cut-off values are selected \citep{bib35} \citep{bib372}. In this case,
a strategy called “SnNout”, which maximizes test sensitivity, could be used to rule out
the presence of HF \citep{bib391}. Furthermore, CNH assay also shows high (>95\%) negative predictive values \citep{bib35} \citep{bib320}, which can help to confirm the absence of HF. This is the rationale for choosing the CNH assay as the first step for an algorithm for the diagnosis of HF
\citep{bib369} \citep{bib370}. Such a clinical strategy has proved successful in some recent studies evaluating the cost-effectiveness of using plasma BNP measurements for screening of cardiac
dysfunction in the general population \citep{bib392} \citep{bib393} \citep{bib394}.


\subsubsection{ Biological Variation of Plasma BNP: a Problem or a Clinical Resource?}

The variability of measured plasma concentrations of many substances is due to three
different sources: pre-analytical, analytical and inherent biological variation. The latter is
usually described as a random variation around a homeostatic setting point, and defined
as the intra-individual or within-subject biological variation \citep{bib395}. Several physiological
parameters and endogenous substances are closely regulated by complex biological mechanisms in such a way as to vary little. If a random variation is assumed for an analyte,
the effect of imprecision on dispersion can be taken into consideration for setting generally
applicable quality specifications for assay performance in order to indicate an acceptable
(or desirable) degree of assay precision. Accordingly, the desirable assay imprecision
should be equal to or less than half of the intra-individual biological variation \citep{bib396} \citep{bib397}.

In order to achieve a correct interpretation of serial test results that are collected for
follow-up or tailored treatment of HF patients, several studies \citep{bib397} \citep{bib398} \citep{bib399} \citep{bib3100} \citep{bib3101} recently evaluated the biological variation of BNP and its related peptides, in both healthy subjects and
cardiac patients.
Due to secretory bursts and its rapid turnover (half-life about 15-20 min) \citep{bib31} \citep{bib3102}
, it is not surprising that intra-individual
biological variations of plasma BNP levels were found to be very large, in both healthy
subjects and patients with heart failure (ranging from 30 to 50\%) \citep{bib397} \citep{bib399} \citep{bib3100} \citep{bib3101}. Assuming a random variation around a homeostatic setting point, the calculated reference
change values at 95\% confidence interval ranged from 99 to 130\% for BNP in healthy subjects and patients \citep{bib397} \citep{bib399} \citep{bib3100} \citep{bib3101}. According to this estimated confidence interval, only a
decrease of more than 50\% or a more than 2-fold increase in plasma BNP should be
assumed to be statistically significant in an individual patient.
In contrast with this assumption, a recent clinical trial \citep{bib390} has suggested that a
BNP decrease inferior to this calculated reference change could be clinically relevant
in patients with heart failure. In this study, only the group of patients treated with the
β-blocker agent carvedilol, who respond on average with a decrease of only 38\% in
plasma BNP, showed a clinical improvement \citep{bib390}. Furthermore, several studies have
demonstrated that cardiovascular risk (mortality or major cardiovascular events)
increases continuously and progressively throughout the whole range of BNP concentrations in patients with cardiovascular diseases \citep{bib35} \citep{bib377} \citep{bib378} \citep{bib379}.

In order to explain these conflicting findings, it should be taken into account that
BNP secretion is closely regulated by specific pathophysiological mechanisms. Thus,
changes in plasma hormone levels cannot be interpreted as random variations around
a setting point, but as strictly determined by the activation level of the counter-regulatory system and by changes in hemodynamics. The clinician should look at the intra-individual variation in BNP as a mirror of variation in neuro-endocrine network activity.
According to this hypothesis, it was suggested to consider all changes in BNP concentration as potentially clinically relevant, even when narrower than the calculated
intra-individual biological variation \citep{bib3103}. In other words, BNP variations should be
interpreted and considered by physicians, as the variability of heart rhythm and blood
pressure, by taking into account clinical history and examination, comprehensive of the
response to specific treatments, as well as of laboratory and instrumental test findings.
There is another important practical consequence of this approach. According to an
intra-individual biological variation of about 30\%, the estimated imprecision goal for
BNP immunoassays should be 15\% (i.e., equal to half of the intra-individual variation)
\citep{bib395} \citep{bib396}. On the contrary, we suggest that all the measured variations of plasma BNP
greater than two to three times the analytical imprecision of assay used should be potentially considered as clinically relevant. Following this approach, the assay imprecision
of BNP immunoassays should be as low as possible.
Of course, the great number of pathophysiological mechanisms affecting the CNH
system makes it sometimes difficult for clinicians to recognize the cause(s) of variations
in its activity. However, we believe that the BNP assay should be considered as an intellectual spur for the search for an explanation of variations in hormone concentrations,
which should always be related to pathophysiological stimuli and/or pharmacological
interventions. In this sense, the assessment of BNP time-course concentrations is a novel,
meaningful diagnostic tool for the follow-up of patients with cardiac disease.

\subsection{ CNH Assay as Diagnostic and Prognostic Tool in Cardiac Diseases}

In this second part of the Chapter 5, we will review the clinical relevance of CNH assay
(especially of BNP and NT-proBNP). There has been an explosion of clinical researches and trials concerning the routinary use of CNH (especially BNP) assay in the diagnosis and risk stratification of patients with cardiovascular diseases in the first years of
the new century . It is clearly impossible to cover all these scientific contributes. In order to reduce the number of references
without reducing the efficacy and objectiveness, we have used systematic reviews and
meta-analyses, when available.
First, we will review the clinical relevance of CNH assay in the screening and classification of patients with cardiac dysfunction, divided according to the severity of disesae and age, or associated to some particular clinical conditions (such as myocardial
infarction or treatment with cardiotoxic drugs) . Moreover, we will discuss the diagnostic accuracy and clinical relevance of assay in coronary artery disease , where more recently BNP and NT-proBNP are increasingly frequently assayed with (almost in part) unexpected results. Finally, the diagnostic accuracy of CNH assay will be compared to that of other clinical tests, such as ECG,
echocardiogram and chest radiogram .
The last part of the Chapter is dedicated to prognostic relevance of CNH assay in
cardiac diseases and general population , to relevance of
measurement of CNH in tailoring the therapy , and to its cost-effectiveness  in patients with HF.

\subsubsection{ Use of CNH Assay in the Screening and Classification}

of Patients with Cardiac Dysfunction
The diagnosis of HF can often be difficult, mainly in primary care settings, where
patients may present with non-specific symptoms and signs, such as dyspnea, fatigue,
and ankle swelling \citep{bib368} \citep{bib369} \citep{bib370}. In several population-based studies, fewer than 40\% of
patients with a suspected diagnosis of HF in primary care had this diagnosis confirmed
by more specific and accurate clinical investigations, which are often expensive, timeconsuming and demanding for the patient \citep{bib368} \citep{bib369} \citep{bib370} \citep{bib3104} \citep{bib3105}. As a result, a relatively simple and inexpensive biochemical test (such as the CNH assay) may be very useful to
confirm the clinical suspicion of HF in this clinical setting \citep{bib35} \citep{bib334} \citep{bib335}.

The aim of the following paragraphs will be to discuss the diagnostic accuracy of
CNH assays (especially BNP/NT-proBNP assays) following the principal recommendations of evidence-based laboratory medicine principles \citep{bib371} \citep{bib3106} \citep{bib3107}.


\subsubsection{ Diagnostic Accuracy of CNH Assay in Asymptomatic,}

Mild Ventricular Systolic Dysfunction
Patients with asymptomatic left ventricular systolic dysfunction are likely to have lower
plasma BNP than those with overt HF \citep{bib32} \citep{bib33} \citep{bib34} \citep{bib35} \citep{bib334} \citep{bib335} \citep{bib362} \citep{bib363} \citep{bib364} \citep{bib365} \citep{bib366} \citep{bib372}, as shown in Fig. 3.13.
Two large studies \citep{bib39} \citep{bib366} evaluated the diagnostic accuracy of the CNH assay as a
screening method in a general population. The first study analyzed the Framingham
Heart Study cohort (3,177 individuals) using BNP and NT-proANP in the evaluation of
left ventricular hypertrophy and systolic dysfunction in a community population \citep{bib39}.

The presence of the disease was evaluated by using echocardiographic findings (the
prevalence of left ventricular systolic dysfunction was 9.3\% in the 1,470 men and 2.5\%
in the 1,707 women tested, respectively). The area under the curve (AUC) of receiver
operating characteristic (ROC) analysis for CNH assay for identifying both left ventricular hypertrophy and systolic dysfunction was on average about 0.75, with a good specificity (assumed 95\% both for men and women) and negative predictive value (NPV, on
average ranging from 92\% to 97\% in men, and from 91\% to 98\% in women), but a poor
sensitivity (i.e., ranging from 27\% to 28\% in men, and from 13\% to 40\% in women) and
positive predictive value (PPV, from 22\% to 38\% in men, and from 5\% to 40\% in women),
using gender-related BNP cut-off values \citep{bib39}.

The aim of the second study was to examine the validity of plasma BNP measurement
(with the same IRMA method as the other study) for detection of various cardiac abnormalities in a rural Japanese population (1,098 subjects, 693 men and 405 women), with
a low prevalence of coronary heart disease and left ventricular systolic dysfunction
(i.e., only 37 participants, corresponding to 3.0\%, showed an EF <30\%) \citep{bib366}. The diagnosis was carried out by two independent cardiologists based on a medical questionnaire, chest radiogram, electrocardiogram (ECG), and echocardiographic report. The optimal threshold for identification of disease was a BNP concentration of 50 ng/l (14.4
pmol/l), with an area under the ROC curve of 0.970, a sensitivity of 89.7\%, a specificity of 95.7\%, PPV of 44.3\%, and NPV of 99.6\%, respectively.
The conclusions of these two studies, though similar in aim, as well as in clinical and
experimental protocols, were strongly conflicting. The Japanese study suggested that the
BNP assay is a very efficient and cost-effective mass screening technique for identifying
patients with various cardiac abnormalities regardless of etiology and degree of left ventricular
systolic dysfunction \citep{bib366}, while the Framingham study suggested only limited usefulness of
the CNH assay as a mass screening tool for this clinical condition, especially in women \citep{bib39}.

This discrepancy may be due to the different gold standard used for the diagnosis of
heart failure adopted in the two studies, as discussed in a previous paragraph . However, these two studies, taken as a whole, indicate that the CNH assay may
have only a limited usefulness as a screening method for HF in a general population,
owing to the poor sensitivity and PPV. However, both studies also found good specificity
and NPV, thus suggesting that the CNH assay may be used to rule out HF in an asymptomatic (or pauci-symptomatic) individual.

\subsubsection{ Diagnostic Accuracy of CNH Assay in Patients with Suspected HF}

Many studies \citep{bib392} \citep{bib3108} \citep{bib3109} \citep{bib3110} \citep{bib3111} \citep{bib3112} \citep{bib3113} \citep{bib3114} \citep{bib3115} \citep{bib3116} \citep{bib3117} \citep{bib3118} \citep{bib3119} \citep{bib3120} \citep{bib3121} \citep{bib3121} \citep{bib3123} \citep{bib3124} \citep{bib3125} \citep{bib3126} \citep{bib3127} \citep{bib3128} have suggested that the CNH assay could be useful as a
screening method and/or for the differential diagnosis in patients suspected of HF in
different clinical settings: a) randomly selected high-risk community populations; b) primary-care patients with a new diagnosis of HF; c) patients with acute dyspnea in the
emergency department; d) consecutive unselected hospital inpatients; e) patients admitted to the intensive care unit.
The studies concerning the diagnostic accuracy of the CNH assay in patients with HF
have also been analyzed by two recent systematic reviews \citep{bib35} \citep{bib372}; unfortunately, these
studies showed quite heterogeneous data, thus introducing some bias in the statistical
analysis. In particular, even some high-quality studies were not designed with the primary goal of evaluating the diagnostic accuracy of the CNH assay. Indeed, this aim was
considered only at a post-hoc analysis and retrospectively assessed in blood samples collected for different original purposes, some years before the actual evaluation of diagnostic accuracy. This may introduce a significant bias, although its true clinical relevance is difficult to assess \citep{bib35}.

The second most important cause of heterogeneous data is the different reference
(gold) standard used to evaluate the diagnostic accuracy of the CNH assay. In some
studies, the patients studied were stratified and grouped according to clinical severity,
as described by functional classification (usually NYHA classification). In other studies, only echocardiographic measurements were used as “gold standard” for the accuracy of the CNH assay for the diagnosis of left ventricular dysfunction (and not for the
clinical diagnosis of HF) \citep{bib35} \citep{bib372}.

Furthermore, comparison of studies concerning the diagnostic accuracy of CNH
assay is also difficult because different populations were enrolled and different immunoassays were used. Indeed, diagnostic accuracy (especially predictive values) is strictly
dependent on disease prevalence (pre-test probability), which greatly varies according
to the clinical setting considered (i.e., screening for general population, out-patients
seen by a general practitioner, or in primary care, emergency department, coronary
care unit, and so on). In particular, the prevalence of HF in the populations studied varied from less than 5\% (in studies of screening of asymptomatic population) to more
than 50\% (in studies including patients referred to hospital with suspicion of HF) \citep{bib35}.

Another factor, often underestimated, is that the “gold standard” (which is not an objective rule, but a clinical synthesis or another diagnostic test) could vary with the disease
prevalence, sometimes in a different manner than the CNH assay.
Finally, recent data reported in the literature suggested that diagnostic accuracy may
vary significantly in relation with the specific cardiac peptide measured and/or
immunoassay used \citep{bib35} \citep{bib319} \citep{bib321} \citep{bib372} \citep{bib3129} \citep{bib3130} \citep{bib3131}. At the present time, the different BNP
immunoassays also show greatly different imprecision \citep{bib3132}. Consequently, it is not
clear whether the observed significant variation in diagnostic accuracy is due to a difference in pathophysiological behavior of measured peptides and/or in assay performance \citep{bib35}. Unfortunately, some studies do not clearly indicate the type of immunoassay used to measure CNH, while the majority do not report the assay performance (and
often even the reference values) evaluated in their own laboratory.
Taking all studies as a whole, a recent meta-analysis showed that the odds ratio for
diagnostic accuracy of BNP assay in different groups of patients with suspected HF is
highly significant \citep{bib372}. In particular, the pooled diagnostic odds ratio, when clinical
criteria were used as gold standard for HF, was 30.9 (95\% confidence interval 27.0-35.4),
while it fell to 11.9 (8.4-16.1) when a value ≤40\% of left ventricular EF, estimated by
echocardiography, was used as reference standard \citep{bib372}.

Diastolic dyfunction can play a major role in determining signs and symptoms in
patients with congestive HF \citep{bib365} \citep{bib3120} \citep{bib3121}. Although Doppler echocardiography is currently used to examine the left ventricular diastolic filling dynamics, the limitations of
this technique suggest the need for other objective measures \citep{bib3122}. Despite these limitations, several studies indicated that the CNH assay, in particular the BNP assay, may
be useful for the diagnosis of left ventricular diastolic dysfunction \citep{bib3123} \citep{bib3125} \citep{bib3126}. However, some conflicting results have been reported, probably due to the different cause of
and/or mechanisms responsible for cardiac dysfunction \citep{bib3127} \citep{bib3128}. In a recent metaanalysis, including data of three studies comparing the diagnostic accuracy of BNP
assay in patients with diastolic dysfunction, the pooled odds ratio was 28.3 (95\% confidence interval 2.66-300.5) \citep{bib372}. However, a bias may affect these studies, as suggested
by the significant test of heterogeneity (c 2 = 128.4, p < 0.001) \citep{bib372}.

Finally,Wright et al. \citep{bib3133} evaluated the effect of NT-proBNP assay on the clinical diagnostic accuracy of HF in primary care by means of a prospective, randomized controlled trial in 305 patients. Each patient was randomized in two groups, one in which
the general practioner had at their disposal the NT-proBNP assay results (NT-proBNP
assay group), while the other did not (control group). The diagnostic accuracy improved
by 21\% in the NT-proBNP assay group and by 8\% in the control group (p = 0.002). This
study indicates that NT-proBNP measurement significantly improves the clinical diagnostic accuracy of HF in general practice \citep{bib3133}.


\subsubsection{ Diagnostic Accuracy of CNH Assay in Patients with Acute Myocardial Infarction}

Circulating levels of CNH increase after acute myocardial infarction (AMI); the extent
of the increase is related to the size of the infarct \citep{bib3134} \citep{bib3135} \citep{bib3136} \citep{bib3137}. Patients with smaller infarcts
tend to have a monophasic increase in plasma BNP, peaking at 20 hours after the onset
of symptoms; on the other hand, those with larger infarcts, lower EF, and clinical signs
of HF may present a further peak at 5 days after admission \citep{bib3135}. Other studies are less
convincing regarding the ability of the CNH assay to identify patients with significant
myocardial dysfunction after AMI \citep{bib3138} \citep{bib3139}. These conflicting results could be due to
the differences in sample collection time, type of CNH (ANP, BNP, or NT-proBNP) measured, type of assay (competitive vs non-competitive), and inclusion criteria adopted. However, persisting elevation of CNH levels at 1 or 2 months after AMI usually suggests a high
risk of adverse remodeling and subsequent HF \citep{bib35}, although this finding should be confirmed by further specifically addressed studies.
The diagnostic accuracy of the BNP assay in patients with myocardial infarction
was evaluated in a recent meta-analysis \citep{bib372} taking into account only two studies
\citep{bib3140} \citep{bib3141}, fitting the inclusion criteria of this analysis; the pooled odds ratio was 9.4
(95\% confidence interval 4.5-19.4).

\subsubsection{ Diagnostic Accuracy of CNH Assay in Elderly People}

Heart failure is primarily a disease of old age; chronic HF increases in prevalence with
aging from <1\% in people aged <65 years to >5\% in those >65 years of age, and this
clinical condition is the first cause of morbidity and mortality in older people \citep{bib368} \citep{bib369} \citep{bib370} \citep{bib3142}. A recent study demonstrated that elderly patients present with more advanced HF,
as evidenced by their higher morbidity and mortality rate along with greater neurohormonal activation \citep{bib3142}. According to these findings, elderly people should be considered to be a population with high risk for developing HF and so the BNP/NT-proBNP assay may be useful as a screening test for HF in older age. Indeed, several studies
reported that the BNP/NT-proBNP assay could be clinically useful in elderly people
suspected to have HF \citep{bib316} \citep{bib3142} \citep{bib3143} \citep{bib3144} \citep{bib3145} \citep{bib3146} \citep{bib3147}.

Two studies compared the diagnostic accuracy of the BNP assay and that of ECG in
elderly people screened for HF. A prospective cohort study specifically evaluated the
diagnostic accuracy for HF of BNP assay in 299 consecutive patients (mean age 79 years,
65\% women) attending day-hospital over a period of 13 months \citep{bib3143}. This study suggested that both BNP assay and ECG were sensitive in detecting left ventricular systolic
dysfunction, but lacked specificity (but the combination of the two tests improved diagnostic accuracy) \citep{bib3143}. The other study reported that both the ECG and the plasma concentration of BNP were highly efficient in excluding left ventricular systolic dysfunction in 407 75-year-old subjects \citep{bib3144}. However, compared with the BNP, the ECG yields
a lower number of false-positive cases. Therefore, this study suggested that in screening for left ventricular systolic dysfunction in elderly people, the BNP assay has a diagnostic value in addition to the ECG, but only in individuals with abnormal ECG \citep{bib3144}.

Another study indicated that the BNP assay may be particularly useful in ederly
patients, especially in differentiating cardiogenic pulmonary edema from respiratory
causes of dyspnea \citep{bib3146}. Screening of populations with more than 1\% prevalence of
HF (such as people with age more than 60 years) with BNP followed by echocardiography
should provide a health benefit at a cost that is comparable to or less than other accepted health interventions \citep{bib3145}. Finally, the NT-proBNP assay was demonstrated to be
useful for detecting HF among people living in elderly nursing homes \citep{bib3147}.


\subsubsection{ Detection of Drug Cardiotoxicity by means of BNP Assay}

Another example of the clinical relevance of BNP assay is the possibility of identifying
HF caused by drug cardiotoxicity \citep{bib3148} \citep{bib3149} \citep{bib3150} \citep{bib3151} \citep{bib3152} \citep{bib3153}. Cardiotoxicity is a potential side-effect of
some chemotherapeutic agents \citep{bib3154}. The anthracycline class of cytotoxic antibiotics are
the most famous, but other chemotherapeutic agents can also cause serious cardiotoxicity and are not so well recognized (including cyclophosphamide, ifosfamide, mitomycin and fluorouracil) \citep{bib3154}. In a large retrospective study of more than 4,000 patients,
an incidence of clinical chronic HF of 2.2\% was found \citep{bib3155}. However, a more recent study
suggests that about 20\% of patients treated with anthracycline show a decrease in ventricular function below the normal limits, if followed with serial measurements of left
ventricular EF, assessed by an accurate radionuclide method \citep{bib3156}. This supports the recommendations of monitoring cardiac function in anthracycline-treated patients with
accurate and suitable methods.
Experimental studies in rats indicated that plasma BNP and serum cardiac troponin
T (cTnT) significantly increased from 6 to 12 weeks after adriamycin administration with
deterioration of cardiac function, as assessed by echocardiography \citep{bib3157}. However, in
this study the increase in cTnT was antecedent to the increase in BNP and the deterioration of cardiac function \citep{bib3157}.

Several studies demonstrated that the release pattern of cardiac-specific troponin I
(cTnI) after high-dose chemotherapy is a sensitive and reliable marker of acute minor
myocardial damage and is able to identify patients at different risks of cardiac events
3 years later \citep{bib3158} \citep{bib3159} \citep{bib3160}. On the other hand, several studies demonstrated that the BNP
assay may also be a marker of chemotherapy cardiotoxicity \citep{bib3148} \citep{bib3149} \citep{bib3150} \citep{bib3151} \citep{bib3152}. In particular,
two recent studies \citep{bib3161} \citep{bib3162} also suggested that BNP/NT-proBNP assay is a predictive
marker of cardiac dysfunction in patients affected by aggressive malignancies and treated with high-dose chemotherapy. The acute release of circulating levels of troponin
should be only a mirror of the death of myocardiocytes, while the persistent increase in
BNP, after several days or weeks from the administration of cardiotoxic drug, should be
specifically related to ventricular remodeling and myocardial dysfunction. Further studies are necessary to confirm these data, and to evaluate the respective diagnostic accuracy and clinical relevance of troponin and BNP assays in the follow-up of patients
treated with potentially cardiotoxic drugs.

\subsubsection{ Diagnostic Accuracy of CNH Assay in Coronary Artery Disease}

Two very recent studies suggested that the BNP assay could be a reliable marker of
ischemia in patients with coronary artery disease \citep{bib3163} \citep{bib3164}. Exercise electrocardiography
is the most widely used non-invasive method to detect the presence of coronary artery
disease; however, its usefulness is limited by relatively modest sensitivity and specificity \citep{bib3165} \citep{bib3166} \citep{bib3167}. Other more accurate non-invasive tests, such as exercise echocardiography and exercise testing with radionuclide imaging, are less widely available and considerably more expensive.
It could be hypothesized that exercise-induced ischemia results in increased wall
stress and triggers release of CNH from myocardiocytes. According to this hypothesis,
Foote et al. \citep{bib3163} measured NT-proBNP and BNP in blood samples from a group of
normal volunteers, and two groups of patients, one with and the other without coronary
artery disease, before and after maximal exercise. Post-exercise increases in NT-proBNP and BNP were approximately 4-fold higher in the ischemic group than in the nonischemic group; while in volunteers, the increase was almost identical to that of the
non-ischemic patient group. At equal specificity to the ECG (58.8\%), the sensitivities
of the BNP/NT-proBNP assay in detecting ischemia were 90 and 80\%, respectively; in
contrast, the sensitivity of the exercise ECG was only 37.5\%.
In the study by Sabatine et al. \citep{bib3164}, transient myocardial ischemia was associated with
an immediate rise in circulating BNP levels, and the magnitude of the rise was proportional
to the severity of ischemia. These findings demonstrate an important link between the
severity of an acute ischemic insult and the circulating levels of BNP. However, further
studies are necessary to evaluate the relevance of the BNP/NT-proBNP assay.

\subsection{ Comparison between the Diagnostic Accuracy}

of CNH Assay and that of Other Tests and Clinical Investigations
Signs and symptoms correlate poorly with the presence of HF, for this reason diagnosis relies on clinical judgment based on a combination of history, physical examination
and appropriate investigation \citep{bib365} \citep{bib369} \citep{bib370} \citep{bib3168}.

Davie et al. \citep{bib3169} found that left ventricular systolic dysfunction was virtually never present if the ECG was normal (sensitivity 94\%, NPV 98\%), and a screening ECG reduced the
need for echocardiograms by 50\%. However, several studies demonstrate that CNH assay
can improve the diagnostic accuracy of history, ECG, and chest radiography \citep{bib3167} \citep{bib3168} \citep{bib3169} \citep{bib3170} \citep{bib3171} \citep{bib3172} \citep{bib3173} \citep{bib3174}.

Cowie et al. \citep{bib3110} reported that ROC curves for BNP (AUC = 0.96), ANP (0.93), and
NT-proANP (0.89) were better than that for cardiothoracic ratio on chest radiogram
(0.79) in screening for patients likely to have HF and requiring further clinical assessment. Nielsen et al. \citep{bib392} found that BNP assay showed a diagnostic accuracy better than
ECG in a random sample of 1,257 community subjects.Another study suggested that BNP
assay together with the presence of major ECG abnormalities and history reduced by
a factor of six (in comparison to consideration of BNP assay in isolation) the number
of subjects requiring echocardiography to detect one case of myocardial dysfunction in
a large population screening (1,360 patients tested) \citep{bib3171}.

Several studies have compared the diagnostic accuracy of BNP assay and ECG in
the elderly population. In 75-year-old subjects both the ECG and the plasma concentration
of BNP are highly efficient in excluding left ventricular systolic dysfunction, as recently suggested \citep{bib3173}. In another study, several types of structural heart disease, in particular
valvular heart disease, could be identified exclusively by BNP testing, suggesting that BNP
measurement can make a significant contribution to screening for CHF precursors
when used in combination with ECG in elderly populations (856 subjects enrolled, with
age ≥65 years) \citep{bib3174}. However, compared with the BNP assay, the ECG yielded a lower
number of false-positive cases in another study \citep{bib3172}. In screening for left ventricular
systolic dysfunction, the BNP has a diagnostic value in addition to the ECG, but only in
individuals with abnormal ECG \citep{bib3172}.

NT-proBNP alone was a better predictor of left ventricular dysfunction than any
other single or combination of factors, while the ECG had a poor predictive value for left
ventricular systolic dysfunction in identifying patients with left ventricular systolic
dysfunction in a high-risk population (243 patients, 129 men, median age 73 years,
range 20-94) \citep{bib3173}.

Another study \citep{bib3170} compared the diagnostic accuracy of BNP assay, ECG, chest
radiography and echocardiography; the results of this study confirmed that sensitivity, specificity and accuracy of BNP assay and echocardiography were significantly better than those of ECG and chest radiography \citep{bib3170}.

A huge number of studies have compared the diagnostic accuracy of BNP assay and
echocardiography in patients with acute or chronic HF \citep{bib35} \citep{bib362} \citep{bib363} \citep{bib364} \citep{bib365} \citep{bib372} \citep{bib3175}. According to
the guidelines for the diagnosis of heart failure proposed by the Task Force on Heart Failure of the European Society of Cardiology \citep{bib369}, echocardiography is recommended as
the most practical tool to demonstrate cardiac dysfunction. However, a recent epidemiological study \citep{bib3176} suggested that the diagnosis of HF might be better defined
in terms of symptoms, elevated neuro-hormones and impaired cardiac workload.
In more than 50\% of the clinical studies, the echocardiographic investigation has
been considered as the gold standard (more exactly the reference standard method)
for the evaluation of diagnostic accuracy of CNH assay in patients with heart failure
\citep{bib35} \citep{bib372}.When we use the echocardiography investigation as reference method, we assume
that BNP assay cannot theoretically have a better diagnostic accuracy than it. Indeed,
conflicting results were found when different clinical settings (patient’s selection) and/or
gold standard were used \citep{bib35} \citep{bib372}. Choy et al. \citep{bib3177} showed that in post-AMI patients
plasma BNP is superior to all clinical indices of left ventricular systolic dysfunction
(EF <40\%), including signs and symptoms and a clinical score (Peel Index). Dokainish
et al. \citep{bib3178} found that BNP assay and comprehensive Doppler echocardiography have
similar diagnostic accuracy for congestive HF, although echo-Doppler trended toward
higher specificity than BNP for congestive HF. Moreover, serial BNP measurements
during the treatment of acute HF provide incremental prognostic information over
clinical presentation and repetitive echocardiographic examination \citep{bib3179}. Mak et al.
\citep{bib3180} suggested that both BNP assay and a complete echocardiographic examination do
not have adequate discriminatory power to be used in isolation in the evaluation of left
ventricular diastolic function. Therefore, all available information, including systolic
function, chamber dimensions and all Doppler variables must be considered in the
analysis of individual patients \citep{bib3180}. Steg et al. \citep{bib3181} recently assessed the diagnostic performance of BNP testing and echocardiographic assessment of left ventricular systolic
function, separately and combined, for the identification of congestive HF in 1,586
patients presenting to the emergency department with acute dyspnea. The proportions
of patients who were correctly classified were 67\% for BNP alone, 55\% for EF alone,
82\% for the two variables together, and 97.3\% when clinical, ECG, and chest radiograph
data were added. This study suggested that BNP measurement was superior to twodimensional echocardiographic determination of left ventricular EF in identifying congestive HF, regardless of the threshold value. The two methods combined have marked
additive diagnostic value \citep{bib3181}.

Some recent studies indicated that BNP/NT-proBNP levels do not accurately predict serial hemodynamic changes and consequently do not obviate the need for pulmonary artery catheterization in patients requiring invasive hemodynamic monitoring
\citep{bib3182} \citep{bib3183} \citep{bib3184}.

In conclusion, BNP assay shows significantly higher predictive characteristics than
ECG and chest radiography, and a cost-benefit value significantly greater than that of
echocardiography \citep{bib392} \citep{bib3170}. BNP measurement may exclude normal heart with high
probability owing to its high degree of sensitivity and NPV when used in screening
high-risk populations, therefore reducing the echocardiographic diagnostic burden;
this is the rationale for considering the BNP assay in the first step of an algorithm for
the differential diagnosis of heart failure \citep{bib337} \citep{bib365} \citep{bib369} \citep{bib370} \citep{bib3168} \citep{bib3175}.


\subsection{ Use of CNH Assay as Prognostic Marker in Cardiovascular Diseases}

Several well-designed and conducted studies suggested that the CNH assay may be useful as a prognostic marker mainly in two clinical conditions: HF and acute coronary
artery syndromes (ACS), as also demonstrated by systematic reviews \citep{bib35} \citep{bib376} \citep{bib3184} \citep{bib3185}.


\subsubsection{ Prognosis in HF}

The prognostic role of CNH assay (especially NT-proANP, BNP and NT-proBNP) in
patients with HF is well demonstrated by a huge number of studies \citep{bib316} \citep{bib3147} \citep{bib3186} \citep{bib3187} \citep{bib3188} \citep{bib3189} \citep{bib3190} \citep{bib3191} \citep{bib3192} \citep{bib3193} \citep{bib3194} \citep{bib3195} \citep{bib3196} \citep{bib3197} \citep{bib3198} \citep{bib3199}.

In all these studies, CNH concentrations were always found to be independent risk
markers for morbidity (increased future major cardiovascular events and/or hospitalization) and/or mortality in patients with acute or chronic HF \citep{bib35}. In some studies CNH
levels were stronger predictors of mortality and/or major cardiovascular events than left
ventricular EF, NYHA class, and/or presence of diabetes or hypertension, as well as sex
and age in patients with chronic HF \citep{bib3188} \citep{bib3189} \citep{bib3192} \citep{bib3194} \citep{bib3195} \citep{bib3196} \citep{bib3198} \citep{bib3199}. In some recent studies \citep{bib3197} \citep{bib3199}, NT-proANP was shown to be a more powerful predictor than BNP and
NT-proBNP, but these differences may be due to the analytical performance of methods
used rather than a true specific characteristic of peptide measured in the study. Further
studies are necessary to clarify whether there is a true difference among the NT-proANP,
BNP and NT-proBNP as predictor markers in patients with HF.
A continuous relationship was generally found between BNP levels and mortality
\citep{bib3194}, thus suggesting that it is not possible to observe a threshold for the risk. On average, a systematic analysis of the most important studies suggested an odds ratio of
about 2 for the risk of mortality in patients with BNP values above the cut-off \citep{bib35}.

Further evaluation needs the clinical relevance of combined use of CNH assay with
other functional parameters (such as peak oxygen consumption) \citep{bib390} \citep{bib3201} \citep{bib3202}, neurohormones \citep{bib3203} \citep{bib3204} \citep{bib3205} \citep{bib3206}, and markers of fibrinolysis \citep{bib3207} or cardiac necrosis \citep{bib3208} \citep{bib3209}.

Some studies suggest that the combination of BNP assay with the assessment of peak
oxygen consumption \citep{bib3202} or with specific markers of cardiac necrosis \citep{bib3209} improves
risk stratification of patients with HF.
Sereral studies reported that some neuro-effectors, hormones (including peptide,
thyroid and steroid hormones), and cytokines are increased in patients with HF, as
recently reviewed \citep{bib32} \citep{bib35} \citep{bib3210} \citep{bib3211} \citep{bib3212} \citep{bib3213} \citep{bib3214} \citep{bib3215} \citep{bib3216} \citep{bib3217} \citep{bib3218} \citep{bib3219} \citep{bib3220} . In general, BNP assay was found to be the most powerful indicator
for poor outcome compared to other neuro-effectors and hormones, at least in the
largest studies \citep{bib35} \citep{bib3204} \citep{bib3206}. However, more data on the clinical relevance of combination of two or more neuro-hormones in risk stratification of HF are necessary.

\subsubsection{ Prognosis in ACS}

Acute coronary artery syndrome encompasses a continuum of cardiac ischemic events,
ranging from unstable angina pectoris, with no biochemical evidence of myocardial
necrosis, to ST-elevation AMI \citep{bib3221} \citep{bib3222}. The prognosis of patients with ACS varies
widely, and several clinical, electrocardiographic, and biochemical markers have been
used to identify high-risk individuals in need of aggressive intervention \citep{bib3222}.

Several studies reported that CNH assay (in particular BNP and NT-proBNP) provides valuable prognostic information in patients with ACS \citep{bib3117} \citep{bib3138} \citep{bib3223} \citep{bib3224} \citep{bib3225} \citep{bib3226} \citep{bib3227} \citep{bib3228} \citep{bib3229} \citep{bib3230} \citep{bib3231}. A recent
meta-analysis confirmed the powerful prognostic value of BNP/NT-proBNP in patients
with ACS for death both in the short term (<50 days, mean odds ratio 3.38, CI 95\% 2.44-4.68) and long term (>10 months, mean odds ratio 4.31, 3.77-4.94) \citep{bib376}.

As for the prognostic role of BNP assay in patients with HF, more data on the clinical
relevance of combination with other markers in risk stratification of HF are necessary.
However, some studies suggest that BNP levels improve a simple risk score in patients with
unstable angina and non-ST-elevation myocardial infarction \citep{bib3239} as well as the riskassessment performance obtained with the cTnI and hs-CRP assays in patients with STsegment elevation myocardial infarction \citep{bib3234}. These data suggest the utility of a “multimarker strategy” or biomarker profile to characterize patients with ACS \citep{bib378}.

A recent study \citep{bib3240} reported that there may be clear differences among the profiles of individual natriuretic peptide levels in the 2 years after AMI. Similar profiles
were found with BNP and NT-proBNP plasma concetrations, while those of NT-proANP
and CNP differed significantly in a cohort of 236 patients with AMI complicated by
clinical, radiologic or echocardiographic evidence of left ventricular dysfunction. Moreover, a single measurement of plasma natriuretic peptide levels during the hospital
admission provides limited prognostic information, while NT-proBNP measured by an
ELISA method in the 30 days after AMI identifies a cohort of patients at increased risk
of adverse outcome thereafter \citep{bib3240}.

Two recent studies reported that in patients with clinically stable, angiographically
documented coronary artery disease, plasma BNP \citep{bib3241} or NT-proBNP \citep{bib3242} levels are
independently related to long-term survival in a multivariate model. These studies suggested that BNP and NT-proBNP are markers of long-term mortality even in patients
with stable coronary disease and add prognostic information above and beyond that provided by conventional cardiovascular risk factors and the degree of left ventricular systolic dysfunction \citep{bib3241} \citep{bib3242}.

In order to explain these clinical findings, it is important to note that experimental
studies in animals reported that myocardial ischemia or even hypoxia per se could
induce the synthesis/secretion of CNH (in particular BNP) from the intact heart in vivo
as well as ventricular cells in culture \citep{bib3243} \citep{bib3244}. Furthermore, these experimental data
are also in accordance with recent clinical studies indicating that transient myocardial
ischemia in patients with stable coronary artery disease is associated with an immediate rise in circulating BNP levels, and that the magnitude of rise is proportional to the
severity of ischemia \citep{bib3163} \citep{bib3164} \citep{bib3246}.


\subsubsection{ Prognostic Relevance of CNH Assay in the General Population}

While measurement of plasma BNP concentration has been shown to be useful in the
diagnosis of HF (especially acute HF), its role as a screening test for detection of preclinical ventricular remodeling or dysfunction in the general population has not been
established \citep{bib35} \citep{bib3246} \citep{bib3247}. However, some studies evaluated the prognostic relevance of
CNH assay in the general population, especially in high-risk populations, such as elderly people \citep{bib316} \citep{bib3187} \citep{bib3224} \citep{bib3244} \citep{bib3245} \citep{bib3246} \citep{bib3247} \citep{bib3248}. These studies demonstrated that CNH (and especially BNP/NT-proBNP) levels are a sensitive and accurate biochemical marker of an
increased risk of cardiac morbidity and total mortality in very elderly persons \citep{bib316} \citep{bib3147} \citep{bib3188} \citep{bib3227} \citep{bib3248} \citep{bib3249}. However, in a community-based study (mean age 56 years) there
were no significant trends of increasing incidence of hypertension across BNP categories in men or women, while higher plasma BNP levels were associated with increased
risk of BP progression in men but not women \citep{bib3250}. The differences between the results
of this study \citep{bib3250} and those of others \citep{bib316} \citep{bib3185} \citep{bib3227} \citep{bib3248} \citep{bib3249} may be due to the different age and gender of subjects studied. Probably, the prognostic power of BNP assay
increases progressively with age and may be gender-dependent.
Additional investigations are warranted to confirm the powerful prognostic value of
BNP assay in elderly people and to elucidate the basis for these gender-related differences \citep{bib3246} \citep{bib3247}.


\subsection{ CNH Assay in Management of Patients with HF}


\subsubsection{ Clinical Relevance of CNH Assay in Tailoring the Therapy of HF}

Medical therapy for HF is based on improving the symptoms and signs of fluid retention (change in dyspnea, edemas, and body weight are the usual markers of response to
treatment) and titrating the dosage of drugs (such as diuretics, ACE inhibitors, β-blockers, and spironolactone) following the evidence from randomized clinical trials \citep{bib365} \citep{bib368} \citep{bib369} \citep{bib370}. Currently, there is no specific surrogate end-point for treating patients with HF
that can be used to fine-tune therapy \citep{bib365} \citep{bib368} \citep{bib369} \citep{bib370}.

Many authors have suggested that the results of CNH assay (especially BNP/NTproBNP assay) may be useful in monitoring and tailoring the medical therapy in HF
patients, and in providing a practical objective indicator of optimal therapy \citep{bib361} \citep{bib362} \citep{bib363} \citep{bib364} \citep{bib365} \citep{bib368} \citep{bib369} \citep{bib370} \citep{bib385} \citep{bib390} \citep{bib3168} \citep{bib3252} \citep{bib3253} \citep{bib3254} \citep{bib3255} \citep{bib3256} \citep{bib3257} \citep{bib3258} \citep{bib3259} \citep{bib3260} \citep{bib3261} \citep{bib3262} \citep{bib3263} \citep{bib3264} \citep{bib3265}, including patients subjected to cardiac transplantation \citep{bib3266}.

CNH usually respond to effective treatment with drugs \citep{bib35} \citep{bib361} \citep{bib362} \citep{bib363} \citep{bib364} \citep{bib365} or left ventricular
assist device \citep{bib3267} \citep{bib3268} with a prompt reduction of their circulating levels. Indeed, ACE
inhibitors, valsartan, diuretics, nitrates, and endothelin receptor antagonists have been
shown to reduce plasma CNH levels in parallel with hemodynamic and clinical improvement \citep{bib362} \citep{bib363} \citep{bib3252} \citep{bib3258} \citep{bib3271} \citep{bib3272} \citep{bib3273} \citep{bib3274} \citep{bib3275} \citep{bib3276} \citep{bib3277}.

More variable effects on plasma CNH levels have been reported after therapy with
β-blockers \citep{bib390} \citep{bib3278} \citep{bib3279} \citep{bib3280} \citep{bib3281} \citep{bib3282} \citep{bib3283} \citep{bib3284} \citep{bib3285} \citep{bib3286} \citep{bib3287} \citep{bib3288} \citep{bib3289} \citep{bib3290} \citep{bib3291} \citep{bib3292}. Some authors suggested that these variable effects may be at
least in part attributable to different specificities or to ancillary properties of β-blockers \citep{bib362}. Ohta et al. \citep{bib3293} reported that both high and low doses of carvedilol have the
effect of increasing plasma ANP and BNP levels in rats. This effect was closely related
to the upregulation of ANP and BNP mRNA expression, and the down-regulation of
NPR-C mRNA expression in the heart \citep{bib3293}. According to these data, we could assume
that an acute administration of β-blockers causes an early rise in plasma CNH, while sustained treatment, significantly improving cardiac function and clinical conditions,
induces a significant fall in hormone levels \citep{bib390} \citep{bib3284} \citep{bib3287} \citep{bib3289} \citep{bib3290}.

Despite this huge number of studies suggesting the clinical relevance of monitoring patients with HF by means of CNH assay, at present only two studies \citep{bib3252} \citep{bib3253} have
been designed specifically to evaluate the clinical use of CNH assay in monitoring and
tailoring the medical therapy in patients with HF.
Murdoch et al. \citep{bib3252} studied 20 patients with mild to moderate chronic HF and
receiving stable conventional therapy, who were randomly assigned to titration of ACEinhibitor dosage, according to serial measurement of plasma BNP or to optimal empirical ACE-inhibitor therapy for 8 weeks. Only the BNP-driven approach was associated
with a significant reduction in plasma BNP concentration throughout the duration of
the study and a significantly greater suppression when compared with empiric therapy after 4 weeks (mean reduction in BNP group -42.1\%, 95\% CI -58.2, -19.7; mean reduction in empiric therapy group -12.0\%, 95\% CI CI -31.8, 13.8; P= 0.03) \citep{bib3252}.

Troughton et al. \citep{bib3253} studied 69 patients with impaired systolic function (EF <40\%)
and symptomatic HF (NYHA class II-IV), who were randomized to receive treatment
guided by either plasma NT-proBNP concentration or standardized clinical assessment.
During the follow-up (minimum 6 months, median 9.5 months), there were fewer total
cardiovascular events (death, hospital admission, or HF decompensation) in the NTproBNP-guided group than in the clinical group (19 vs 54, p = 0.02). At 6 months, 27\%
of patients in the NT-proBNP-guided group and 53\% in the clinical group had experienced a first cardiovascular event (p = 0.034). Changes in left ventricular function, quality of life, renal function, and adverse events were similar in both groups \citep{bib3253}.

Morimoto et al. \citep{bib3263} conducted a cost-effectiveness analysis of regular BNP measurement in the outpatient setting. The target population was symptomatic CHF patients
aged 35-85 years, recently discharged from the hospital. Intervention was BNP measurement once every 3 months (BNP group) or no BNP measurement (clinical group).
The baseline analysis during the 9-month period after hospitalization suggested that the
introduction of BNP measurement in heart failure management is not only cost-effective by reducing hospitalization, but also improves the outcome of patients, as assessed
by (quality-adjusted life year) analysis \citep{bib3263}.

A recent randomized clinical trial compared the titration of β-blocker therapy with bisoprolol according to plasma levels of BNP wih empiric clinical therapy based on signs and
symptoms \citep{bib3294}. Forty-one patients with heart failure were randomized into a clinical
trial. The clinical group had β-blocker dosage increased according to standard care, whereas the BNP group had β-blocker dosage up-titrated according to plasma BNP levels plus
standard care. The primary outcome was mean β-blocker dose achieved after 3 months.
BNP-guided up-titration of β-blocker in ambulatory patients with heart failure did not result
significantly different doses of β-blocker at the end of 3 months. However, 45\% of patients
in the clinical group were on the maximum dose of β-blocker vs. only 19\% of patients in
the BNP group, although left ventricular ejection fraction was significantly improved in
both groups by 7.3\%. The slightly lower doses in the BNP group were possibly better tolerated than the doses achieved in the clinical group. Furthermore, a trend toward better
quality of life was seen in the BNP group \citep{bib3294}.

Accurate diagnosis of clinical deterioration in heart failure can be difficult \citep{bib368} \citep{bib369}. To
prevent development into overt congestion, which often requires hospitalization, early
diagnosis is of paramount importance. There is a need for objective measurements to aid
early diagnosis in a setting where symptoms may be non-specific and abnormalities on
physical examination often subtle and minor \citep{bib3295}. Heart failure guidelines recommend
the use of weight gain monitoring to help in this task, with the added advantage that
patient self-care is encouraged \citep{bib368} \citep{bib369}. It is advised that an increase of 2 Kg over stable
body weight over a period of 48-72 h should initiate contact with medical or nursing
personnel \citep{bib368} \citep{bib369} \citep{bib3295}. However, Lewin et al. recently suggested that neither weight gain
nor increase in BNP are adequately sensitive as a screen for clinical deterioration in
patients with established heart failure (34 clinically stable and other 43 with clinical
deterioration of heart failure status) \citep{bib3295}. In particular, weight gain is very insensitive,
though an increase of 2 Kg demonstrates high specificity for clinical deterioration. On
the other hand, BNP change appears to provide better sensitivity than weight change, but
it has poor specificity in an established heart failure population \citep{bib3295}.

In conclusion, all the clinical trials reported above have several limitations, such as
the relatively low number of patients enrolled (usually less than 100), the limitation
due to the fact that the study was performed in a single center, and/or the non-complete group randomization. Therefore, further larger randomized, multicenter clinical
trias are necessary to definitevely demonstrate the usefulness of CNH assay in monitoring
and tailoring the medical therapy in patients with heart failure, as also indicated in a
recent conseunsus conference \citep{bib3296}. Indeed, at the time of writing this review, some
multicenter randomized clinical trials were in progress. Preliminary results of one of these
clinical trials \citep{bib3297} were recently presented an international meeting. This study suggested that the use of BNP plasma levels to guide medical therapy reduced the death and
hospital admission for heart failure as well as the delayed time to first event compared
with clinically guided treatment \citep{bib3297}.


\subsubsection{ Cost-Effectiveness of CNH Assay in Management of Patients with Acute or Chronic HF}

Several studies suggested that BNP assay may reduce the need for cardiac investigations \citep{bib365} \citep{bib369} \citep{bib3145} \citep{bib3147} \citep{bib3175}. Indeed, ruling out suspicion of HF by CNH test would make
it unnecessary to carry out other investigations, which are often time-consuming, expensive, invasive, and sometimes potentially harmful for the patient \citep{bib365} \citep{bib369}. Some studies
have been designed to test this possibility \citep{bib392} \citep{bib393} \citep{bib394} \citep{bib3147} \citep{bib3263} \citep{bib3264} \citep{bib3265}.

Nielsen et al. \citep{bib392} sought to assess the cost-effectiveness of using plasma BNP as a preechocardiographic screening test for left ventricular systolic dysfunction in the general population. Screening high-risk subjects by BNP before echocardiography could
reduce the cost per detected case of left ventricular systolic dysfunction by 26\% for the
cost ratio of 1/20 (BNP/echocardiogram). Greater reduced costs (up to 50\%) can be
predicted for the group of low-risk subjects \citep{bib392}. Other studies reported similar results
\citep{bib393} \citep{bib3145}, including a study on old people living in nursing homes \citep{bib3147}.

Mueller et al. \citep{bib3265} conducted a prospective, randomized, controlled study of 452
patients who presented to the emergency department with acute dyspnea: 225 patients
were randomly assigned to a diagnostic strategy involving the measurement of BNP,
and 227 were assessed in a standard manner. This study indicated that BNP assay improved
the evaluation and treatment of patients with acute dyspnea and thereby reduced the
time to discharge and the total cost of treatment in the emergency department \citep{bib3265}.

Other studies suggested that BNP testing can also reduce the cost for hospitalization in
patients with heart failure \citep{bib3263} \citep{bib3264}, including old people living in nursing homes \citep{bib3147}.

However, the cost-effectiveness analysis strongly depends on the relative cost of the
BNP test compared to that of echocardiograms and/or hospitalization, as well as on
the prevalence of HF in the population screened. Unfortunately, these parameters can
vary considerably among departments, countries, and health-care systems; so that each
laboratory/clinical department should analyze the cost-effectiveness in its own economical framework. Furthermore, cost-effectiveness analysis is also dependent on the
sensitivity of BNP assay for detecting HF. Cost-effectiveness will improve if more specific assays are used: this would decrease the number of subjects with false-positive
results, and consequently the number of further useless investigations. However, further
larger and randomized clinical trials are necessary to confirm the clinical relevance
and cost-effectiveness of BNP assay in the follow-up of patients with HF.

\subsection{ Summary and Conclusion}

Several recent experimental and clinical studies strongly suggest that the circulating
CNH should be better considered as an index of activation of the neuro-endocrine system, rather than a marker of myocardial dysfunction, as recently reviewed \citep{bib3298}. The activation or deactivation of the CNH system is almost always the resultant of one or more
physiological or pathological changes. For this reason, the results of CNH assays must
be interpreted by taking into account clinical history and examination, as well as all
laboratory and instrumental tests. Of course, the great number of pathophysiological
mechanisms that can affect the CNH system render sometimes difficult for clinicians
to recognize the cause(s) of variations in its activity. CNH assays should be considered
as an intellectual spur for the search of pathophysiological mechanisms that can satisfactorily explain the measured variations in hormone concentrations. On the other
hand, CNH measurements add a complementary information to other instrumental
and investigative tests \citep{bib3296}.

The clinical relevance of CNH measurement (especially BNP and NT-proBNP assay)
as diagnostic and prognostic marker in patients with cardiac disease have been recently confirmed \citep{bib35} \citep{bib372} \citep{bib376} \citep{bib3186} \citep{bib3296} \citep{bib3299} \citep{bib3300}. In populations with higher prevalence of cardiac diseases, including only individuals with clinical suspicion of HF, the diagnostic sensitivity and negative prediective value of BNP/NT-proBNP assay are both 95\%, or even
more, as long as appropriate cut-off values are selected \citep{bib35} \citep{bib372} \citep{bib3296}. This is the rationale for choosing the BNP/NT-proBNP assay as the first step for an algorithm for the
diagnosis of HF together with ECG and/or chest radiography \citep{bib369} \citep{bib370} \citep{bib3296} \citep{bib3299} \citep{bib3300}.

Such a clinical strategy has been proved successful in some recent studies evaluating the
cost-effectiveness of using plasma BNP measurements for screening of cardiac dysfunction in the general population \citep{bib392} \citep{bib393} \citep{bib394}.

The prognostic value of CNH assay for cardiac and non-cardiac death or major cardiovascular events is now well demonstrated in patients with acute or chronic HF \citep{bib35} \citep{bib372} \citep{bib376} \citep{bib3186} \citep{bib3296} \citep{bib3299} \citep{bib3300}, as well as with stable artery coronary disease or ACS \citep{bib35} \citep{bib376} \citep{bib3117} \citep{bib3138} \citep{bib3223} \citep{bib3224} \citep{bib3225} \citep{bib3226} \citep{bib3227} \citep{bib3228} \citep{bib3229} \citep{bib3230} \citep{bib3231} \citep{bib3232} \citep{bib3233} \citep{bib3234} \citep{bib3235} \citep{bib3236} \citep{bib3237} \citep{bib3238} \citep{bib3239} \citep{bib3240} \citep{bib3241} \citep{bib3242}. An increasing number of studies indicate that BNP/NT-proBNP concentration is also an independent risk factor for mortality (cardiac and/or total) in non-cardiac disease, including pulmonary embolism and hypertension, renal failure, septic
shock, amyloidosis, sarcoidosis and diabetes mellitus (see also Chapter 7 for more
details), as recently reviewed \citep{bib3298}. Increased CNH levels in these non-cardiac diseases
are a useful indication for the clinician that the heart is “under stress”.
Despite this huge number of studies suggesting the clinical relevance of monitoring patients with HF by means of CNH assay \citep{bib35}, at present, only two studies \citep{bib3252} \citep{bib3253}
were designed to specifically evaluate the clinical use of CNH assay in monitoring and
tailoring the medical therapy in patients with HF. We hope that the results of larger
randomized, multicenter clinical trias, now in progress, are able to confirm the usefulness of CNH assay in monitoring and tailoring the medical therapy in patients with
heart failure \citep{bib3296}. Indeed, preliminary results of a multicenter, randomized clinical
trial \citep{bib3297} have recently suggested that the use of BNP plasma levels to guide medical
therapy reduced the death and hospital admission for heart failure as well as the delayed
time to first event compared with clinically guided treatment.


\bibliography{lit}

\end{document}
